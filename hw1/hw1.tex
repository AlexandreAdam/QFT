\documentclass{article}
\usepackage[a4paper, margin=2cm]{geometry}

\usepackage[utf8]{inputenc} % allow utf-8 input
\usepackage[T1]{fontenc}    % use 8-bit T1 fonts
\usepackage{hyperref}       % hyperlinks
\hypersetup{
    colorlinks,
    linkcolor={red!50!black},
    citecolor={blue!50!black},
    urlcolor={blue!80!blue}
}
\usepackage{xcolor}
\usepackage{nicefrac}       % compact symbols for 1/2, etc.
\usepackage{bm, bbm}

\usepackage{amsmath}
\usepackage{amssymb}
\usepackage{mathtools}
\usepackage{amstext}
\usepackage{amsthm}
\usepackage{physics}
\usepackage{tensor}

\usepackage{fancyhdr}
\pagestyle{fancy}
\rhead{Alexandre Adam \\ 20090755}
\lhead{Théorie des champs 1 (PHY 6812) \\ William Witczak-Krempa}
\chead{Devoir 1}
\rfoot{\today}
\cfoot{\thepage}

\numberwithin{equation}{section}
\renewcommand\thesubsection{\alph{subsection})}
\renewcommand\thesubsubsection{\Roman{subsubsection}}
\DeclareRobustCommand{\bbone}{\text{\usefont{U}{bbold}{m}{n}1}}

%\newtheorem*{remark}{Remark}
\theoremstyle{solution}
\newtheorem{solution}{Réponse}[section]

\renewcommand*{\proofname}{Solution}

\begin{document}
\section{Un peu de relativité}

\subsection{}
\textbf{Classiquement, un photon se propageant dans le vide peut-il se désintégrer en une paire
électron-positron ? 
%Expliquez votre résultat mathématique de manière heuristique. 
Que ce passera-t-il en théorie quantique des champs (QFT) ?}
\subsection{}
\textbf{Quelle est la condition pour qu’un tenseur soit invariant sous transformations de Lorentz ?}
\vspace{2ex}

Soit un tenseur $T$ de rang $(m,n)$, avec les éléments $T\indices{^{\mu_1\dots \mu_m}_{\nu_1\dots\nu_n}}$, et la métrique de Minkowsky $\eta$. Soit une transformation de 
Lorentz $\Lambda \in \mathrm{SO}^{+}(1,3)$ définit tel que
\begin{equation}
        \eta^{\mu \nu} = \Lambda\indices{^{\mu}_\alpha} \Lambda\indices{^{\nu}_\beta} \eta^{\alpha\beta}\, .
\end{equation} 
On requiert que la transformation de Lorentz ait un déterminant unité puisqu'elle doit satisfaire la condition
\begin{equation}
        \Lambda(-v) = \Lambda(v)^{-1}\, ,
\end{equation} 
où $v$ est le paramètre de la transformation (vitesse pour un boost, angle pour une rotation). Puisque 
$\Lambda(-v) = R(\pi)^{T} \Lambda(v) R(\pi) \implies \mathrm{det}(\Lambda(-v)) = \mathrm{det}(\Lambda(v))$, il suit que, 
$\mathrm{det}(\Lambda) = \pm 1$. Finalement, on requiert que $\Lambda(0) = \bbone$, la matrice identité, donc $ \mathrm{det}(\Lambda(0)) = 1\implies \mathrm{det}(\Lambda(v)) = 1$. 

L'inverse transposée de la transformation $(\Lambda^{-1})\indices{^{\mu}_{\nu}} = \Lambda\indices{_\mu^{\nu}}$ agit sur les indices covariants et est définit 
tel que
\begin{equation}
        \eta_{\mu \nu}  = \Lambda\indices{_{\mu}^{\alpha}}\Lambda\indices{_{\nu}^{\beta}}\eta_{\alpha \beta}
\end{equation} 
Pour que $T$ soit un invariant de Lorentz, il doit donc satisfaire
\begin{equation}
        \boxed{ T\indices{^{\mu_1\dots\mu_m}_{\nu_1\dots\nu_n}} = \Lambda\indices{^{\mu_1}_{\alpha_1}} \dots \Lambda\indices{^{\mu_m}_{\alpha_m}} 
        \Lambda\indices{_{\nu_1}^{\beta_1}}\dots \Lambda\indices{_{\nu_n}^{\beta_n}} T\indices{^{\alpha_1\dots \alpha_m}_{\beta_1 \dots \beta_n}}} \, .
\end{equation}

\subsection{}
\textbf{Les tenseurs $\delta_{\mu}^{\nu}$ et  $\epsilon^{\mu\nu\rho\sigma}$ (Levi-Civita) sont-ils invariants de Lorentz ?}
\vspace{2ex}
\begin{solution}
        Le delta de Kronecker $\delta_\mu^{\nu}$ est un invariant de Lorentz.
\end{solution}
\begin{proof}
On remarque que $\delta^{\nu}_{\mu} = \eta_{\mu \alpha} \eta^{\alpha\nu}$. Puisque la métrique de Minkowsky est une invariant de 
Lorentz (par définition), alors $\delta^{\nu}_{\mu}$ l'est aussi.
\end{proof}
%\begin{proof}[Solution alternative]
%Soit $\Lambda$ une transformation de Lorentz. Alors,
%\begin{align}
        %(\delta')^{\nu}_\mu &= \Lambda\indices{^{\nu}_{\alpha}} \Lambda\indices{^{\beta}_{\mu}} \delta^{\beta}_{\alpha} \\
%\end{align}
        
%\end{proof}

\begin{solution}
Le tenseur de Levi-Civita est un invariant de Lorentz.
\end{solution}
\begin{proof}
On applique une transformation de Lorentz sur le tenseur de Levi-Civita
\begin{align*}
        (\epsilon')^{\mu\nu\rho\sigma} = \Lambda\indices{^{\mu}_{\alpha}} \Lambda\indices{^{\nu}_{\beta}} \Lambda\indices{^{\rho}_{\gamma}} \Lambda\indices{^{\sigma}_{\delta}} \epsilon^{\alpha\beta\gamma\delta }\, .
\end{align*} 
On examine maintenant certains éléments du tenseur transformé. Je note $S_n^{+} = \{\sigma \in S_n \mid \mathrm{sgn}(\sigma) = +1\}$ les permutations paires 
pour $n$ indices, où $\mathrm{sgn}(\sigma)$ dénote le signe du déterminant de la matrice associé à la permutation $\sigma$.
Par exemple, $S_3^{+} = \{(1\, 2\, 3),\, (3\, 1\, 2), (2\, 3\, 1)\}$; la matrice associé 
à $\sigma = (3\, 1\, 2)$ est
\begin{equation*}
M_{\sigma} = \begin{pmatrix}
        0 & 0 & 1 \\
        1 & 0 & 0 \\
        0 & 1 & 0
\end{pmatrix}\, .
\end{equation*} 
Elle possède un déterminant positif, $\mathrm{det}(M_\sigma) = 1$, de sortes que $\mathrm{sgn}(\sigma) = +1$. 
L'ensemble des permutations impaires est noté $S_n^{-}$. Naturellement, $|S_n^{+}| = |S_n^{-}| = \frac{n!}{2}$. 
En utilisant la propriété complètement antisymétrique du tenseur de Levi-Civita
\begin{equation}
        \epsilon^{\mu\nu\rho\sigma} = \begin{cases}
                +1 & \mu\nu\rho\sigma \in S_4^{+} \\
                -1 & \mu\nu\rho\sigma \in S_4^{-} \\
                0 & \mathrm{autrement}
        \end{cases}\, ,
\end{equation} 
on trouve que
\begin{equation}
        (\epsilon')^{\mu\nu\rho\sigma} = 
        \sum_{\alpha\beta\gamma\delta \in S_4^{+}} 
 \Lambda\indices{^{\mu}_{\alpha}} \Lambda\indices{^{\nu}_{\beta}} \Lambda\indices{^{\rho}_{\gamma}} \Lambda\indices{^{\sigma}_{\delta}} 
-
        \sum_{\alpha\beta\gamma\delta \in S_4^{-}} 
 \Lambda\indices{^{\mu}_{\alpha}} \Lambda\indices{^{\nu}_{\beta}} \Lambda\indices{^{\rho}_{\gamma}} \Lambda\indices{^{\sigma}_{\delta}} \, .
\end{equation} 

\paragraph{Cas des indices identiques:} Sous cette forme, on peut résoudre les valeurs du tenseur transformé lorsqu'au moins 2 indices sont identiques. Dans ce cas, 
$(\epsilon')^{\mu\nu\rho\sigma} = \epsilon^{\mu\nu\rho\sigma} = 0$. Par exemple, 
considérons le cas $\mu\nu\rho\sigma = 0012$, où les deux premiers indices sont identiques.
\begin{align*}
        (\epsilon')^{0012} &= 
        \sum_{\alpha\beta\gamma\delta \in S_4^{+}} 
 \Lambda\indices{^{0}_{\alpha}} \Lambda\indices{^{0}_{\beta}} \Lambda\indices{^{1}_{\gamma}} \Lambda\indices{^{2}_{\delta}} 
-
        \sum_{\alpha\beta\gamma\delta \in S_4^{-}} 
 \Lambda\indices{^{0}_{\alpha}} \Lambda\indices{^{0}_{\beta}} \Lambda\indices{^{1}_{\gamma}} \Lambda\indices{^{2 }_{\delta}}  \\
                           &= \dots + 
 \Lambda\indices{^{0}_{0}} \Lambda\indices{^{0}_{1}} \Lambda\indices{^{1}_{2}} \Lambda\indices{^{2}_{3}} 
-
\Lambda\indices{^{0}_{1}} \Lambda\indices{^{0}_{0}} \Lambda\indices{^{1}_{2}} \Lambda\indices{^{2}_{3}} 
-\dots \\
&= 0\, .
\end{align*}
En effet, pour chaque permutations $\alpha\beta\gamma\delta \in S_4^{+}$, on peut échanger $\alpha$ et $\beta$, 
de sortes qu'on peut toujours trouver le terme $\beta\alpha\gamma\delta \in S_4^{-}$ dans la somme 
sur les indices impaires qui annule le terme $\alpha\beta\gamma\delta$ dans la première somme.
Cet argument se généralise à tout autre cas où au moins deux indices parmis $\mu\nu\rho\sigma$ sont identiques.

\paragraph{Cas des indices distincts:} 
Je poursuis la démonstration avec l'exemple $\mu\nu\rho\sigma=0123$. Pour le résoudre, on doit démontrer que 
$(\epsilon')^{0123} = \mathrm{det}(\Lambda) = 1$.  La formule de Leibniz pour le déterminant nous indique que
\begin{equation}\label{eq:Leibniz}
        \mathrm{det}(\Lambda) = \Lambda\indices{^{0}_{\alpha}}\Lambda\indices{^{1}_{\beta}}\Lambda\indices{^{2}_{\gamma}}\Lambda\indices{^{3}_{\delta}}\epsilon^{\alpha\beta\gamma\delta}\, .
\end{equation} 
Le côté droit de l'égalité est précisément le résultat de la transformation de Lorentz pour l'élement $0123$ du tenseur de Levi-Civita, 
donc $(\epsilon')^{0123} = \mathrm{det}(\Lambda) = 1$. 
Les autres cas suivent par la permutation des indices contravariant au côté droit de la formule de Leibniz. 
Par exemple, supposons que $\mu\nu\rho\sigma = 1032 \in S_4^{+}$:
\begin{align*}
        (\epsilon')^{1032} &= \Lambda\indices{^{1}_{\alpha}}\Lambda\indices{^{0}_{\beta}}\Lambda\indices{^{3}_{\gamma}}\Lambda\indices{^{2}_{\delta}}\epsilon^{\alpha\beta\gamma\delta}\\
                           &= \Lambda\indices{^{0}_{\beta}}\Lambda\indices{^{1}_{\alpha}}\Lambda\indices{^{2}_{\delta}}\Lambda\indices{^{3}_{\gamma}}\epsilon^{\alpha\beta\gamma\delta}  \hspace{1cm} \{\text{Réarrangement des termes}\} \\
               &= \Lambda\indices{^{0}_{\beta}}\Lambda\indices{^{1}_{\alpha}}\Lambda\indices{^{2}_{\delta}}\Lambda\indices{^{3}_{\gamma}}\epsilon^{\beta\alpha\delta\gamma}  \hspace{1cm} \{\text{Permutation paire des indices du tenseur de Levi-Civita} \} \\
               &= \Lambda\indices{^{0}_{\alpha}}\Lambda\indices{^{1}_{\beta}}\Lambda\indices{^{2}_{\gamma}}\Lambda\indices{^{3}_{\delta}}\epsilon^{\alpha\beta\gamma\delta}  \hspace{1cm} \{\text{Redéfinition des indices factices} \} \\
               &= \mathrm{det}(\Lambda) = 1\, .\\ 
\end{align*} 
Comme l'argument est général, on a que $(\epsilon')^{\mu\nu\rho\sigma} = \epsilon^{\mu\nu\rho\sigma},\,\, \forall \mu\nu\rho\sigma \in S_4^{+}$.
L'argument pour une permutation impaire est très similaire et nous mène à conclure que 
\begin{equation}
(\epsilon')^{\mu\nu\rho\sigma} = -\mathrm{det}(\Lambda) = \epsilon^{\mu\nu\rho\sigma}, \,\, \forall \mu\nu\rho\sigma \in S_4^{-}\, .
\end{equation} 
Donc, ayant couvert tous les cas possible, on conclut que
\begin{equation}
        (\epsilon')^{\mu\nu\rho\sigma} = \epsilon^{\mu\nu\rho\sigma}\, .
\end{equation} 
\end{proof}
        

\section{Invariance d'échelle}
Soit un champ scalaire Klein-Gordon $\phi$ de masse $m$ en $d$ dimensions spatiales. Considérons une transformation continue
\begin{align}
        \label{eq:transformation}
        x &\rightarrow bx \\
        \phi(x) &\rightarrow b^{-\Delta}\phi(x)  
\end{align}
où $b \in \mathbb{R}_{>0}$ et $\Delta \in \mathbb{R}$.

\subsection{}
\textbf{Quelle sont les conditions pour que \eqref{eq:transformation} soit une symétrie de la théorie? Appelons l’action
de cette théorie $S_{\star}$. Quel est le courant de Noether associé?}
\vspace{2ex}

La transformation \eqref{eq:transformation} est une symétrie de la théorie si l'action 
\begin{equation}\label{eq:action}
        S_\star = \int d^{d+1}x\, \mathcal{L}_\star(\phi(x), \partial_{\mu}\phi(x))
\end{equation} 
est invariante sous l'application de la transformation. Puisque l'élément de volume devient
\begin{equation}
        d^{d+1}x \rightarrow b^{d+1}(d^{d+1}x)\, ,
\end{equation} 
alors la transformation est une symétrie de la théorie si et seulement si
\begin{equation}
       \mathcal{L}_{\star} \rightarrow b^{-(d+1)}\mathcal{L_{\star}} + \partial_{\mu}\mathcal{J}^{\mu}\, .
\end{equation} 
La divergence $\partial_{\mu}\mathcal{J}^{\mu}$ devient un terme de surface dans l'intégrale de l'action, qui 
s'annule par définition de la dérivée fonctionnelle de l'action.
%\paragraph{Cas $m=0$:} 
%On considère en premier lieu le Lagrangien d'un champ sans masse. 
Sachant que la dérivée se transforme comme
\begin{equation}
        \partial^{\mu} \rightarrow b^{-1}\partial^{\mu}\, ,
\end{equation}
alors on peut montrer que le Lagrangien de Klein-Gordon se transforme comme
\begin{equation}\label{eq:lagrangian transform}
        \mathcal{L}_\star \rightarrow  \mathcal{L}_\star' = b^{-2-2\Delta}\frac{1}{2}(\partial_{\mu}\phi)(\partial^\mu \phi) -m^2b^{-2\Delta}\phi^2 %= b^{-2-2\Delta}\mathcal{L}_\star\, .
\end{equation} 
Pour factoriser le facteur d'échelle, on doit absolument avoir $\boxed{m=0}$, de sortes que
\begin{equation}\label{eq:lagrangian transform}
        \mathcal{L}_\star \rightarrow  \mathcal{L}_\star' = b^{-2-2\Delta}\frac{1}{2}(\partial_{\mu}\phi)(\partial^\mu \phi) = b^{-2-2\Delta}\mathcal{L}_\star\, .
\end{equation} 
Ainsi, la condition pour que la théorie possède une symétrie d'échelle est
\begin{equation}\label{eq:Delta}
        \boxed{\Delta = \frac{d - 1}{2}}\, .
\end{equation} 
Pour déterminer le courant de Noether, on doit considérer la transformation infinitésimale. On pose $b = 1 - \epsilon$, où $\epsilon \rightarrow 0$. 
On écrit la variation infinitésimale du champ, $\delta \phi$, comme une transformation active de $\phi$:
\begin{equation}
       \delta \phi = \phi'(x) - \phi(x) \, ,
\end{equation} 
où
\begin{align*}
        \phi'(x) &= b^{-\Delta}\phi(b^{-1}x) \\
              &= (1 + \epsilon\Delta) \phi(x + \epsilon x) + \mathcal{O}(\epsilon^{2}) \\
              &= \phi(x) + \epsilon(\Delta + x^{\mu}\partial_{\mu})\phi(x) + \mathcal{O}(\epsilon^{2})\, .
\end{align*}
Donc,
\begin{equation}
        \delta \phi = \epsilon(\Delta + x^\mu \partial_\mu)\phi\, .
\end{equation} 
Puisque l'opérateur $\partial_\mu$ commute avec la variation infinitésimale du champ $\phi$, on peut écrire
\begin{equation}
         \delta (\partial_\mu \phi) = \partial_{\mu}\delta\phi =  \epsilon(\Delta + 1 + x^{\mu}\partial_\mu)\partial_\mu \phi + \mathcal{O}(\epsilon^2)
\end{equation} 
La variation infinitésimal du Lagrangien devient alors
\begin{align*}
        \delta  \mathcal{L}_\star &= (\partial^\mu \phi)(\partial_\mu \delta \phi) \\
                &= \epsilon (\Delta + 1 + x^\mu \partial_\mu)(\partial^\mu \phi) (\partial_\mu \phi) \\
                &= \epsilon (\mathcal{L}_\star + x^\mu \partial_\mu \mathcal{L}_\star) + \epsilon \Delta \mathcal{L}_\star\\
                &= \epsilon \partial_\mu (x^\mu \mathcal{L}_\star) + \epsilon \Delta \mathcal{L}_\star\, ,
\end{align*}
où on a absorber le facteur numérique dans l'infinitésimale et où on a laisser tomber la notation $\mathcal{O}(\epsilon^2)$ qui est sous-entendue. 
On trouve ainsi que le Lagrangien ne varie que par le facteur d'échelle, $\epsilon\Delta \mathcal{L}_\star$, qui laisse l'action invariante lorsque \eqref{eq:Delta} est respectée, et un terme de surface. 
On peut donc construire un courant de Noether
\begin{equation}
        j^{\mu} = \frac{\partial \mathcal{L}}{\partial (\partial_{\mu}\phi)} \delta  \phi  - \mathcal{J}^{\mu}\, ,
\end{equation} 
où on trouve, à l'aide de notre expression pour $\delta \mathcal{L}_\star$,
\begin{equation}
        \mathcal{J}^{\mu} = x^{\mu}\mathcal{L}_\star\, .
\end{equation} 
Ainsi
\begin{equation}\label{eq:Noether current}
        \boxed{j^{\mu}_\star = (\partial^{\mu} \phi)(\Delta + x^{\mu}\partial_{\mu})\phi - x^{\mu}\mathcal{L}_\star}
\end{equation} 


\subsection{}
\textbf{Soit une quantité $\mathcal{O}(x)$ qui dépend du champ et ses dérivées au point $x$. Posons que $\mathcal{O}$ 
transforme comme $\phi$ sous \eqref{eq:transformation}, mais avec $\Delta$ remplacé par $\Delta_\mathcal{O}$, appelé la dimension d’échelle
de $\mathcal{O}$. Pour les conditions trouvées en a), quelle est la dimension d’échelle de la densité
Lagrangienne $\mathcal{L}_{\star}$ et de $\phi^{n}$, où $n \in \mathbb{N}$}.
\vspace{2ex}

Avec les conditions trouvées en \textbf{a)}, on a $\boxed{\Delta_{\mathcal{L}_\star} = d+1}$ et $\boxed{\Delta_{\phi^{n}} = \frac{n(d-1)}{2}}$

\subsection{}
\textbf{On considère le Lagrangien avec un terme d'interaction}
\begin{equation}\label{eq:Lagrangien2c}
       \mathcal{L}_{\mathrm{int}} = \mathcal{L}_{\star} - \lambda \phi^{2n} 
\end{equation} 
\textbf{où $n \in \mathbb{N}$}.
\textbf{Quel doit être le signe de $\lambda$ pour que la théorie soit physiquement 
raisonnable? Quelle est l’équation du mouvement pour cette théorie interagissante. Quelle est la nouvelle difficulté?}

\vspace{2ex}

Le signe de $\lambda $ gouverne la forme du potentiel $V(\phi) = \frac{1}{2}m^{2}\phi^{2} + \lambda \phi^{2n}$. 
Dans le cas où $\lambda > 0$, alors le potentiel possède un état fondamental avec une énergie minimale, où $\phi = 0$. 
Dans le cas où $\lambda < 0$, alors on peut, en principe, avoir des états avec une énergie arbitrairement négative.
Donc il n'y a pas d'état fondamental. Ainsi, on doit avoir $\lambda \in \mathbb{R}_{>0}$.
L'équation du mouvement du champ $\phi(x)$ gouverné par $\mathcal{L}_{\mathrm{int}}$ est
\begin{equation}
        \boxed{\Box\phi + \frac{\partial V}{\partial \phi} = \partial^\mu \partial_{\mu} \phi + m^2\phi + 2n\lambda \phi^{2n-1} = 0}
\end{equation} 
La difficulté est de résoudre cette équation différentielle pour $\phi$, qui n'est plus une combination linéaire d'ondes planes. 

\subsection{}
\textbf{En $d \in \{1,2,3\}$ dimensions, quelles sont les conditions pour que la théorie interagissante soit invariante
sous une transformation d’échelle \eqref{eq:transformation}.}
\vspace{2ex}

Le terme d'interaction rajouté, $V_{\mathrm{\phi}}(\phi) = \lambda \phi^{2n}$, se transforme comme
\begin{equation}
        V_{\mathrm{int}} \rightarrow b^{-n(d-1)}V_{\mathrm{int}}
\end{equation} 
puisque $\Delta_{\phi^{n}} = \frac{n(d-1)}{2}$.
Pour que le Lagrangien se transforme comme $\mathcal{L}_{\mathrm{int}} \rightarrow b^{-(d+1)}\mathcal{L}_{\mathrm{int}}$, on doit avoir $m=0$ et 
\begin{equation}
        \boxed{n = \frac{d+1}{d-1}}
\end{equation} 
Ainsi, il n'y a pas de symétrie d'échelle pour les théories interagissantes avec $d=1$ dimension. 
Pour $d=2$, la théorie interagissante avec $n=3$ possède une symétrie d'échelle. Pour $d=3$, la théorie interagissante 
avec $n=2$ possède une symétrie d'échelle.

\section{Champs de jauge}
Soit l'action de Maxwell
\begin{equation}\label{eq:MaxwellAction}
       S = \int d^{d+1}x\, F_{\mu\nu}F^{\mu\nu}
\end{equation} 
où $d \geq 1$.
\subsection{}
\textbf{Démontrer que l’action est invariante sous transformations de jauge : $A_{\mu}(x) \rightarrow A_{\mu}(x) +
\partial_{\mu}f(x)$, où $f$ est une fonction scalaire suffisamment lisse.}
\vspace{2ex}

%Par définition du tenseur électromagnétique
%\begin{equation}\label{eq:F}
        %F_{\mu\nu} = \partial_{\mu}A_\nu - \partial_{\nu}A_\mu\, ,
%\end{equation} 
%la transformation de jauge induit 
%\begin{equation}
        %F_{\mu\nu} \rightarrow F_{\mu \nu} + \partial_{\mu}\partial_{\nu}f - \partial_{\nu}\partial_{\mu}f\, .
%\end{equation} 
\begin{proof}
Soit le Lagrangien de Maxwell
\begin{equation}
        \mathcal{L} = -\frac{1}{4}F^{\mu \nu}F_{\mu\nu} %= -\frac{1}{2}(\partial^{\mu}A^{\nu})(\partial_{\nu}A_\mu) + \frac{1}{2}(\partial_{\mu}A^{\mu})^{2}
\end{equation} 
où $F_{\mu\nu}$ est le tenseur électromagnétique % It is antisymmetric, with a nul trace
\begin{equation}\label{eq:F}
        F_{\mu\nu} = \partial_{\mu}A_\nu - \partial_{\nu}A_\mu\, .
\end{equation} 
Pour déterminer si la transformation de l'action est invariante sous la transformation de jauge, on doit s'assurer que $\delta S = 0$.
On commence par étudier la variation infinitésimal du Lagrangien
\begin{equation}
        \delta \mathcal{L} = -\frac{1}{2}F^{\mu \nu} \delta F_{\mu \nu}
\end{equation} 
La variation infinitésimale de $F_{\mu \nu}$ sous la transformation de jauge est
\begin{align*}
        \delta F_{\mu \nu} &= \partial_{\mu}\delta A_{\nu} - \partial_{\nu} \delta A_\mu \\
        &= \partial_{\mu}\partial_\nu f - \partial_{\nu} \partial_\mu f \\
        &= 0\, .
\end{align*}
La dernière ligne est une conséquence du théorème de Schwarz, qui dicte que les dérivée secondes d'une fonction scalaire sont symétriques dans une région $\Omega \subset \mathbb{R}^{d+1}$ qui 
contient un point $\mathcal{O} \in \Omega$ où la fonction $f:\Omega \rightarrow \mathbb{R}$ possède des dérivées secondes continues (en d'autre mots, la fonction est lisse). 
Il suit que l'action est invariante sous la transformation de jauge, 
$\delta S = \int d^{d+1} \delta L = 0$.
\end{proof}

\subsection{}
\textbf{Démontrer que l’action est invariante sous transformations de Lorentz. Est-ce qu’il existe
un terme de masse pour le champ $A_{\mu}$ qui serait invariant de jauge et invariant de Lorentz?}
\vspace{2ex}
\begin{proof}
Soit la transformation de Lorentz
\begin{equation}\label{eq:transf1}
        x^{\mu} \rightarrow \Lambda\indices{^\mu_\nu} x^\nu \, .
\end{equation} 
On étudie maintenant comment le Lagrangien se transforme. Pour se faire, on doit déterminer comment une 
dérivée se transforme. On trouve
\begin{equation}
        \partial_\mu \rightarrow \partial_\mu' = \frac{\partial }{\partial (\Lambda\indices{^\mu_\nu} x^\nu)} = (\Lambda^{-1})\indices{^{\nu}_\mu}\partial_{\nu}= \Lambda\indices{_\mu^\nu}\partial_{\nu}\, ,
\end{equation} 
où on a introduit l'inverse transposée de la transformation de Lorentz $(\Lambda^{-1})\indices{^{\mu}_{\nu}} = \Lambda\indices{_\mu^{\nu}}$. Avec ce résultat, 
on déduit la règle de transformation pour le vecteur covariant $A_{\mu}$
\begin{equation}
        A_{\mu} \rightarrow  \Lambda\indices{_{\mu}^{\nu}}A_{\nu}\, .
\end{equation} 
On calcule ensuite la transformation du tenseur électromagnétique
\begin{align}
        F_{\mu\nu}\rightarrow F^{'}_{\mu\nu} &=\Lambda\indices{_{\mu}^{\rho}}\Lambda\indices{_{\nu}^{\sigma}}\partial_{[\rho} A_{\sigma]}\, .
\end{align}
Finalement, on calcule la transformation du Lagrangien. Notons que les vecteurs contravariants $A^{\mu}$ et $\partial^{\mu}$ se transforme 
exactement comme \eqref{eq:transf1}, de sortes que
\begin{align*}
        \mathcal{L} \rightarrow \mathcal{L}' &= -\frac{1}{4}(F')^{\mu\nu}(F')_{\mu\nu} \\
        &= -\frac{1}{4}(\Lambda\indices{^{\mu}_{\alpha}}\Lambda\indices{^{\nu}_{\beta}}\partial^{[\alpha}A^{\beta]})(\Lambda\indices{_{\mu}^{\rho}}\Lambda\indices{_{\nu}^{\sigma}}\partial_{[\rho} A_{\sigma]}) \\
        &= -\frac{1}{4}\delta_\alpha^{\rho}\delta_\beta^{\sigma}(\partial^{[\alpha}A^{\beta]})(\partial_{[\rho} A_{\sigma]}) \\
        &= -\frac{1}{4}(\partial^{[\alpha}A^{\beta]})(\partial_{[\alpha} A_{\beta]}) \\
        &= -\frac{1}{4}F^{\alpha\beta}F_{\alpha\beta} \\
        &= \mathcal{L}\, .
\end{align*}
Donc l'action est invariante sous une transformation de Lorentz
\begin{equation}
        S \rightarrow S' = \int d^{d+1}x\, \mathcal{L}' = \int d^{d+1}x\, \mathcal{L} = S\, .
\end{equation} 

%Oui si le terme de masse est une transformation de jauge... 
%\begin{solution}
      %Le champ $B_\mu = A_\mu + ...$ est aussi un invariant de jauge de Lorentz et possède un terme de masse. 
%\end{solution}
%\begin{proof}
        
%\end{proof}

\end{proof}
\subsection{}
\textbf{Déterminer les équations du mouvement de la théorie. Est-ce qu’elles correspondent aux
équations classiques de Maxwell? Si oui, sous quelles conditions?}
\vspace{2ex}

%Comme le Lagrangien ne dépend que du tenseur électromagnétique, on a que les équations du mouvements sont proportionnelles à
%\begin{equation}
        %\partial_{\mu}\left( \frac{\partial L}{\partial F_{\mu\nu}} \right) \propto \boxed{-\partial_{\mu}F^{\mu\nu} = 0}
%\end{equation} 
%En principe, on devrais prendre la dérivée en terme de $\partial_{\mu}A_\nu$ pour obtenir les bons préfacteur (puisqu'on a normaliser le Lagrangien par $\frac{1}{4}$). 
%Toutefois, ce raccourcis nous donne la bonne équation du mouvement.
On commence par calculer
\begin{align*}
        \frac{\partial \mathcal{L}}{\partial (\partial_{\mu}A_\nu)} &= -\frac{1}{4}\frac{\partial }{\partial (\partial_{\mu}A_\nu)} (\partial^{[\rho}A^{\sigma]})(\partial_{[\rho}A_{\sigma]}) \\
                &= -\frac{1}{2}\frac{\partial }{\partial (\partial_{\mu}A_\nu)} \big((\partial^{\rho}A^{\sigma})(\partial_{\rho}A_\sigma) - (\partial^{\rho}A^{\sigma})(\partial_{\sigma}A_\rho) \big)\\
                &= -(\partial^{\rho}A^{\sigma})\delta_\rho^{\mu}\delta_\sigma^{\nu} + (\partial^{\rho}A^{\sigma})\delta_\sigma^{\mu}\delta_\rho^{\nu}\\
                &= -(\partial^{\mu}A^{\nu}) + (\partial^{\nu}A^{\mu})\\
                &= -F^{\mu\nu}\, .
\end{align*}
D'où les équations du mouvement
\begin{equation}\label{eq:eq mouvement}
        \boxed{\partial_{\mu}F^{\mu\nu} = 0 }\, .
\end{equation} 

On peut maintenant vérifier que cette équation nous permet de retrouver les équations classiques de Maxwell. On pose
\begin{equation}
        A^{\mu} = (\phi, \mathbf{A})\, ,
\end{equation} 
et
\begin{equation}
        \partial^{\mu} = (\partial_{t}, -\grad)\, .
\end{equation} 
On utilise les définitions $\mathbf{E} = -\grad \phi - \partial_t \mathbf{A}$ et $\mathbf{B} = \grad \times \mathbf{A}$ pour les 
champs électriques et magnétiques respectivement.
On commence par l'équation correspondant avec $\nu=0$.
On trouve que
\begin{align*}
        \partial_{\mu}F^{\mu 0} &= \partial_{\mu} \partial^{\mu}\phi - \partial_{\mu}\partial_{t}A^{\mu} \\
                &= (\partial_{t}^{2} - \grad^{2})\phi - (\partial_{t}^{2}\phi - \grad \cdot \partial_{t}\mathbf{A}) \\
                &= \grad \cdot (-\grad \phi - \partial_t \mathbf{A}) \\
                &= \grad \cdot \mathbf{E}\, .
\end{align*}
D'où $\grad \cdot \mathbf{E} = 0$, la première équation de Maxwell dans le vide.
On poursuit avec $\nu = i \in \{1,2,3\}$:
\begin{align*}
        \partial_{\mu}F^{\mu i} &= \partial_{\mu}\partial^{\mu}A^{i} - \partial_{\mu}\partial^{i}A^{\mu} \\
                &= (\partial_{t}^{2} - \grad^{2})A^{i} - \partial_{t}\partial^{i}\phi - \partial_{j}\partial^{i}A^{j} \\
                &= (\partial_{t}^{2} - \grad^{2})A^{i} - \partial_{t}\partial^{i}\phi - \partial^{i}\partial_j A^{j}\, ,
\end{align*}
où la dernière ligne suit du théorème de Schwarz. 
On additionne ensuite les équations pour chaque indice $i$, de sortes que 
\begin{align*}
        0 &= \partial_{t}^{2}\mathbf{A} - \grad^{2}\mathbf{A} + \partial_{t}\grad \phi + \grad (\grad \cdot \mathbf{A}) \\
        \implies 0 &= -\partial_{t}\mathbf{E} + \grad \times \mathbf{B}\, ,
\end{align*}
où on a utiliser l'identité vectorielle $\grad^{2}\mathbf{A} = \grad (\grad \cdot \mathbf{A}) - \grad \times (\grad \times \mathbf{A})$. On trouve finalement la loi d'Ampère dans le vide. 
%La seconde loi de Maxwell suit de l'identité vectorielle $\grad \cdot (\grad \times \mathbf{A}) = 0$.

%TODO: a quelles condition


\subsection{}
\textbf{Est-ce que la théorie de Maxwell est invariante sous une transformation d’échelle? Si
oui, quelle sont les conditions appropriées, ainsi que les dimensions d’échelle du champ de jauge
et du champ électrique.}
\vspace{2ex}

Soit la transformation d'échelle
\begin{equation}
\begin{split}
        x &\rightarrow  bx \\
        A^{\mu} &\rightarrow b^{-1}A^{\mu}\,. 
\end{split}
\end{equation} 
Le Lagrangien de Maxwell se transforme comme
\begin{equation}
        \mathcal{L} \rightarrow b^{-4}\mathcal{L}
\end{equation} 
Ce qui laisse invariant l'action
\begin{equation}
        S \rightarrow S' =  b^{d+1 - 4}\int d^{d+1}x \,\mathcal{L} = b^{d-3}S\, ,
\end{equation} 
\textbf{seulement si $\boxed{d=3}$.} La dimension d'échelle du champ de jauge $A^{\mu}$ est $\boxed{\Delta_{A^{\mu}} = 1}$, de sortes que la 
dimension d'échelle du champ électrique se doit d'être $\boxed{\Delta_{\mathbf{E}} = 2}$ puisque
\begin{equation}
        \Delta_{\mathbf{E}} = \Delta_{\partial^{\mu}} + \Delta_{A^{\mu}}
\end{equation} 



\subsection{}
\textbf{Quel est le champ canoniquement conjugué à $A_\mu$ . 
\vspace{2ex}
%Est-ce qu’il y a quelque chose de bizarre
%avec votre réponse ?
}

\section{Phonons}
\subsection{}
\textbf{En vous basant sur l’analyse faite en classe en 1+1 dimensions, obtenez la théorie classique des champs décrivant les vibrations d’un cristal cubique en 2 et 3 dimensions spatiales.
Utilisez l’approche "vache sphérique" : posez une forme simple pour l’énergie potentielle
d’élongation-compression qui généralise le cas en d = 1. Le vrai cas des phonons est plus
difficile à traiter}
\subsection{}
\textbf{Quelles sont les symétries continues de cette théorie en 2+1 et 3+1 dimensions ? Travaillez avec les symétries internes seulement, 
c’est-à-dire les symétries ne faisant pas intervenir d’opération sur l’espace-temps. Déterminez les courants de Noether associés.}

\section{Quantification 101}
%Considérons la chaîne harmonique classique en 1 dimension spatiale, telle que vue en classe.
\subsection{}
%Faites la quantification canonique de la chaîne discrète. Vous aurez alors un système
%quantique non-relativiste avec un nombre infini, mais discret, de degrés de libertés quan-
%tiques. Nous sommes donc encore dans le contexte de la mécanique quantique habituelle.
%Rappel : {A, B} → −i[ Â, B̂], avec ~ = 1. Utilisez le point de vue de Schrödinger.
\subsection{}
%Prenez ensuite la limite du continu de cette théorie. Donnez toutes les relations de com-
%mutation entre les opérateurs de champ issus des opérateurs φ̂ I et p̂ I . Ces commutateurs
%respectent-ils la causalité ? Ne vous inquiétez pas trop de la rigueur mathématique de l’ex-
%pansion de Taylor pour des opérateurs.
\subsection{}
%Évaluez ∂ t 2 h φ̂(x)i t à l’aide de la théorie discrète. Passez au continu à la fin du calcul.
%Votre résultat est-il relié à l’équation de Klein-Gordon ? Expliquez.

\end{document}

