\documentclass{article}
\usepackage[a4paper, margin=2cm]{geometry}

\usepackage{amsmath}
\usepackage{amssymb}
\usepackage{mathtools}
\usepackage{amstext}
\usepackage{amsthm}
\usepackage{fancyhdr}
\usepackage[utf8]{inputenc} % allow utf-8 input
\usepackage[T1]{fontenc}    % use 8-bit T1 fonts


\usepackage{graphicx}
\usepackage{float}
\usepackage{caption}
\usepackage{subcaption}
\usepackage{booktabs}
\usepackage{physics, tensor}
\usepackage{slashed, cancel}

\graphicspath{{figures/}}

\pagestyle{fancy}
\rhead{Alexandre Adam\\ 20090755}
\lhead{William Witczak-Krempa \\ PHY 6812: Théorie des champs 1}
\chead{Devoir 3}
\rfoot{23 novembre 2022}
\cfoot{\thepage}

\newcommand{\angstrom}{\textup{\AA}}
\numberwithin{equation}{section}
\renewcommand\thesubsection{\alph{subsection})}
\renewcommand\thesubsubsection{\Roman{subsubsection}}
\newcommand{\s}{\hspace{0.1cm}}
\DeclareRobustCommand{\bbzero}{\text{\usefont{U}{bbold}{m}{n}0}}
\DeclareRobustCommand{\bbone}{\text{\usefont{U}{bbold}{m}{n}1}}


\theoremstyle{solution}
\newtheorem{solution}{Réponse}[section]

\renewcommand*{\proofname}{Solution}


\begin{document}
\section{Représentations projectives}
%On cherche à montrer que les spineurs de Weyl et Dirac forment des représentations projectives du groupe
%de Lorentz. 
%En d’autres termes, montrez que ces représentations admettent au moins un 2-cocycle non-trivial.
\begin{solution}
Les spineurs de Weyl, et par extension la représentation réductible de Dirac, 
sont des représentations projectives du groupe de Lorentz $SO(3, 1)$ puisqu'elles admettent 
une phase $e^{i\phi(\Lambda_1, \Lambda_2)} = \pm 1 $, soit
\begin{equation}
        M(\Lambda_1) M(\Lambda_2) = \pm M(\Lambda_1 \Lambda_2)
\end{equation} 
qui ne fait pas partie du groupe d'équivalence triviale ($\phi = 0$). 
\end{solution}
\begin{proof}
La théorie des représentations du groupe de Lorentz $SO(3, 1)$ indique les spineurs de Weyl se transforme comme
\begin{equation}
        \begin{split}
                \psi_{L}(x) &\rightarrow M_{\frac{1}{2}, 0}(\Lambda) \psi_L(x) \\
                \psi_R(x) &\rightarrow M_{0, \frac{1}{2}}(\Lambda) \psi_R(x)
        \end{split}
\end{equation} 
où les matrices $M$ font partie des groupes de Lie $(\frac{1}{2}, 0)$ et $(0, \frac{1}{2})$ respectivement. En particulier, 
pour les transformations de Lorentz connectées de façon continue à l'identité $\Lambda \in SO^{+}(3, 1)$, on a
\begin{equation}\label{eq:rep1}
        \begin{split}
                M_{\frac{1}{2}, 0} &= \exp \left( -i \boldsymbol{ \theta}\cdot \frac{\boldsymbol{ \sigma} }{2} - \boldsymbol{ \beta} \cdot \frac{\boldsymbol{ \sigma} }{2} \right) \\
                M_{0, \frac{1}{2}} &= \exp(-i \boldsymbol{ \theta} \cdot \frac{\boldsymbol{ \sigma} }{2} + \boldsymbol{ \beta} \frac{\boldsymbol{ \sigma} }{2}  )
        \end{split}
\end{equation} 
pour $\boldsymbol{ \theta}, \boldsymbol{ \beta} \in \mathbb{R}^{3}$. Pour construire le 2-cocycle, on considère une 
transformation de la transformation vers son inverse $M \rightarrow  -M$. Étant donner la représentation \eqref{eq:rep1}, 
on peut construire une rotation $\theta_3 = \pi$ dans la direction de $\sigma^{3}$,
\begin{equation}
        M(\Lambda(\pi)) = \exp \left( -i \frac{\pi}{2}\sigma^{3} \right) = -i\sigma^{3}\, .
\end{equation} 
Ceci implique que 
\begin{equation}
        M(\Lambda(\pi))M(\Lambda(\pi)) = -1\, .
\end{equation} 
Or, 
\begin{equation}
       M(\Lambda(\pi)\Lambda(\pi)) = M(\Lambda(2\pi)) = M(\bbone) = 1 \, .
\end{equation} 
De sortes que
\begin{equation}
        M(\Lambda(\pi))M(\Lambda(\pi)) = - M(\Lambda(\pi)\Lambda(\pi))\, .
\end{equation} 
La phase introduite par cette transformation, $e^{i\phi(\pi, \pi)} = -1$, est irréductible, en ce sens qu'on ne peut pas redéfinir la 
représentation pour prendre en compte le signe qui apparaît après une rotation de $2\pi$ des spineurs de Weyl.
Ainsi, les spineurs de Weyl forment une représentation projective du groupe de Lorentz.
%%On commence par construire une représentation honnête du groupe de Lorentz dans la base de Weyl. On note  
%%l'effet d'une transformation de Lorentz
%%\begin{equation}
        %%\begin{split}
        %%\psi_{L}(x) &\rightarrow M_{\frac{1}{2}, 0}(\Lambda) \psi_L(x) \\
        %%\psi_R(x) &\rightarrow M_{0, \frac{1}{2}}(\Lambda) \psi_R(x)
        %%\end{split}
%%\end{equation} 
%%où $M_{\frac{1}{2}, 0}(\Lambda)$ et $M_{0, \frac{1}{2}}(\Lambda)$ sont des représentations du groupe de Lorentz $SO(3, 1)$ dans 
%%les groupes de Lie irréductibles $(\frac{1}{2}, 0)$ et $(0, \frac{1}{2})$. Pour déterminer une représentation des matrices $M(\Lambda)$, 
%%on utilise le groupe universel de $SO(3, 1)$, soit $SL(2, \mathbb{C})$
%\begin{equation}
       %M: SO(3, 1) \rightarrow  SL(2, \mathbb{C}) 
%\end{equation} 
%où
%\begin{equation}
        %SL(2, \mathbb{C}) = 
        %\left\{ \begin{pmatrix}
                %a & b \\
                %c & d
        %\end{pmatrix}
        %: (a,b,c,d) \in \mathbb{C}^{4}, ad - bc = 1
         %\right\}
%\end{equation} 
%On utilise le théorème de décomposition polaire pour exprimer $M \in SL(2, \mathbb{C})$
%\begin{equation}
        %M = \lambda e^{h}
%\end{equation} 
%où $\lambda \in SU(2)$ et $h \in \mathfrak{su}(2) = \mathrm{span}(i\sigma^{1}, i\sigma^{2}, i\sigma^{3})$. La construction du 2-cocycle se fait alors 
%en notant que les matrices
\end{proof}

\section{Charge d’un spineur}
\subsection{}
On considère une transformation interne agissant sur un spineur de Dirac
\begin{equation}\label{eq:t_interne}
	\begin{split}
		\psi(x) &\rightarrow e^{i\theta}\psi(x)
	\end{split}
\end{equation} 
où $\theta \in \mathbb{R}$. On cherche à savoir si cette transformation est une symétrie du Lagrangien de Dirac
\begin{equation}\label{eq:dirac_L}
	\mathcal{L} = \bar{\psi}(i \slashed{\partial} - m)\psi
\end{equation} 

\begin{solution}
La transformation interne \eqref{eq:t_interne}, soit une transformation $U(1)$, est une symétrie du Lagrangien de Dirac.	
\end{solution}
\begin{proof}
La preuve suit par calcul direct
\begin{align*}
        \mathcal{L} \rightarrow \mathcal{L}'  &= \bar{\psi}'  (i \slashed{\partial} - m) \psi' \\
                &= e^{-i\theta} \psi^{\dagger} \gamma^0(i \slashed{\partial} - m) e^{i\theta} \psi \\
                &= \bar{\psi} (i \slashed{\partial} - m)  \psi \\
                &= \mathcal{L}
\end{align*}	
On a utilisé le fait que $\theta \in \mathbb{R}$ est un paramètre constant et global, donc $e^{-i\theta}$ commute avec 
la dérivée $\partial_\mu$.
\end{proof}

\begin{solution}
Le courant de Noether associé à la transformation interne \eqref{eq:t_interne} est (à une phase $\pm 1$ près)
\begin{equation}
        \boxed{j^{\mu} = \bar{\psi} \gamma^{\mu} \psi}
\end{equation} 
\end{solution}
\begin{proof}
Soit la transformation interne infinitésimale
\begin{equation}
        \psi \rightarrow  (1 - i \theta) \psi
\end{equation} 
La transformation infinitésimale du spineur de Dirac est donc 
\begin{equation}
        \delta \psi = -i \psi
\end{equation} 
Il suit de la définition du courant de Noether que
\begin{align*}
        j^{\mu} &= \frac{\partial \mathcal{L}}{\partial (\partial_{\mu} \psi)} \delta \psi \\
        &= \left(  \frac{\partial }{\partial (\partial_{\mu}\psi)} \bar{\psi}(i \gamma^{\nu} \partial_\nu - m) \psi\right) \delta\psi\\
        &= \bar{\psi}\gamma^{\nu} \delta^{\mu}_\nu \psi \\
        &= \bar{\psi}\gamma^{\mu}\psi
\end{align*} 
\end{proof}


\subsection{}
\begin{solution}
Lorsque $m = 0$, les représentations chirales sont découplées et on obtient deux courants de Noether, une 
pour les spineurs de chiralité droite ($s=R$) et une pour les spineurs de chiralité gauche ($s=L$), soit 
une symétrie $U(1) \times U(1)$:
\begin{equation}
        \boxed{j^{\mu}_s = \bar{\psi}_s \sigma^{\mu}_s \psi_s}
\end{equation} 
où $\sigma_R^{\mu} = \sigma^{\mu}$ et $\sigma_L^{\mu} = \bar{\sigma}^{\mu}$, et $\sigma^{\mu} = (\bbone, \sigma^{i})$.\\ 
Dans la limite ultra relativiste $m \rightarrow  0$, le courant de Noether est plutôt
\begin{equation}
        \boxed{j^{\mu} = \bar{ \psi} \gamma^{\mu}\gamma^{5} \psi}\, .
\end{equation} 
Les représentations chirales gauche et droite sont corrélées et spin en sens inverse.
\end{solution}
\begin{proof}
        
On considère le Lagrangien de Dirac avec masse nulle, $m = 0$. 
On utilise la représentation de Weyl (ou représentation chirale) pour représenter les spineurs, soit
\begin{equation}
        \psi = \begin{pmatrix}
                \psi_L \\ \psi_R
        \end{pmatrix}\, ,
\end{equation} 
et
\begin{equation}
        \gamma^0 = \begin{pmatrix}
                \bbzero & \bbone\\
                \bbone& \bbzero
        \end{pmatrix}
        , 
        \hspace{1cm}
        \gamma^{i} = 
        \begin{pmatrix}
                \bbzero & \sigma^{i} \\
                \bar{\sigma}^{i}& \bbzero
        \end{pmatrix}\, .
\end{equation} 
où $\sigma^{i}$ sont les matrices de Pauli et $\bar{\sigma} = -\sigma^{i}$.
Dans ce cas, une symétrie du Lagrangien émerge du fait que les 
représentations chirales se découplent dans l'équation de Dirac. En effet, on obtient les équations de Weyl
\begin{equation}
         i\gamma^{\mu}\partial_\mu \psi = 
         \begin{pmatrix}
                 i \sigma^{\mu} \partial_\mu \psi_R \\[1ex]
                 i \bar{\sigma}^{\mu}\partial_\mu \psi_L 
         \end{pmatrix}
         = 0\, .
\end{equation} 
De ce fait, le Lagrangien se découple en deux termes
\begin{align*}
        \mathcal{L} &=  i\bar{\psi} \gamma^{\mu}\partial_\mu \psi \\
        &= i
        \begin{pmatrix}
                \psi_L^{\dagger} \\
                \psi_R^{\dagger}
        \end{pmatrix}
        \gamma^0
         \begin{pmatrix}
                 \sigma^{\mu} \partial_\mu \psi_R \\[1ex]
                 \bar{\sigma}^{\mu}\partial_\mu \psi_L 
         \end{pmatrix} \\
         &= 
        \begin{pmatrix}
                \psi_R^{\dagger} \\
                \psi_L^{\dagger}
        \end{pmatrix}
         \begin{pmatrix}
                 \sigma^{\mu} \partial_\mu \psi_R \\[1ex]
                 \bar{\sigma}^{\mu}\partial_\mu \psi_L 
         \end{pmatrix} \, .
\end{align*}
D'où
\begin{equation}
        \mathcal{L} = i \psi_L^{\dagger} \bar{\sigma}^{\mu} \partial_\mu \psi_L + i \psi_r^{\dagger} \sigma^{\nu}\partial_\nu \psi_R \, .
\end{equation}
Ainsi, pour des particules de masses nulles, le Lagrangien a maintenant deux symétries internes, soit $U(1)\times U(1)$, donc deux 
courants de Noether
\begin{equation}
        j^{\mu}_s = \bar{\psi}_s \sigma^{\mu}_s \psi_s\, ,
\end{equation} 
où $\sigma^{\mu}_R = \sigma^{\mu}$ et $\sigma^{\mu}_L = \bar{\sigma}^{\mu}$. Ces courants sont valides strictement lorsque les représentations chirales 
sont indépendantes, et n'est donc pas valide dans une limite ultra relativiste. 

En effet, la symétrie du Lagrangien dans la limite où $m \rightarrow  0$, soit une limite ultra relativiste, est légèrement différente. 
Dans ce cas, on peut construire la symétrie axiale
\begin{equation}\label{eq:t_axiale}
        \psi(x) \rightarrow e^{i \theta \gamma^{5}} \psi(x) \, ,
\end{equation} 
où, dans la représentation de Weyl, 
\begin{equation}
        \gamma^{5} \equiv i\gamma^0\gamma^{1}\gamma^{2}\gamma^{3} = 
        i
        \begin{pmatrix}
                -\sigma^{1}\sigma^{2}\sigma^{2} & 0 \\
                0 & \sigma^{1}\sigma^{2}\sigma^{3}
        \end{pmatrix}
        =
        \begin{pmatrix}
                -\bbone & 0 \\ 
                0 & \bbone
        \end{pmatrix}\, .
\end{equation} 
On peut montrer que cette transformation est une symétrie du Lagrangien de Dirac en utilisant le fait que $(\gamma^{5})^{\dagger} = \gamma^{5}$, 
$(\gamma^{5})^{2} = \bbone$ et 
la formule d'Euler qui résulte de ce fait
\begin{equation}
        e^{i\theta \gamma^{5}} = \cos(\theta) + i \gamma^{5}\sin(\theta)
\end{equation} 
On utilise aussi la relation d'anticommutation
\begin{equation}
        \{ \gamma^{5}, \gamma^{\mu} \} = 0\, ,
\end{equation} 
de sortes que $e^{i \theta \gamma^{5}} \gamma^{\mu} = \gamma^{\mu}e^{-i \theta \gamma^{5}}$ et
\begin{align*}
        \mathcal{L} \rightarrow  \mathcal{L}' &=\psi^{\dagger} e^{-i \theta \gamma^{5}}\gamma^0  (i\gamma^\mu \partial_\mu - m) e^{i\theta \gamma^{5}}\psi \\
                &=\psi^{\dagger} \gamma^0  e^{i \theta \gamma^{5}}(i\gamma^\mu \partial_\mu - m) e^{i\theta \gamma^{5}}\psi \\
                &= \bar{\psi} (i\gamma^\mu \partial_\mu - me^{2i\theta \gamma^{5}}) \psi \\
\end{align*}
Ainsi, dans la limite $m \rightarrow 0$, la transformation axiale \eqref{eq:t_axiale} est une symétrie du Lagrangien, qui correspond effectivement 
à deux symétries $U(1)$ où les paramètres des transformations correspondent à $\theta_R = -\theta_L$.  
On peut calculer le courant de Noether qui correspond à cette symétrie. On considère la transformation infinitésimale
\begin{equation}
        \psi(x) \rightarrow  (1 - i \theta \gamma^{5}) \psi(x)\, ,
\end{equation} 
d'où
\begin{equation}
        \delta \psi = -i \gamma^{5}
\end{equation} 
et
\begin{equation}
        j^{\mu} = \bar{\psi} \gamma^{\mu}\gamma^{5}\psi
\end{equation} 
\end{proof} 



\section{Invariance d’échelle}
On considère le champ de Dirac avec $d=3$ dimensions spatiales. On considère la transformation d'échelle
\begin{equation}\label{eq:t_echelle}
        \begin{split}
                x &\rightarrow  b x \\
                \psi(x) &\rightarrow b^{-\Delta} \psi(x)
        \end{split}
\end{equation} 
où $b \in \mathbb{R}_{>0}$ et $\Delta \in \mathbb{R}$.

\subsection{}
\textbf{Quelle sont les conditions pour que \eqref{eq:t_echelle} soit une symétrie de la théorie? Appelons l’action
de cette théorie $S_{\star}$. Quel est le courant de Noether associé?} \\
On ne considère que le cas avec $d = 3$ dimensions spatiales.
La transformation \eqref{eq:t_echelle} est une symétrie de la théorie si l'action 
\begin{equation}\label{eq:action}
        S_\star = \int d^{4}x\, \bar{\psi}(x)(i \slashed{\partial} - m)\psi(x)
\end{equation} 
est invariante sous l'application de la transformation. Puisque l'élément de volume devient
\begin{equation}
        d^{4}x \rightarrow b^{4}(d^{4}x)\, ,
\end{equation} 
et que l'opérateur $\slashed{\partial} \rightarrow b^{-1} \slashed{\partial}$ possède une facteur d'échelle $\Delta_{\slashed{\partial}} = 1$, alors 
la transformation d'échelle est une symétrie du Lagrangien si et seulement si
\begin{equation}
        \boxed{
        \begin{split}
              m = 0 \\
              \Delta_\psi = \frac{3}{2}
        \end{split}
}
\end{equation} 
On pose $b = 1 - \epsilon$, où $|\epsilon| \ll 1$. Ainsi, 
\begin{equation}
        \psi(x) \rightarrow  (1 + \epsilon\Delta ) \psi(x)
\end{equation} 
On prend le point de vue d'une transformation active du champ
\begin{equation}
        \delta \psi = \psi'(x) - \psi(x) \, .
\end{equation} 
On doit donc déterminer la quantité
\begin{align*}
        \psi'(x) &=  (1 + \epsilon\Delta ) \psi(x + \epsilon x) + \mathcal{O}(\epsilon^{2}) \\
\end{align*}
Or, pour simplifier ce terme, on doit déterminer la dérivée de Lie du champs spineur $\psi(x)$, ce qui est non-trivial.
\section{Théorie Yukawa classique}
Soit la densité lagrangienne pour un champ scalaire réel $\phi$ et un spineur de Dirac $\psi$
\begin{equation}
        \mathcal{L} = \frac{1}{2}\partial^{\mu}\phi \partial_\mu \phi - \frac{m^{2}}{2}\phi^{2} - \lambda \phi^{4} + \bar{\psi}(i \slashed{\partial} - M) \psi - g \phi\bar{\psi}\psi
\end{equation} 
où $m, M$ sont des masses et $\lambda,g$ sont des constantes de couplage.

\subsection{}
\textbf{Est-ce une théorie relativiste?}\\
\begin{solution}
       La théorie de Yukawa classique est une théorie relativiste. 
\end{solution}
\begin{proof}
Soit une transformation de Lorentz $\Lambda \in SO(3, 1)$. La théorie de Yukawa classique est relativiste si 
la densité lagrangienne est un scalaire de Lorentz. Par l'étude des densité lagrangienne de Klein-Gordon,  
on sait que les trois premiers termes sont des scalaires de Lorents. En effet, le champ scalaire $\phi$ est un invariant de Lorentz 
et, puisque $\Lambda\indices{^{\rho}_{\nu}}\Lambda\indices{_{\rho}^{\mu}} = \delta^{\mu}_\nu$, 
\begin{align*}
       \partial_\mu \phi \partial^{\mu}\phi &\rightarrow \Lambda\indices{_{\mu}^{\rho}}\partial_\rho \phi \Lambda\indices{^{\mu}_{\sigma}}\partial^{\sigma}\phi \\
                &\rightarrow  \delta^{\rho}_\sigma \partial_\rho \phi\partial^{\sigma}\phi \\
                &\rightarrow  \partial_\mu \phi \partial^{\mu}\phi  \\
\end{align*} 
On sait aussi que $\phi \overset{\Lambda}{\rightarrow}  \phi$.
La densité lagrangienne de Dirac est aussi un invariant de Lorentz, sachant que 
\begin{equation}
        \begin{split}
                \psi \rightarrow \Lambda_D \psi \\
                \bar{\psi} \rightarrow \bar{\psi}\Lambda_D^{-1}
        \end{split}
\end{equation} 
où $\Lambda_D \in SO(3, 1)$ est une transformation de Lorentz dans la représentation de Dirac $(\frac{1}{2}, 0) \oplus (0, \frac{1}{2})$, 
et que les matrices $\gamma^{\mu}$ se transforme comme
\begin{equation}
        \Lambda_D^{-1} \gamma^{\mu} \Lambda_D = \Lambda\indices{^{\mu}_{\nu}}\gamma^{\nu}\, ,
\end{equation} 
on obtient 
\begin{align*}
       \mathcal{L}_D \rightarrow \mathcal{L}'_D &= \bar{\psi}\Lambda_D^{-1}(i\gamma^{\mu} \Lambda\indices{_{\mu}^{\nu}}\partial_\nu - M)\Lambda_D \psi \\
                 &= \bar{\psi}(i \Lambda_D^{-1}\gamma^{\mu} \Lambda_D \Lambda\indices{_{\mu}^{\nu}}\partial_\nu - M)\psi \\
                 &= \bar{\psi}(i \Lambda\indices{^{\mu}_{\rho}}\gamma^{\rho}\Lambda\indices{_{\mu}^{\nu}}\partial_\nu - M)\psi \\
                 &= \bar{\psi}(i \delta^{\nu}_\rho\gamma^{\rho}\partial_\nu - M)\psi \\
                 &= \mathcal{L}_D
\end{align*}
Finalement,
\begin{equation}
        \bar{\psi}\psi \rightarrow \bar{\psi}\Lambda_D^{-1}\Lambda_D \psi = \bar{\psi} \psi
\end{equation} 
Donc
\begin{equation}
        \mathcal{L} \overset{\Lambda}{\rightarrow} \mathcal{L} 
\end{equation} 
\end{proof}

\subsection{}
\textbf{On cherche les équations du mouvement pour la théorie de Yukawa classique}.\\
\begin{solution}
       L'équation de mouvement pour $\phi$ est
       \begin{equation}
               \boxed{\partial_\mu\partial^{\mu} \phi = -m^{2} \phi - 4\lambda \phi^{3} - g \bar{ \psi} \psi }
       \end{equation} 
\end{solution}
\begin{proof}
La preuve suit par calcul direct. On a
\begin{equation}
        \frac{\partial L}{\partial \phi} = -m^{2}\phi - 4\lambda \phi^{3} - g \bar{\psi}\psi\, ,
\end{equation} 
et
\begin{equation}
        \frac{\partial L}{\partial (\partial_\mu \phi)} = \partial^{\mu}\phi\, .
\end{equation} 
De sortes que
\begin{equation}
        \frac{\partial \mathcal{L}}{\partial \phi} - \partial_\mu \left( \frac{\partial \mathcal{L}}{\partial (\partial_\mu \phi)} \right)
        = -m^{2}\phi - 4\lambda \phi^{3} -g \bar{\psi} \psi - \partial_\mu \partial^{\mu} \phi
        = 0
\end{equation} 
\end{proof}
\begin{solution}
       L'équation de mouvement pour $\psi$ est
       \begin{equation}
               \boxed{i \slashed{\partial} \bar{\psi} = (g \phi - M) \bar{\psi}}
       \end{equation} 
\end{solution}
\begin{proof}
La preuve suit par calcul direct. On a
\begin{equation}
        \frac{\partial \mathcal{L}}{\partial \psi} = g\phi \bar{\psi} - M \bar{\psi}\, ,
\end{equation} 
et
\begin{equation}
        \frac{\partial \mathcal{L}}{\partial (\partial_\mu \psi)} = i \bar{\psi} \gamma^{\mu}\, .
\end{equation} 
D'où
\begin{equation}
        \frac{\partial \mathcal{L}}{\partial \psi} - \partial_\mu \left( \frac{\partial \mathcal{L}}{\partial (\partial_\mu \psi)} \right)
        =
        g\phi \bar{\psi} - M \bar{\psi}
        -
         i\partial_\mu \gamma^{\mu}\bar{\psi} 
        = 0
\end{equation} 
\end{proof}

\begin{solution}
       L'équation de mouvement pour $\bar{\psi}$ est 
       \begin{equation}
               \boxed{(i\slashed{\partial} - M - g\phi) \psi = 0 }
       \end{equation} 
\end{solution}
\begin{proof}
On a que
\begin{equation}
        \frac{\partial \mathcal{L}}{\partial \bar{\psi}} = (i\slashed{\partial} - M) \psi - g \phi \psi\, ,
\end{equation} 
et
\begin{equation}
        \frac{\partial \mathcal{L}}{\partial (\partial_\mu \bar{\psi})} = 0\, ,
\end{equation} 
d'où
\begin{equation}
        \frac{\partial \mathcal{L}}{\partial \bar{\psi}} = (i \slashed{\partial} - M) \psi - g \phi \psi = 0\, .
\end{equation} 
\end{proof}

\subsection{}
\textbf{On cherche les conditions sous lesquelles la théorie de Yukawa classique possède une invariance d'échelle.}
\begin{solution}
Les conditions pour que la théorie de Yukawa classique soit invariante d'échelle sont
\begin{equation}
        \boxed{
        \begin{split}
        \Delta_{\phi} &= 1 \\
        \Delta_{\psi} &=  \frac{3}{2} \\
        M &= m = 0 \\
        \end{split}
}
\end{equation} 
\begin{proof}
Chaque terme du Lagrangien doit posséder un facteur d'échelle total de $4$ pour que la théorie soit invariant d'échelle. 
Ainsi,

        %\mathcal{L} = \frac{1}{2}\partial^{\mu}\phi \partial_\mu \phi - \frac{m^{2}}{2}\phi^{2} - \lambda \phi^{4} + \bar{\psi}(i \slashed{\partial} - M) \psi - g \phi\bar{\psi}\psi
\begin{equation}
        \frac{1}{2}\partial^{\mu}\phi \partial_\mu \phi \rightarrow b^{-2-2\Delta_{\phi}}  \frac{1}{2}\partial^{\mu}\phi \partial_\mu \phi \implies \Delta_{\phi} = 1 
\end{equation} 
et il suit que $m = 0$, autrement le terme $\propto m^{2} \phi^2$ n'a pas le bon facteur d'échelle. Comme 
le terme $\lambda \phi^{4}$ a le bon facteur d'échelle, alors les conditions $\Delta_{\phi} = 1$ et $m = 0$ sont suffisantes 
et nécessaires pour les 3 premiers termes. Comme dérivé au numéro précédent, le Lagrangien de Dirac impose que
\begin{equation}
        \Delta_{\psi} = \frac{3}{2}\hspace{1cm} M = 0
\end{equation} 
On remarque finalement que le terme de couplage
\begin{equation}
        g \phi \bar{\psi} \psi \rightarrow b^{-4} g\phi \bar{\psi}\psi
\end{equation} 
avec les conditions trouvées. 
\end{proof}
        
\end{solution}

\subsection{}
\textbf{On prend $M = 0$ et $\phi(x) = \phi_0$, une constante de l'espace-temps. On pose aussi $|g| \ll 1$. On cherche une solution pour $\phi_0$.}
%\begin{solution}
%\begin{equation}
        %\boxed{\phi_0 = }
%\end{equation} 
%\begin{proof}
        
%\end{proof}
        
%\end{solution}

\section{Opérateurs qui anti-commutent}
On considère l'Hamiltonien suivant
\begin{equation}\label{eq:hamiltonien5}
       \hat{H} = \sum_{a, b}^{N} \hat{t}_{ab}\hat{c}^{\dagger}_a \hat{c}_b \, ,
\end{equation} 
avec les relations d'anti-commutations suivantes
\begin{equation}\label{eq:commut5}
        \{\hat{c}_a, \hat{c}^{\dagger}_b \} = \delta_{ab} \hspace{1cm} \{ \hat{c}_a, \hat{c}_b\} = 0\, .
\end{equation} 

\subsection{}
\textbf{On cherche la condition nécessaire telle que $\hat{H}$ soit un opérateur hermitien.}
\begin{solution}
        Les éléments de la matrice $t_{ab}$ sont réels.
\end{solution}
\begin{proof}
Supposons que $\hat{t}_{ab} \in \mathbb{R}$. Il suit que $\hat{t}^{\dagger}_{ab} = \hat{t}_{ba}$ et
\begin{align*}
       \hat{H}^{\dagger} &=  \sum_{a,b}(\hat{t}_{ab}\hat{c}^{\dagger}_a \hat{c}_b)^{\dagger} \\
                         &= \sum_{a,b} \hat{t}_{ba} \hat{c}^{\dagger}_b \hat{c}_a \hspace{1cm} \{\hat{t}_{ab} \in \mathbb{R}\}\\
                         &= \hat{H}
\end{align*} 
où la dernière ligne suit du fait que les indices $a \leftrightarrow b$ sont symétriques.
La condition $\hat{t}_{ab} \in \mathbb{R}$ suit du fait qu'on requiert $[\hat{t}_{ab},\, \hat{c}^{\dagger}_b \hat{c}_a] = 0$ et $\hat{t}_{ab}^{*} = \hat{t}_{ab}$.
\end{proof}

\subsection{}
\begin{solution}
        La dimension de l'espace d'Hilbert de l'Hamiltonien \eqref{eq:hamiltonien5}, avec $N=1$, est de $2$. 
        L'espace de Fock fermionique à 1 particule est décrit par la base complète
        \begin{equation}\label{eq:base_fermion}
                | 0 \rangle = \begin{pmatrix}
                        1 \\ 0
                \end{pmatrix},
                \hspace{1cm} 
                | 1 \rangle = \begin{pmatrix}
                        0 \\ 1
                \end{pmatrix}
        \end{equation} 
        et les opérateurs
        \begin{equation}
                \hat{c}_1 = \begin{pmatrix}
                        0 & 1 \\
                        0 & 0
                \end{pmatrix}
                ,\hspace{1cm}
                \hat{c}^{\dagger}_1 = \begin{pmatrix}
                        0 & 0 \\
                        1 & 0
                \end{pmatrix}
        \end{equation} 
\end{solution}
\begin{proof}
On suppose $N=1$. On s'intéresse aux états stationnaires de l'Hamiltonien \eqref{eq:hamiltonien5}. Par définition, 
l'état fondamental $| 0 \rangle $ est un état stationnaire de l'Hamiltonien avec valeur propre $0$ ($\hat{c} | 0 \rangle  = 0$). 
On s'intéresse ensuite à $| 1 \rangle = \hat{c}^{\dagger}_1 | 0 \rangle $
\begin{align*}
        \hat{H}| 1 \rangle &= \hat{t}_{11} \hat{c}^{\dagger}_1 \hat{c}_1 \hat{c}^{\dagger}_1 | 0 \rangle \\
        &= \hat{t}_{11} \hat{c}^{\dagger}_1 (1 - \hat{c}^{\dagger}_1 \hat{c}_1)| 0 \rangle \\
        &= \hat{t}_{11}| 1 \rangle 
\end{align*}
où la seconde ligne suite de la relation d'anticommutation $\{\hat{c}_1, \hat{c}^{\dagger}_1\} = 1$.
Donc, l'état $| 1 \rangle $ est un état stationnaire de l'Hamiltonien avec valeur propre $\hat{t}_{11}$. 
Ensuite, on considère l'état $| 2 \rangle = \hat{c}^{\dagger}_1 \hat{c}^{\dagger}_1 | 0 \rangle $
\begin{align*}
        \hat{H} | 2 \rangle &= \hat{t}_{11} \hat{c}^{\dagger}_1 \hat{c}_1 \hat{c}^{\dagger}_1 \hat{c}^{\dagger}_1 | 0 \rangle  \\
                &= \hat{t}_{11}\hat{c}^{\dagger}_1\hat{c}^{\dagger}_1 | 0 \rangle  - \hat{t}_{11}\hat{c}^{\dagger}_1\hat{c}^{\dagger}_1 \hat{c}_1\hat{c}^{\dagger}_1 | 0 \rangle  \\
                &= \hat{t}_{11}\hat{c}^{\dagger}_1\hat{c}^{\dagger}_1 | 0 \rangle  - \hat{t}_{11}\hat{c}^{\dagger}_1\hat{c}^{\dagger}_1 | 0 \rangle  \\
                &= 0
\end{align*}
On trouve que l'état $| 2 \rangle $ possède le même niveau d'énergie que $| 0 \rangle $.
Donc il n'y a pas d'état avec deux particules. Il suit par induction 
qu'il n'y a pas d'états à $n$ particules, puisque $| 3 \rangle = | 1 \rangle $, etc.
On conclut que l'Hamiltonien \eqref{eq:hamiltonien5}, avec $N=1$, décrit un fermion puisque la théorie obéit le principe d'exclusion de Pauli.
\end{proof}
\begin{solution}
        La dimension de l'espace d'Hilbert de l'Hamiltonien \eqref{eq:hamiltonien5}, avec $N=1$ et les relations  
        d'anticommutations \eqref{eq:commut5} remplacée par des relations de commutations
        \begin{equation}
                [\hat{c}_a, \hat{c}^{\dagger}_b ] = \delta_{ab} \hspace{1cm} [ \hat{c}_a, \hat{c}_b] = 0\, ,
        \end{equation} 
        est $\infty$. L'espace de Fock bosonique à 1 particule est décrit par la base complète
        \begin{equation}
                | 0 \rangle = \begin{pmatrix}
                        1 \\ 0 \\ 0 \\\vdots
                \end{pmatrix},
                \hspace{1cm} 
                | 1 \rangle = \begin{pmatrix}
                        0 \\ 1 \\ 0 \\\vdots
                \end{pmatrix},
                \hspace{1cm} 
                | 2 \rangle = \begin{pmatrix}
                        0 \\ 0 \\ 1 \\ \vdots
                \end{pmatrix},
                \hspace{1cm}
                 \dots
        \end{equation} 
        et les opérateurs
        \begin{equation}
                \hat{c}_1 = \begin{pmatrix}
                        0 & 1 & 0 & \dots \\
                        0 & 0 & \sqrt{2} & \dots \\
                        0 & 0 & 0 & \dots \\
                        \vdots & \vdots & \vdots & \ddots
                \end{pmatrix}
                ,\hspace{1cm}
                \hat{c}^{\dagger}_1 = \begin{pmatrix}
                        0 & 0 & 0 &  \dots \\
                        1 & 0 & 0 &  \dots \\
                        0 & \sqrt{2} & 0 &  \dots  \\
                        \vdots & \vdots & \vdots & \ddots 
                \end{pmatrix}
        \end{equation} 
\end{solution}
\begin{proof}
On s'intéresse en premier lieu à $| 1 \rangle = \hat{c}^{\dagger}_1 | 0 \rangle $
\begin{align*}
        \hat{H}| 1 \rangle &= \hat{t}_{11} \hat{c}^{\dagger}_1 \hat{c}_1 \hat{c}^{\dagger}_1 | 0 \rangle \\
        &= \hat{t}_{11} \hat{c}^{\dagger}_1 (1 + \hat{c}^{\dagger}_1 \hat{c}_1)| 0 \rangle \\
        &= \hat{t}_{11}| 1 \rangle 
\end{align*}
où la seconde ligne suite de la relation d'commutation $[\hat{c}_1, \hat{c}^{\dagger}_1] = 1$.
Donc, l'état $| 1 \rangle $ est un état stationnaire de l'Hamiltonien avec valeur propre $\hat{t}_{11}$. 
Ensuite, on considère l'état $| 2 \rangle = \hat{c}^{\dagger}_1 \hat{c}^{\dagger}_1 | 0 \rangle $
\begin{align*}
        \hat{H} | 2 \rangle &= \hat{t}_{11} \hat{c}^{\dagger}_1 \hat{c}_1 \hat{c}^{\dagger}_1 \hat{c}^{\dagger}_1 | 0 \rangle  \\
                &= \hat{t}_{11}\hat{c}^{\dagger}_1\hat{c}^{\dagger}_1 | 0 \rangle  + \hat{t}_{11}\hat{c}^{\dagger}_1\hat{c}^{\dagger}_1 \hat{c}_1\hat{c}^{\dagger}_1 | 0 \rangle  \\
                &= \hat{t}_{11}\hat{c}^{\dagger}_1\hat{c}^{\dagger}_1 | 0 \rangle  + \hat{t}_{11}\hat{c}^{\dagger}_1\hat{c}^{\dagger}_1 | 0 \rangle  \\
                &= 2\hat{t}_{11}| 2 \rangle 
\end{align*}
Donc l'état $| 2 \rangle $ est un état stationnaire de l'Hamiltonien avec valeur propre $2 \hat{t}_{11}$. Il suit par 
induction que $| n \rangle $ est aussi un état stationnaire de l'Hamiltonien. On obtient donc un espace d'Hilbert 
infini qui décrit une particule bosonique. Chaque états admet $n \in \mathbb{N}$ particules.
\end{proof}

\subsection{}
On définit les opérateurs Majorana
\begin{equation}
        \hat{\chi}_1 = \frac{1}{\sqrt{2}}(\hat{c} + \hat{c}^{\dagger}), \hspace{1cm} \hat{\chi}_2= \frac{-i}{\sqrt{2}}(\hat{c} - \hat{c}^{\dagger}) 
\end{equation} 
En terme des relations d'anticommutations \eqref{eq:commut5}, on a que
\begin{align*}
        \{\hat{\chi}_1, \hat{\chi}_2\} &= \frac{-i}{2} \{\hat{c} + \hat{c}^{\dagger},\, \hat{c} - \hat{c}^{\dagger}\}  \\
                &= \frac{-i}{2}(-\{\hat{c}, \hat{c}^{\dagger}\} + \{\hat{c}^{\dagger}, \hat{c}\}) \\
\implies     \Aboxed{ \{\hat{\chi}_1, \hat{\chi}_2\} &= 0 } 
\end{align*}
On observe que les opérateurs Majorana sont hermitiens, $\hat{\chi}_{a}^{\dagger} = \hat{\chi}_a$, et que 
\begin{align*}
        \hat{\chi}_1\hat{\chi}_2 &= \frac{-i}{2}(\hat{c} + \hat{c}^{\dagger})(\hat{c} - \hat{c}^{\dagger}) \\
        &= \frac{-i}{2}(\hat{c}\hat{c} - \hat{c}^{\dagger}\hat{c}^{\dagger} + 2\hat{c}^{\dagger}\hat{c} - 1)
\end{align*}
Or comme $\hat{c}\hat{c} = \hat{c}^{\dagger}\hat{c}^{\dagger} = 0$ pour des fermions (voir équation \eqref{eq:base_fermion}), alors
\begin{equation}
        \hat{c}^{\dagger}\hat{c} = i \hat{\chi}_1\hat{\chi_2} + 1
\end{equation} 
Ainsi, on trouve l'Hamiltonien
\begin{equation}
        \boxed{\hat{H} = \hat{t}_{11}(i\hat{\chi}_1\hat{\chi}_2 + 1 )} 
\end{equation} 
Les états propres de $\hat{\chi}_1$, dans la base complète \eqref{eq:base_fermion}, sont
\begin{equation}
        \boxed{| \chi_{11} \rangle = \frac{1}{\sqrt{2}} \begin{pmatrix}
                1 \\ 1
        \end{pmatrix}
        ,
        \hspace{1cm}
        | \chi_{12} \rangle = \frac{1}{\sqrt{2}}\begin{pmatrix}
                1 \\ -1
        \end{pmatrix}
}
\end{equation} 
En effet, 
\begin{align*}
        \hat{\chi}_1 | \chi_{11} \rangle &= \frac{1}{\sqrt{2}} \hat{\chi_1}(| 0 \rangle + | 1 \rangle ) \\
                        &= \frac{1}{2} (| 1 \rangle + | 0 \rangle ) \\
                        &= \frac{1}{\sqrt{2}}| \chi_{11} \rangle \\
        \hat{\chi}_1 | \chi_{12} \rangle &= \frac{1}{\sqrt{2}}\hat{\chi}_1 (| 0 \rangle - | 1 \rangle ) \\
                &= \frac{1}{2}(| 1 \rangle - | 0 \rangle  ) \\
                &= -\frac{1}{\sqrt{2}}| \chi_{12} \rangle 
\end{align*} 
Les états propres de $\hat{\chi}_2$, dans la base complète \eqref{eq:base_fermion}, sont
\begin{equation}
        \boxed{| \chi_{21} \rangle = \frac{1}{\sqrt{2}} \begin{pmatrix}
                1 \\ i
        \end{pmatrix}
        ,
        \hspace{1cm}
        | \chi_{22} \rangle = \frac{1}{\sqrt{2}}\begin{pmatrix}
                1 \\ -i
        \end{pmatrix}
}
\end{equation} 
En effet, 
\begin{align*}
        \hat{\chi}_2 | \chi_{21} \rangle &= \frac{1}{\sqrt{2}} \hat{\chi_2}(| 0 \rangle + i| 1 \rangle ) \\
                        &= \frac{1}{2} (i| 1 \rangle - i^2| 0 \rangle ) \\
                        &= \frac{1}{\sqrt{2}}| \chi_{21} \rangle \\
        \hat{\chi}_2 | \chi_{22} \rangle &= \frac{1}{\sqrt{2}}\hat{\chi}_2 (| 0 \rangle - i| 1 \rangle ) \\
                &= \frac{1}{2}(i| 1 \rangle + i^2| 0 \rangle  ) \\
                &= -\frac{1}{\sqrt{2}}| \chi_{22} \rangle 
\end{align*} 

\subsection{}
\begin{solution}
La dimension de l'espace d'Hilbert associé à l'Hamiltonien \eqref{eq:hamiltonien5} est $2^{N}$, $N \in \mathbb{N}$. Les états 
\begin{equation}
        | n_1, n_2, \dots, n_N \rangle, \hspace{1cm} n_i \in \{0, 1\}
\end{equation} 
sont associé à une énergie
\begin{equation}
        E = \sum_{i = 1}^{N}n_{i}\hat{t}_{ii}
\end{equation} 
\end{solution}
\begin{proof}
On commence par le cas $N=2$. Dans ce cas, on peut construire les états
\begin{align*}
        | 0, 0 \rangle  = | 0 \rangle \hspace{1cm} N = 0 \\
        | 1, 0 \rangle = \hat{c}_1^{\dagger}| 0 \rangle   \hspace{.5cm} | 0, 1 \rangle = c_2^{\dagger}| 0 \rangle  \hspace{1cm} N = 1 \\
        | 1, 1 \rangle = c_1^{\dagger}c_2^{\dagger}| 0 \rangle  \hspace{1cm} N = 2
\end{align*}
On montre que $| 1, 1 \rangle $ est un état stationnaire de $\hat{H}$.
En utilisant la relation d'anticommutation $\{\hat{c}_a, \hat{c}_b^{\dagger}\} = \delta_{ab}$, on a
\begin{align*}
       \hat{H}| 1, 1 \rangle  &= \sum_{a, b} \hat{t}_{ab}\hat{c}^{\dagger}_a \hat{c}_b \hat{c}_1^{\dagger}\hat{c}_2^{\dagger} | 0 \rangle  \\
                        &= \sum_{a, b} \hat{t}_{ab} \hat{c}^{\dagger}_a(\delta_{b1} - \hat{c}_1^{\dagger} \hat{c}_b)\hat{c}^{\dagger}_2 | 0 \rangle  \\
                        &= \sum_{a, b} \hat{t}_{ab} \hat{c}^{\dagger}_{a}\hat{c}^{\dagger}_2\delta_{b1}| 0 \rangle  - \sum_{a,b}\hat{t}_{ab}\hat{c}^{\dagger}_a\hat{c}^{\dagger}_1 \delta_{b2} | 0 \rangle 
\end{align*}
En utilisant le fait que $\hat{c}_{a}^{\dagger}\hat{c}_a^{\dagger} = 0$, on obtient
\begin{equation}
        \hat{H}| 1, 1 \rangle =  (\hat{t}_{11}  - \hat{t}_{22})| 1, 1 \rangle 
\end{equation} 
Les états $| 1, 0 \rangle $ et $| 0, 1 \rangle $ ne sont pas des états stationnaires de l'Hamiltonien à moins que $\hat{t}_{12} = \hat{t}_{21}= 0$. En effet, 
par des manipulations similaires on obtient
\begin{align*}
        \hat{H} | 1, 0 \rangle &= \hat{t}_{11} |1, 0  \rangle + \hat{t}_{21}| 0, 1 \rangle 
\end{align*}
Toutefois, on remarque que la superposition $| \xi_1 \rangle  = (\hat{t}_{22} - \hat{t}_{12})| 1, 0 \rangle + (\hat{t}_{11} - \hat{t}_{12})| 0, 1 \rangle$ est un état stationnaire. 
En effet, considérons l'état $| \xi \rangle  = A | 1, 0 \rangle + B | 0, 1 \rangle $
\begin{align*}
        \hat{H}| \xi \rangle &= \sum_{a, b} \hat{t}_{ab}\hat{c}^{\dagger}_a\hat{c}_b (A\hat{c}^{\dagger}_1 + B\hat{c}^{\dagger}_2)| 0 \rangle  \\
        &= \sum_{a, b} \hat{t}_{ab}\hat{c}^{\dagger}_a(A\delta_{b1} + B\delta_{b2})| 0 \rangle  \\
        &= (A\hat{t}_{11} + B\hat{t}_{12})| 1, 0 \rangle + (A\hat{t}_{21} + B\hat{t}_{22})| 0, 1 \rangle \\
\end{align*}
$| \xi \rangle $ est un état stationnaire si et seulement si
\begin{equation}
        A = B \frac{\hat{t}_{22} - \hat{t}_{12}}{\hat{t}_{11} - \hat{t}_{21}}\, .
\end{equation} 
Le choix $A = \hat{t}_{22} - \hat{t}_{12}$ et $B = \hat{t}_{11} - \hat{t}_{21}$ est le plus simple. Ainsi, l'état $| \xi_1 \rangle $ 
est un état stationnaire avec une énergie
\begin{equation}
        E_{\xi_1} = \hat{t}_{11}\hat{t}_{22} - \hat{t}_{21}\hat{t}_{12}
\end{equation} 


%Les états stationnaires de 
%\begin{equation}
       %\hat{H}=  \sum_{a,b}\hat{t}_{ab}\hat{c}^{\dagger}_a \hat{c}_b\, ,
%\end{equation} 
%sont
%\begin{equation}
        %| n_1, n_2, \dots, n_N \rangle = \left(  \prod_{\{i | n_i = 1\}}\hat{c}^{\dagger}_i \right)|0\rangle  
%\end{equation} 
%On peut montrer que c'est un état stationnaire de l'Hamiltonien 
%\begin{align*}
       %\hat{H}| n_1, n_2 \dots, n_N \rangle &=  
%\end{align*}
%On peut construire les états stationnaires de l'Hamiltonien avec l'espace de Fock
%\begin{equation}
        %F(H) = \bigoplus_{n=1}^{N} S_{-}H^{\otimes n}
%\end{equation} 
%où $S_{-}$ est un tenseur complètement antisymétrique. Par exemple, pour $N=2$ on a 
%\begin{align*}
        %| 0, 0 \rangle \hspace{1cm} N = 0 \\
                %| 1, 0 \rangle  \hspace{.5cm} | 0, 1 \rangle \hspace{1cm} N = 1 \\
        %| 1, 1 \rangle \hspace{1cm} N = 2
%\end{align*}
%Pour $N = 3$, on a $2^{3} = 8$ états stationnaires, soit
%\begin{align*}
        %| 0, 0, 0 \rangle \hspace{1cm} N = 0 \\
        %| 1, 0, 0 \rangle  \hspace{.5cm} | 0, 1, 0 \rangle  \hspace{.5cm} | 0, 0, 1 \rangle \hspace{1cm} N = 1 \\
        %| 1, 1, 0 \rangle  \hspace{.5cm} | 0, 1, 1 \rangle  \hspace{.5cm} | 1, 0, 1 \rangle \hspace{1cm} N = 2 \\
        %| 1, 1, 1 \rangle \hspace{1cm} N = 3
%\end{align*}
%Dans la représentation choisie, un état est décrit par $N$ nombres naturels $n_j \in \{0, 1\}$, qui ne peuvent prendre que deux valeurs. Ainsi, 
%le système avec $N$ fermions est décrit par les états
%\begin{equation}
        %| n_1, n_2, \dots, n_N \rangle\, , 
%\end{equation}
%soit une base complète de l'Hamiltonien
%Par induction, l'Hamiltonien avec $N$ fermions possède $2^{N}$ états stationnaires, ce qui correspond à un espace de Fock ave
\end{proof}



\section{Fermions Majorana (Peskins \& Schroeder 3.4)}
On considère les deux composantes du champ de chiralité gauche $\chi_a(x)$, $a \in \{1, 2\}$, tels que
\begin{equation}\label{eq:main6}
        i \bar{\sigma}^{\mu}\partial_\mu \chi(x) - i m \sigma^{2}\chi^{*}(x) = 0
\end{equation} 
où $\bar{\sigma}^{\mu} = (\bbone, -\sigma^{1}, -\sigma^{2}, -\sigma^{3}) = (\bbone, -\boldsymbol{ \sigma} )$ 
est un vecteur incluant les matrices de Pauli et le second terme est appelé le terme de masse de Majorana. 

\subsection{}
On veut montrer que \eqref{eq:main6} est un invariant de Lorentz et que \eqref{eq:main6} implique l'équation de Klein-Gordon. 
Pour se faire, on déduit la relation de transformation de $\sigma^{\mu}$ et $\bar{\sigma}^{\mu}$ à partir de la 
transformation connue des matrices de Dirac 
\begin{equation}\label{eq:gam6}
        M_D(\Lambda) \gamma^{\mu}M_D(\Lambda^{-1}) = \Lambda\indices{^{\mu}_{\nu}}\gamma^{\nu}
\end{equation} 
où $M_D(\Lambda)$ est la représentation de la transformation de Lorentz dans la base de Dirac $(\frac{1}{2}, 0) \oplus (0, \frac{1}{2})$. 
On écrit maintenant la matrice $\gamma^{\mu}$ dans la base chirale de Weyl
\begin{equation}
        \gamma^{\mu} = 
        \begin{pmatrix}
                0 & \sigma^{\mu} \\
                \bar{\sigma}^{\mu} & 0
        \end{pmatrix}
\end{equation} 
On fait de même pour les matrices $M(\Lambda)$, 
qui est une matrice bloc diagonale en terme des représentations irréductibles
\begin{equation}
        M_D(\Lambda) =
        \begin{pmatrix}
                M_{\frac{1}{2},0}(\Lambda) & 0 \\
        0 & M_{0, \frac{1}{2}}(\Lambda)
        \end{pmatrix}
\end{equation} 
Il suit que
\begin{align*}
        M_D(\Lambda) \gamma^{\mu}M_D(\Lambda^{-1}) &= 
        \begin{pmatrix}
                M_{\frac{1}{2},0}(\Lambda) & 0 \\
        0 & M_{0, \frac{1}{2}}(\Lambda)
        \end{pmatrix}
        \begin{pmatrix}
                0 & \sigma^{\mu} \\
                \bar{\sigma}^{\mu} & 0
        \end{pmatrix}
        \begin{pmatrix}
                M_{\frac{1}{2},0}(\Lambda^{-1}) & 0 \\
        0 & M_{0, \frac{1}{2}}(\Lambda^{-1})
        \end{pmatrix} \\
        %&= 
        %\begin{pmatrix}
                %M_{\frac{1}{2},0}(\Lambda) & 0 \\
        %0 & M_{0, \frac{1}{2}}(\Lambda)
        %\end{pmatrix}
        %\begin{pmatrix}
                %0 & \sigma^{\mu}M_{0,\frac{1}{2}}(\Lambda^{-1}) \\
                %\bar{\sigma}^{\mu}M_{\frac{1}{2}, 0}(\Lambda^{-1}) & 0
        %\end{pmatrix} \\
        &= 
        \begin{pmatrix}
                0 & M_{\frac{1}{2},0}(\Lambda)\sigma^{\mu}M_{0,\frac{1}{2}}(\Lambda^{-1})  \\
                M_{0, \frac{1}{2}}(\Lambda)\bar{\sigma}^{\mu} M_{\frac{1}{2},0}(\Lambda^{-1}) & 0
        \end{pmatrix}
\end{align*}
Avec le côté droit de l'équation \eqref{eq:gam6}, on déduit les relations de transformations
\begin{equation}
\begin{split}
        M_{\frac{1}{2},0}(\Lambda)\sigma^{\mu}M_{0,\frac{1}{2}}(\Lambda^{-1}) &= \Lambda\indices{^{\mu}_{\nu}}\sigma^{\nu}\\
        M_{0, \frac{1}{2}}(\Lambda)\bar{\sigma}^{\mu} M_{\frac{1}{2},0}(\Lambda^{-1}) &= \Lambda\indices{^{\mu}_{\nu}}\bar{\sigma}^{\nu}
\end{split}
\end{equation} 
En réarrangeant les termes, on obtient les relations équivalentes
\begin{equation}
\begin{split}
        M_{\frac{1}{2},0}(\Lambda)\sigma^{\mu}&= \sigma^{\nu}\Lambda\indices{_{\nu}^{\mu}}M_{0,\frac{1}{2}}(\Lambda) \\
        M_{0, \frac{1}{2}}(\Lambda)\bar{\sigma}^{\mu}&= \bar{\sigma}^{\nu}\Lambda\indices{_{\nu}^{\mu}} M_{\frac{1}{2},0}(\Lambda)
\end{split}
\end{equation} 
où on a utiliser le fait que $\Lambda\indices{^{\mu}_{\nu}}A^{\nu} = A^{\nu}\Lambda\indices{_{\nu}^{\mu}}$.
On considère maintenant une transformation de Lorentz sur l'équation \eqref{eq:main6} pour la composante de chiralité gauche $(\frac{1}{2}, 0)$. 
On applique donc les transformations
\begin{equation}
        \partial_{\mu} \rightarrow \Lambda\indices{_{\mu}^{\nu}}\partial_\nu\, ,
\end{equation} 
et
\begin{equation}
        \chi(x) \rightarrow M_{\frac{1}{2},0}(\Lambda)\chi(x)\, .
\end{equation} 
de sortes que le premier terme de l'équation \eqref{eq:main6} devient
\begin{align}
        \nonumber
       i \bar{\sigma}^{\mu}\partial_\mu \chi(x) 
       &\rightarrow  i \bar{\sigma}^{\mu}   \Lambda\indices{_{\mu}^{\nu}}\partial_\nu  M_{\frac{1}{2},0}(\Lambda)\chi(x)   \\
       \nonumber
      &\rightarrow  i \Lambda\indices{^{\nu}_{\mu}}\bar{\sigma}^{\mu} M_{\frac{1}{2},0}(\Lambda)\partial_\nu  \chi(x)   \\
      \label{eq:sol61}
        \implies \Aboxed{i \bar{\sigma}^{\mu}\partial_\mu \chi(x)   &\rightarrow   M_{0,\frac{1}{2}}(\Lambda)i\bar{\sigma}^{\nu}\partial_\nu \chi(x)}
\end{align}
On cherche maintenant à déterminer comment $\sigma^{2} \chi^{*}$ se transforme.
Dans la représentation chirale, on a les éléments du groupe $SO^{+}(3, 1)$ suivant
\begin{equation}
        \begin{split}
                M_{\frac{1}{2}, 0}(\Lambda) &= \exp \left( -i \boldsymbol{ \theta} \cdot \frac{\boldsymbol{ \sigma} }{2} - \boldsymbol{ \beta} \cdot \frac{\boldsymbol{ \sigma} }{2} \right) \\
                M_{0, \frac{1}{2}}(\Lambda) &= \exp \left( -i \boldsymbol{ \theta} \cdot \frac{\boldsymbol{ \sigma} }{2} + \boldsymbol{ \beta} \cdot \frac{\boldsymbol{ \sigma} }{2} \right)
        \end{split}
\end{equation} 
où $\boldsymbol{ \theta},\boldsymbol{ \beta} \in \mathbb{R}^{3}$ et $\boldsymbol{ \sigma} = (\sigma^{1}, \sigma^{2}, \sigma^{3})$. On prouve en premier 
lieu que
\begin{equation}
        \sigma^{2} \boldsymbol{ \sigma}^{*} \sigma^{2} = -\boldsymbol{ \sigma} \, . 
\end{equation} 
En effet, en utilisant la loi de contraction $\sigma^{i}\sigma^{j} = \delta^{ij}\bbone + i\epsilon\indices{^{ij}_{k}} \sigma^{k}$, où 
$\epsilon\indices{^{ij}_{k}}$ est le tenseur de Levi-Civita, on obtient
\begingroup
\allowdisplaybreaks
\begin{align*}
        \sigma^{2} (\sigma^{1})^{*}\sigma^{2} &=  \sigma^{2} \sigma^{1}\sigma^{2} \\
                &= -i\sigma^{3} \sigma^{2} \\
                &= -\sigma^{1} \\
        \sigma^{2}(\sigma^{2})^* \sigma^{2} &= -\sigma^{2}\sigma^{2}\sigma^{2} \\
                &= -\sigma^{2} \\
        \sigma^{2} (\sigma^{3})^* \sigma^{2} &= \sigma^{2}\sigma^{3}\sigma^{2} \\
                   &= i\sigma^{1}\sigma^{2} \\
                   &= -\sigma^{3}
\end{align*}
\endgroup
Ainsi, on peut déduire la transformation de $\sigma^{2}M_{\frac{1}{2}, 0}(\Lambda)^{*} \sigma^{2}$
\begin{align*}
        \sigma^{2}M_{\frac{1}{2}, 0}(\Lambda)^{*}\sigma^{2} &=
        \sigma^{2} \exp \left( i \boldsymbol{ \theta} \cdot \frac{\boldsymbol{ \sigma} }{2} - \boldsymbol{ \beta} \cdot \frac{\boldsymbol{ \sigma} }{2} \right)\sigma^{2} \\
        &=  \exp \left( -i \boldsymbol{ \theta} \cdot \frac{\boldsymbol{ \sigma} }{2} + \boldsymbol{ \beta} \cdot \frac{\boldsymbol{ \sigma} }{2} \right) \\
        &= M_{0, \frac{1}{2}}(\Lambda)
\end{align*}
En utilisant l'identité $(\sigma^{i})^{2} = \bbone$, on trouve
\begin{equation}\label{eq:sol62}
        \begin{split}
        \sigma^{2}\chi^{*} &\rightarrow \sigma^{2}M_{\frac{1}{2}, 0}(\Lambda)^{*} (\sigma^{2})^{2}\chi^{*} \\
        \implies \Aboxed{ \sigma^{2}\chi^{*} &\rightarrow  M_{0, \frac{1}{2}}(\Lambda) (\sigma^{2}\chi^{*})}
        \end{split}
\end{equation} 
Donc $\sigma^{2}\chi^{*}$ se transforme comme un spineur de chiralité droite.
En utilisant les équation \eqref{eq:sol61} et \eqref{eq:sol62}, on obtient la loi de transformation pour \eqref{eq:main6}
\begin{equation}
        i \bar{\sigma}^{\mu}\partial_\mu \chi(x) - i m \sigma^{2}\chi^{*}(x) \rightarrow M_{0,\frac{1}{2}}(\Lambda) 
        (\underbrace{i \bar{\sigma}^{\mu}\partial_\mu \chi(x) - i m \sigma^{2}\chi^{*}(x)}_{=0})  = 0
\end{equation} 
soit une équation invariante de Lorentz, tels que requis. 

On s'intéresse maintenant à montrer que \eqref{eq:main6} implique l'équation de Klein-Gordon. On considère en premier lieu 
l'équation \eqref{eq:main6} conjugué
\begin{align*}
        \eqref{eq:main6}^* &\implies  -i (\bar{\sigma}^{\mu})^{*}\partial_\mu \chi^{*}(x) + i m (\sigma^{2})^{*}\chi(x) = 0 \\
                          &\implies -i (\bar{\sigma}^{\mu})^{*}\partial_\mu \chi^{*}(x) - i m \sigma^{2}\chi(x) = 0 \\
\end{align*} 
puisque $(\sigma^{2})^* = -\sigma^{2}$. En isolant $\chi^{*}(x)$ dans \eqref{eq:main6}
\begin{equation}\label{eq:sub6}
        \chi^{*}(x) = \frac{1}{m}\sigma^{2} \bar{\sigma}^{\mu} \partial_\mu\chi(x)\, ,
\end{equation} 
on peut substituer $\chi^{*}$ dans $\eqref{eq:main6}^{*}$ 
\begin{align}
        \nonumber
        \frac{1}{m} (\bar{\sigma}^{\mu})^{*}\partial_\mu \sigma^{2} \bar{\sigma}^{\nu} \partial_\nu \chi(x) + m \sigma^{2}\chi(x) &= 0 \\
        \nonumber
        \implies 
        \sigma^{2}\sigma^{2}(\bar{\sigma}^{\mu})^{*} \sigma^{2} \bar{\sigma}^{\nu} \partial_\mu\partial_\nu \chi(x) + m^2 \sigma^{2}\chi(x) &= 0 \\
        \nonumber
        \implies \sigma^{2}\sigma^{\mu} \bar{\sigma}^{\nu} \partial_\mu\partial_\nu \chi(x) + m^2 \sigma^{2}\chi(x) &= 0 \\
        \label{eq:res61}
        \implies \sigma^{\mu} \bar{\sigma}^{\nu} \partial_\mu\partial_\nu \chi(x) + m^2\chi(x) &= 0 
\end{align} 
puisque $\sigma^{2} (\bar{\sigma}^{\mu})^*\sigma^{2} = \sigma^{2}(\bbone, -\boldsymbol{ \sigma}^{*} )\sigma^{2} = (\bbone, \boldsymbol{ \sigma} ) = \sigma^{\mu}$. 
Les matrices de Pauli $\sigma^{\mu}$ obéissent l'algèbre de Clifford pour $\mathbb{R}^{3}$
(ce qui suit par la loi de contraction utilisée plus haut)
\begin{equation}
        \{\sigma^{i}, \sigma^{j}\} = 2 \delta^{ij}\bbone
\end{equation} 
On peut retrouver l'algèbre de Clifford pour $\mathbb{R}^{1,3}$, gouverné par la métrique de Minkowsky $\eta^{\mu \nu} = \mathrm{diag}(1, -1, -1, -1)$, 
avec l'algèbre légèrement modifié
\begin{equation}
        \{\sigma^{\mu}, \bar{\sigma}^{\nu}\} =  \{\bar{\sigma}^{\mu}, \sigma^{\nu}\} = 2 \eta^{\mu \nu} \bbone\, ,
\end{equation} 
ce qui induit la relation 
\begin{equation}
        \sigma^{\mu}\bar{\sigma}^{\nu} = \eta^{\mu \nu}\bbone \, .
\end{equation} 
Ainsi, on obtient finalement
\begin{align*}
       \eqref{eq:res61}  \implies \eta^{\mu \nu}\partial_\mu\partial_\nu \chi(x) + m^2\chi(x) &= 0  \\
       \implies \partial^{\mu}\partial_\mu \chi(x) + m^2\chi(x) &= 0  
\end{align*}
On trouve donc que \eqref{eq:main6} induit l'équation de Klein-Gordon. 

\subsection{}
On veut montrer que l'action
\begin{equation}
        S = \int d^{4}x\, \bigg( \chi^{\dagger} i \bar{\sigma}^{\mu} \partial_\mu \chi + \frac{im}{2}(\chi^T \sigma^{2} \chi - \chi^{\dagger} \sigma^{2}\chi^{*}) \bigg)
\end{equation} 
est réelle $S = S^{*}$. La preuve suit par calcul direct
\begin{align*}
        S^{*} = S^{\dagger} &= 
        \int d^{4}x\, \bigg( (\chi^{\dagger} i \bar{\sigma}^{\mu} \partial_\mu \chi)^{\dagger} - \frac{im}{2}(\chi^T \sigma^{2} \chi - \chi^{\dagger} \sigma^{2}\chi^{*})^{\dagger} \bigg) \\
        &=  \int d^{4}x\, \bigg( -i (\partial_\mu \chi)^{\dagger}(\bar{\sigma}^{\mu})^{\dagger}\chi - \frac{im}{2}(\chi^{\dagger} (\sigma^{2})^{\dagger} \chi^{*} - \chi^{T} (\sigma^{2})^{\dagger}\chi) \bigg) \\
        &=  \int d^{4}x\, \bigg( -i (\partial_\mu \chi)^{\dagger}\bar{\sigma}^{\mu}\chi + \frac{im}{2}(\chi^{T} \sigma^{2}\chi -\chi^{\dagger} \sigma^{2}\chi^{*})\bigg) \\
\end{align*}
où la dernière ligne suit du fait que les matrices de Pauli sont hermitienne $(\sigma^{\mu})^{\dagger} = \sigma^{\mu}$.
Pour progresser, on intègre par partie le premier terme en assumant que le champ $\chi(x)$ s'annule à 
la frontière d'intégration
\begin{align*}
        \int_{\mathcal{V}} d^{4}x\,  (\partial_\mu \chi)^{\dagger}\bar{\sigma}^{\mu}\chi = \cancelto{0}{\bigg[\chi^{\dagger} \bar{\sigma}^{\mu} \chi \bigg]_{\partial\mathcal{V}}} - 
        \int_{\mathcal{V}} d^{4}x\, \chi^{\dagger} \bar{\sigma}^{\mu} \partial_\mu \chi 
\end{align*}
D'où
\begin{align*}
        S^{\dagger} &= \int d^{4}x\, \bigg( i \chi^{\dagger} \bar{\sigma}^{\mu} \partial_\mu \chi + \frac{im}{2}(\chi^{T} \sigma^{2}\chi -\chi^{\dagger} \sigma^{2}\chi^{*})\bigg) \\
        \implies S^{*} &=  S \\
        \implies S &\in \mathbb{R}
\end{align*}
On cherche maintenant à dériver les équations du mouvement pour $\chi$ et $\chi^{*}$ associée à cette action. 
En particulier, on remarque que la dérivée fonctionnelle partielle $\frac{\delta S}{\delta \chi^{\dagger}}$ encode 
la structure de l'équation de Majorana \eqref{eq:main6}. En effet,
\begin{align*}
        \frac{\delta S}{\delta \chi^{\dagger}} = \int d^{4}x\, 
        \big(i \bar{\sigma}^{\mu} \partial_\mu \chi - im\sigma^{2}\chi^{*} \big) = 0\, ,
\end{align*}
On trouve bien l'équation de Majorana de cette façon.


\subsection{}
Soit un spineur $\psi(x)$ dans le représentation chirale
\begin{equation}
        \psi(x) = 
        \begin{bmatrix}
                \psi_L(x) \\
                \psi_R(x)
        \end{bmatrix}
\end{equation} 
On peut exprimer chaque composante en terme des composantes du champ Majorana $\chi(x)$
\begin{equation}\label{eq:dirac_m_rep}
        \psi_L(x) = \chi_1(x), \hspace{1cm} \psi_R(x) = i \sigma^{2}\chi_2^{*}(x)
\end{equation} 
\begin{solution}
        La densité lagrangienne de Dirac s'écrit en terme du champs Majorana comme
        \begin{equation}\;a
                \mathcal{L}_D =  i \chi_1^{\dagger} \bar{\sigma}^{\mu} \partial_\mu \chi_1 + i\chi_2^{\dagger} \bar{\sigma}^{\mu} \partial_\mu \chi_2 + 
                im(\chi_2^{T} \sigma^{2} \chi_1 - \chi_1^{\dagger} \sigma^{2}\chi_2^{*})
        \end{equation} 
\end{solution}
\begin{proof}
La densité lagrangienne de Dirac s'écrit
\begin{equation}
      \mathcal{L}_D = \bar{\psi}(i \slashed{\partial} - m) \psi
\end{equation} 
En terme du champs Majorana \eqref{eq:dirac_m_rep}, on peut écrire
\begin{align*}
       \mathcal{L}_D = 
       \begin{bmatrix}
               \chi_1^{\dagger} & -i \chi_2^{T}\sigma^{2} 
       \end{bmatrix}      
       \gamma^{0}
       (i \gamma^{\mu} \partial_\mu - m)
       \begin{bmatrix}
               \chi_1 \\[1ex] i \sigma^{2}\chi_2^{*}
       \end{bmatrix}
       \\
\end{align*}
Dans la représentation chirale, on a
\begin{equation}
        \gamma^{\mu} = 
        \begin{bmatrix}
                0 & \sigma^{\mu} \\
                \bar{\sigma}^{\mu} & 0
        \end{bmatrix}\, ,
\end{equation} 
de sortes que
\begin{align*}
        \mathcal{L}_D &= 
       \begin{bmatrix}
               \chi_1^{\dagger} & -i \chi_2^{T}\sigma^{2} 
       \end{bmatrix}      
       \begin{bmatrix}
               0 & \bbone \\
               \bbone & 0
       \end{bmatrix}
       \left(  i 
       \begin{bmatrix}
               0 & \sigma^{\mu} \\
               \bar{\sigma}^{\mu} & 0
       \end{bmatrix}
       \partial_\mu - m
        \right)
       \begin{bmatrix}
               \chi_1 \\[1ex] i \sigma^{2}\chi_2^{*}
       \end{bmatrix}
       \\
       &= 
       \begin{bmatrix}
               -i \chi_2^{T}\sigma^{2} & \chi_1^{\dagger}
       \end{bmatrix}      
       \left(  i 
       \begin{bmatrix}
               0 & \sigma^{\mu} \\
               \bar{\sigma}^{\mu} & 0
       \end{bmatrix}
       \partial_\mu - m
        \right)
       \begin{bmatrix}
               \chi_1 \\[1ex] i \sigma^{2}\chi_2^{*}
       \end{bmatrix}
       \\
       &= 
       \begin{bmatrix}
               i\chi_1^{\dagger} \bar{\sigma}^{\mu} &  \chi_2^{T}\sigma^{2} \sigma^{\mu}
       \end{bmatrix}      
       \begin{bmatrix}
              \partial_\mu \chi_1 \\[1ex] i \sigma^{2} \partial_\mu\chi_2^{*}
       \end{bmatrix}
       +
        im(\chi_2^{T} \sigma^{2} \chi_1 - \chi_1^{\dagger} \sigma^{2}\chi_2^{*})
       \\
       &= 
       i \chi_1^{\dagger}\bar{\sigma}^{\mu}\partial_\mu \chi_1 + i\chi_2^{T}\sigma^{2}\sigma^{\mu}\sigma^{2}\partial_\mu \chi_2^{*}
       +
        im(\chi_2^{T} \sigma^{2} \chi_1 - \chi_1^{\dagger} \sigma^{2}\chi_2^{*})
       \\
\end{align*}
On utilisant le fait que $\sigma^{2}\sigma^{\mu}\sigma^{2} = (\bar{\sigma}^{\mu})^{*}$ 
(ce qui peut être déduit d'un argument similaire fait précédemment pour déduire que 
$\sigma^{2}\boldsymbol{ \sigma}^{*}\sigma^{2} = -\boldsymbol{ \sigma}$), alors
\begin{align*}
       \mathcal{L}_D &= 
       i \chi_1^{\dagger}\bar{\sigma}^{\mu}\partial_\mu \chi_1 + i\chi_2^{T}(\bar{\sigma}^{\mu})^{*}\partial_\mu \chi_2^{*}
       +
        im(\chi_2^{T} \sigma^{2} \chi_1 - \chi_1^{\dagger} \sigma^{2}\chi_2^{*})
\end{align*}
Puisque $\chi_2^{T}(\sigma^{\mu})^{*}\partial_\mu \chi_2^{*} \in \mathbb{C}$, alors on peut prendre la transposé de cette quantité sans 
changer le nombre. Ainsi,
\begin{equation}
        \mathcal{L}_D = 
        i \chi_1^{\dagger}\bar{\sigma}^{\mu}\partial_\mu \chi_1 + i\chi_2^{\dagger}(\bar{\sigma}^{\mu})^{\dagger}\partial_\mu \chi_2
       +
        im(\chi_2^{T} \sigma^{2} \chi_1 - \chi_1^{\dagger} \sigma^{2}\chi_2^{*})
\end{equation} 
Puisque les matrices de Pauli sont hermitiennes, alors on obtient finalement
\begin{equation}
        \mathcal{L}_D = 
        i \chi_1^{\dagger}\bar{\sigma}^{\mu}\partial_\mu \chi_1 + i\chi_2^{\dagger}\bar{\sigma}^{\mu}\partial_\mu \chi_2
       +
        im(\chi_2^{T} \sigma^{2} \chi_1 - \chi_1^{\dagger} \sigma^{2}\chi_2^{*})
\end{equation} 

\end{proof}
On note que la forme de la densité lagrangienne obtenue a une forme très similaire à la densité lagrangienne de Majorana. En fait, 
si on avait $\chi_1 = \chi_2$, on retrouve deux fois la densité lagrangienne de Majorana.


\subsection{}
\begin{solution}
La densité lagrangienne de Dirac dans la base de Majorana
\begin{equation}
        \mathcal{L}_D = 
        i \chi_1^{\dagger}\bar{\sigma}^{\mu}\partial_\mu \chi_1 + i\chi_2^{\dagger}\bar{\sigma}^{\mu}\partial_\mu \chi_2
       +
        im(\chi_2^{T} \sigma^{2} \chi_1 - \chi_1^{\dagger} \sigma^{2}\chi_2^{*})
\end{equation} 
possède une symétrie globale $U(1)$. 
\end{solution}
\begin{proof}
La symétrie globale $U(1)$ transforme le champs de Majorana comme
\begin{equation}
       \chi(x) \overset{U(1)}{\rightarrow } e^{i\theta}\chi(x) = 
       \begin{pmatrix}
               e^{i\theta}\chi_1(x) \\[1ex]
               i \sigma^{2}(e^{-i \theta} \chi_2)^{*}
       \end{pmatrix}
\end{equation} 
avec $\theta \in \mathbb{R}$. Ainsi, $\chi_1 \rightarrow  e^{i\theta}\chi_1$ et $\chi_2 \rightarrow e^{-i\theta}\chi_2$, de sortes que 
\begin{align*}
        \mathcal{L}_D \rightarrow \mathcal{L}_D' &= 
        i e^{-i\theta}\chi_1^{\dagger}\bar{\sigma}^{\mu}\partial_\mu e^{i\theta}\chi_1 + ie^{i\theta}\chi_2^{\dagger}\bar{\sigma}^{\mu}\partial_\mu e^{-i\theta}\chi_2
       +
        im(e^{-i\theta}\chi_2^{T} \sigma^{2} e^{i\theta}\chi_1 - e^{-i\theta}\chi_1^{\dagger} e^{i\theta}\sigma^{2}\chi_2^{*})\\
        &=  \mathcal{L}
\end{align*}
\end{proof}
On calcule maintenant la divergence du courant de Noether pour la densité lagrangienne de Dirac
\begin{equation}
        j^{\mu} = \chi^{\dagger} \bar{\sigma}^{\mu}\chi = \chi_1^{\dagger}\bar{\sigma}\chi_1 - \chi_2^{\dagger} \bar{\sigma}^{\mu} \chi_2\, ,
\end{equation} 
Pour se faire, on doit déterminer les équations du mouvement pour les champs $\chi_1$ et $\chi_2$.
Pour le champ $\chi_1$, on a
\begin{align*}
        \frac{\partial \mathcal{L}_D}{\partial \chi_1} &= im \chi_2^{T}\sigma^{2} \\
        \frac{\partial \mathcal{L}_D}{\partial (\partial_\mu \chi_1)} &= i \chi_1^{\dagger} \bar{\sigma}^{\mu}  \\
        \overset{\mathrm{EL}}{\implies} m \chi_2^{T} \sigma^{2} - \partial_\mu \chi_1^{\dagger} \bar{\sigma}^{\mu} &= 0 \\
        \implies m \sigma^{2} \chi_2^{*} - \bar{\sigma}^{\mu} \partial_\mu\chi_1 &= 0 
\end{align*}
Pour le champ $\chi_2$, on a plutôt
\begin{align*}
       \frac{\partial \mathcal{L}_D}{\partial \chi_2} &= im\chi_1^{T}(\sigma^{2})^{T} = -im\chi_1^{T}\sigma^{2} \\
       \frac{\partial \mathcal{L}_D}{\partial (\partial_\mu \chi_2)} &= i \chi_2^{\dagger}\bar{\sigma}^{\mu} \\
       \overset{\mathrm{EL}}{\implies} -m\chi_1^{T}\sigma^{2} - \partial_\mu \chi_2^{\dagger}\bar{\sigma}^{\mu} &= 0 \\
       \implies -m\sigma^{2}\chi_1^{*} - \bar{\sigma}^{\mu} \partial_\mu \chi_2 &= 0
\end{align*}
On utilise ces relations pour déterminer la divergence du courant de Noether
\begin{align*}
        \partial_\mu j^{\mu} &=  \partial_\mu(\chi_1^{\dagger}\bar{\sigma}\chi_1 - \chi_2^{\dagger} \bar{\sigma}^{\mu} \chi_2) \\
        &= (\partial_\mu \chi_1^{\dagger} \bar{\sigma}^{\mu})\chi_1 + \chi_1^{\dagger}\bar{\sigma}^{\mu}\partial_\mu\chi_1
                - (\partial_\mu \chi_2^{\dagger} \bar{\sigma}^{\mu}) \chi_2 
                - \chi_2^{\dagger} \bar{\sigma}^{\mu} \partial_\mu\chi_2 \\
        &=      m\chi_2^{T}\sigma^{2}\chi_1 
                + m\chi_1^{\dagger}\sigma^{2}\chi_2^{*}
                + m\chi_1^{T}\sigma^{2} \chi_2 
                + m\chi_2^{\dagger} \sigma^{2}\chi_1^{*}\\
\end{align*}
En réarrangeant les termes et en exploitant le fait que chaque terme est un nombre complexe pour prendre
leur transposée
\begin{align*}
        \partial_\mu j^{\mu}   
        &=      m\chi_2^{T}\sigma^{2}\chi_1 
                + m\chi_2^{T}(\sigma^{2})^{T} \chi_1
                + m\chi_1^{\dagger}\sigma^{2}\chi_2^{*}
                + m\chi_1^{\dagger} (\sigma^{2})^{T}\chi_2^{*}\\
        &=      m\chi_2^{T}\sigma^{2}\chi_1 
                - m\chi_2^{T}\sigma^{2} \chi_1
                + m\chi_1^{\dagger}\sigma^{2}\chi_2^{*}
                - m\chi_1^{\dagger} \sigma^{2}\chi_2^{*}\\
        &= 0
\end{align*}
Donc le courant de Noether est bien préservé dans cette théorie, ce qui suit de notre observation précédente comme 
quoi $\mathcal{L}_D$ possède une symétrie $U(1)$. On veut maintenant savoir si ce courant est aussi préservé pour 
la densité lagrangienne de Majorana
\begin{equation}
        \mathcal{L}_M = 
        \chi^{\dagger} i \bar{\sigma}^{\mu} \partial_\mu \chi + \frac{im}{2}(\chi^T \sigma^{2} \chi - \chi^{\dagger} \sigma^{2}\chi^{*})
\end{equation} 
Les équations de mouvement sont
\begin{align*}
        \frac{\partial \mathcal{L}_M}{\partial \chi} &= \frac{im}{2}\chi^{T}\sigma^{2} \\
        \frac{\partial \mathcal{L}_D}{\partial (\partial_\mu \chi)} &= i\chi^{\dagger}\bar{\sigma}^{\mu} \\
        \overset{EL}{\implies} \frac{m}{2}\chi^{T}\sigma^{2} - \partial_{\mu}\chi^{\dagger}\bar{\sigma}^{\mu} &= 0\\
        \implies \frac{m}{2}\sigma^{2}\chi^{*} - \bar{\sigma}^{\mu}\partial_{\mu}\chi &=  0
\end{align*}
D'où
\begin{align*}
        \partial_\mu j^{\mu} &= \partial_\mu(\chi^{\dagger} \bar{\sigma}^{\mu}\chi) \\
                &= (\partial_\mu \chi^{\dagger} \bar{\sigma}^{\mu}) \chi + \chi^{\dagger}\bar{\sigma}^{\mu} \partial_\mu \chi \\
                &= \frac{m}{2}(\chi^{T}\sigma^{2}\chi + \chi^{\dagger}\sigma^{2}\chi^{*}) \\
                &\not= 0
\end{align*}
Ainsi, on obtient que la densité lagrangienne de Majorana possède une symétrie $U(1)$ seulement dans le cas où $m = 0$.
Pour la dernière partie de la question, on construit une théorie avec $N$ champs Majorana 
%pour généraliser le cas spécial 
%$\mathcal{L}_D$, qui possède $N = 2$ champs Majorana:
\begin{equation}
        \mathcal{L}_N = \sum_{a=1}^{N}
        i\chi^{\dagger}_a  \bar{\sigma}^{\mu} \partial_\mu \chi_a + \frac{im}{2}(\chi^T_a \sigma^{2} \chi_a - \chi^{\dagger}_a \sigma^{2}\chi^{*}_a)
\end{equation} 
On considère une symétrie $O(N)$, soit la transformation interne
\begin{equation}
        \psi_a(x) \overset{O(N)}{\rightarrow } A_{ab}\psi_b(x)
\end{equation} 
où $A^{T}A = \bbone_{N\times N}$, avec $A_{ij} \in \mathbb{R}$. On montre que cette transformation est une symétrie de $\mathcal{L}_N$
\begin{align*}
       \mathcal{L}_N \overset{O(N)}{\rightarrow } \mathcal{L}_N' &=  
        \sum_{a=1}^{N}  i\chi^{\dagger}_b A_{ab}^{\dagger} \bar{\sigma}^{\mu} \partial_\mu A_{ac}\chi_c 
        + \frac{im}{2}(\chi^T_b A^{T}_{ab}\sigma^{2} A_{ac}\chi_c - \chi^{\dagger}_b A_{ab}^{\dagger}\sigma^{2}A^{*}_{ac}\chi^{*}_c) \\
        &= 
        \sum_{a=1}^{N}  i\chi^{\dagger}_b  \bar{\sigma}^{\mu} \partial_\mu A_{ab}^{T}A_{ac}\chi_c 
        + \frac{im}{2}(\chi^T_b \sigma^{2}A^{T}_{ab} A_{ac}\chi_c - \chi^{\dagger}_b \sigma^{2}A_{ab}^{T}A_{ac}\chi^{*}_c)\hspace{1cm} \{A_{ij} \in \mathbb{R}\} \\
        &= \sum_{a=1}^{N}  i\chi^{\dagger}_b  \bar{\sigma}^{\mu} \partial_\mu \delta_{ab}\delta_{ac}\chi_c 
        + \frac{im}{2}(\chi^T_b \sigma^{2}\delta_{ab}\delta_{ac}\chi_c - \chi^{\dagger}_b \sigma^{2} \delta_{ab}\delta_{ac}\chi^{*}_c) \\
        &= \mathcal{L}
\end{align*}
Donc, $\mathcal{L}_N$ possède une symétrie $O(N)$.

\end{document}

