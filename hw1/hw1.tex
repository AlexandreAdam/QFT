\documentclass{article}
\usepackage[a4paper, margin=2cm]{geometry}

\usepackage[utf8]{inputenc} % allow utf-8 input
\usepackage[T1]{fontenc}    % use 8-bit T1 fonts
\usepackage{hyperref}       % hyperlinks
\hypersetup{
    colorlinks,
    linkcolor={red!50!black},
    citecolor={blue!50!black},
    urlcolor={blue!80!blue}
}
\usepackage{xcolor}
\usepackage{nicefrac}       % compact symbols for 1/2, etc.
\usepackage{bm, bbm}

\usepackage{amsmath}
\usepackage{amssymb}
\usepackage{mathtools}
\usepackage{amstext}
\usepackage{amsthm}
\usepackage{physics}
\usepackage{tensor}

\usepackage{fancyhdr}
\pagestyle{fancy}
\rhead{Alexandre Adam \\ 20090755}
\lhead{Théorie des champs 1 (PHY 6812) \\ William Witczak-Krempa}
\chead{Devoir 1}
\rfoot{14 Octobre 2022}
\cfoot{\thepage}

\numberwithin{equation}{section}
\renewcommand\thesubsection{\alph{subsection})}
\renewcommand\thesubsubsection{\Roman{subsubsection}}
\DeclareRobustCommand{\bbone}{\text{\usefont{U}{bbold}{m}{n}1}}

%\newtheorem*{remark}{Remark}
\theoremstyle{solution}
\newtheorem{solution}{Réponse}[section]

\renewcommand*{\proofname}{Solution}

\begin{document}
\section{Un peu de relativité}

\subsection{}
%\textbf{Classiquement, un photon se propageant dans le vide peut-il se désintégrer en une paire
%électron-positron ? 
%%Expliquez votre résultat mathématique de manière heuristique. 
%Que ce passera-t-il en théorie quantique des champs (QFT) ?}
\subsection{}
\textbf{Quelle est la condition pour qu’un tenseur soit invariant sous transformations de Lorentz ?}
\vspace{2ex}

Soit un tenseur $T$ de rang $(m,n)$, avec les éléments $T\indices{^{\mu_1\dots \mu_m}_{\nu_1\dots\nu_n}}$, et la métrique de Minkowsky $\eta$. Soit une transformation de 
Lorentz $\Lambda \in \mathrm{SO}^{+}(1,3)$ définit tel que
\begin{equation}
        \eta^{\mu \nu} = \Lambda\indices{^{\mu}_\alpha} \Lambda\indices{^{\nu}_\beta} \eta^{\alpha\beta}\, .
\end{equation} 
On requiert que la transformation de Lorentz ait un déterminant unité puisqu'elle doit satisfaire la condition
\begin{equation}
        \Lambda(-v) = \Lambda(v)^{-1}\, ,
\end{equation} 
où $v$ est le paramètre de la transformation (vitesse pour un boost, angle pour une rotation). Puisque 
$\Lambda(-v) = R(\pi)^{T} \Lambda(v) R(\pi) \implies \mathrm{det}(\Lambda(-v)) = \mathrm{det}(\Lambda(v))$, il suit que, 
$\mathrm{det}(\Lambda) = \pm 1$. Finalement, on requiert que $\Lambda(0) = \bbone$, la matrice identité, donc $ \mathrm{det}(\Lambda(0)) = 1\implies \mathrm{det}(\Lambda(v)) = 1$. 

L'inverse transposée de la transformation $(\Lambda^{-1})\indices{^{\mu}_{\nu}} = \Lambda\indices{_\mu^{\nu}}$ agit sur les indices covariants et est définit 
tel que
\begin{equation}
        \eta_{\mu \nu}  = \Lambda\indices{_{\mu}^{\alpha}}\Lambda\indices{_{\nu}^{\beta}}\eta_{\alpha \beta}
\end{equation} 
Pour que $T$ soit un invariant de Lorentz, il doit donc satisfaire
\begin{equation}
        \boxed{ T\indices{^{\mu_1\dots\mu_m}_{\nu_1\dots\nu_n}} = \Lambda\indices{^{\mu_1}_{\alpha_1}} \dots \Lambda\indices{^{\mu_m}_{\alpha_m}} 
        \Lambda\indices{_{\nu_1}^{\beta_1}}\dots \Lambda\indices{_{\nu_n}^{\beta_n}} T\indices{^{\alpha_1\dots \alpha_m}_{\beta_1 \dots \beta_n}}} \, .
\end{equation}

\subsection{}
\textbf{Les tenseurs $\delta_{\mu}^{\nu}$ et  $\epsilon^{\mu\nu\rho\sigma}$ (Levi-Civita) sont-ils invariants de Lorentz ?}
\vspace{2ex}
\begin{solution}
        Le delta de Kronecker $\delta_\mu^{\nu}$ est un invariant de Lorentz.
\end{solution}
\begin{proof}
On remarque que $\delta^{\nu}_{\mu} = \eta_{\mu \alpha} \eta^{\alpha\nu}$. Puisque la métrique de Minkowsky est une invariant de 
Lorentz (par définition), alors $\delta^{\nu}_{\mu}$ l'est aussi.
\end{proof}
%\begin{proof}[Solution alternative]
%Soit $\Lambda$ une transformation de Lorentz. Alors,
%\begin{align}
        %(\delta')^{\nu}_\mu &= \Lambda\indices{^{\nu}_{\alpha}} \Lambda\indices{^{\beta}_{\mu}} \delta^{\beta}_{\alpha} \\
%\end{align}
        
%\end{proof}

\begin{solution}
Le tenseur de Levi-Civita est un invariant de Lorentz.
\end{solution}
\begin{proof}
On applique une transformation de Lorentz sur le tenseur de Levi-Civita
\begin{align*}
        (\epsilon')^{\mu\nu\rho\sigma} = \Lambda\indices{^{\mu}_{\alpha}} \Lambda\indices{^{\nu}_{\beta}} \Lambda\indices{^{\rho}_{\gamma}} \Lambda\indices{^{\sigma}_{\delta}} \epsilon^{\alpha\beta\gamma\delta }\, .
\end{align*} 
On examine maintenant certains éléments du tenseur transformé. Je note $S_n^{+} = \{\sigma \in S_n \mid \mathrm{sgn}(\sigma) = +1\}$ les permutations paires 
pour $n$ indices, où $\mathrm{sgn}(\sigma)$ dénote le signe du déterminant de la matrice associé à la permutation $\sigma$.
Par exemple, $S_3^{+} = \{(1\, 2\, 3),\, (3\, 1\, 2), (2\, 3\, 1)\}$; la matrice associé 
à $\sigma = (3\, 1\, 2)$ est
\begin{equation*}
M_{\sigma} = \begin{pmatrix}
        0 & 0 & 1 \\
        1 & 0 & 0 \\
        0 & 1 & 0
\end{pmatrix}\, .
\end{equation*} 
Elle possède un déterminant positif, $\mathrm{det}(M_\sigma) = 1$, de sortes que $\mathrm{sgn}(\sigma) = +1$. 
L'ensemble des permutations impaires est noté $S_n^{-}$. Naturellement, $|S_n^{+}| = |S_n^{-}| = \frac{n!}{2}$. 
En utilisant la propriété complètement antisymétrique du tenseur de Levi-Civita
\begin{equation}
        \epsilon^{\mu\nu\rho\sigma} = \begin{cases}
                +1 & \mu\nu\rho\sigma \in S_4^{+} \\
                -1 & \mu\nu\rho\sigma \in S_4^{-} \\
                0 & \mathrm{autrement}
        \end{cases}\, ,
\end{equation} 
on trouve que
\begin{equation}
        (\epsilon')^{\mu\nu\rho\sigma} = 
        \sum_{\alpha\beta\gamma\delta \in S_4^{+}} 
 \Lambda\indices{^{\mu}_{\alpha}} \Lambda\indices{^{\nu}_{\beta}} \Lambda\indices{^{\rho}_{\gamma}} \Lambda\indices{^{\sigma}_{\delta}} 
-
        \sum_{\alpha\beta\gamma\delta \in S_4^{-}} 
 \Lambda\indices{^{\mu}_{\alpha}} \Lambda\indices{^{\nu}_{\beta}} \Lambda\indices{^{\rho}_{\gamma}} \Lambda\indices{^{\sigma}_{\delta}} \, .
\end{equation} 

\paragraph{Cas des indices identiques:} Sous cette forme, on peut résoudre les valeurs du tenseur transformé lorsqu'au moins 2 indices sont identiques. Dans ce cas, 
$(\epsilon')^{\mu\nu\rho\sigma} = \epsilon^{\mu\nu\rho\sigma} = 0$. Par exemple, 
considérons le cas $\mu\nu\rho\sigma = 0012$, où les deux premiers indices sont identiques.
\begin{align*}
        (\epsilon')^{0012} &= 
        \sum_{\alpha\beta\gamma\delta \in S_4^{+}} 
 \Lambda\indices{^{0}_{\alpha}} \Lambda\indices{^{0}_{\beta}} \Lambda\indices{^{1}_{\gamma}} \Lambda\indices{^{2}_{\delta}} 
-
        \sum_{\alpha\beta\gamma\delta \in S_4^{-}} 
 \Lambda\indices{^{0}_{\alpha}} \Lambda\indices{^{0}_{\beta}} \Lambda\indices{^{1}_{\gamma}} \Lambda\indices{^{2 }_{\delta}}  \\
                           &= \dots + 
 \Lambda\indices{^{0}_{0}} \Lambda\indices{^{0}_{1}} \Lambda\indices{^{1}_{2}} \Lambda\indices{^{2}_{3}} 
-
\Lambda\indices{^{0}_{1}} \Lambda\indices{^{0}_{0}} \Lambda\indices{^{1}_{2}} \Lambda\indices{^{2}_{3}} 
-\dots \\
&= 0\, .
\end{align*}
En effet, pour chaque permutations $\alpha\beta\gamma\delta \in S_4^{+}$, on peut échanger $\alpha$ et $\beta$, 
de sortes qu'on peut toujours trouver le terme $\beta\alpha\gamma\delta \in S_4^{-}$ dans la somme 
sur les indices impaires qui annule le terme $\alpha\beta\gamma\delta$ dans la première somme.
Cet argument se généralise à tout autre cas où au moins deux indices parmis $\mu\nu\rho\sigma$ sont identiques.

\paragraph{Cas des indices distincts:} 
Je poursuis la démonstration avec l'exemple $\mu\nu\rho\sigma=0123$. Pour le résoudre, on doit démontrer que 
$(\epsilon')^{0123} = \mathrm{det}(\Lambda) = 1$.  La formule de Leibniz pour le déterminant nous indique que
\begin{equation}\label{eq:Leibniz}
        \mathrm{det}(\Lambda) = \Lambda\indices{^{0}_{\alpha}}\Lambda\indices{^{1}_{\beta}}\Lambda\indices{^{2}_{\gamma}}\Lambda\indices{^{3}_{\delta}}\epsilon^{\alpha\beta\gamma\delta}\, .
\end{equation} 
Le côté droit de l'égalité est précisément le résultat de la transformation de Lorentz pour l'élement $0123$ du tenseur de Levi-Civita, 
donc $(\epsilon')^{0123} = \mathrm{det}(\Lambda) = 1$. 
Les autres cas suivent par la permutation des indices contravariant au côté droit de la formule de Leibniz. 
Par exemple, supposons que $\mu\nu\rho\sigma = 1032 \in S_4^{+}$:
\begin{align*}
        (\epsilon')^{1032} &= \Lambda\indices{^{1}_{\alpha}}\Lambda\indices{^{0}_{\beta}}\Lambda\indices{^{3}_{\gamma}}\Lambda\indices{^{2}_{\delta}}\epsilon^{\alpha\beta\gamma\delta}\\
                           &= \Lambda\indices{^{0}_{\beta}}\Lambda\indices{^{1}_{\alpha}}\Lambda\indices{^{2}_{\delta}}\Lambda\indices{^{3}_{\gamma}}\epsilon^{\alpha\beta\gamma\delta}  \hspace{1cm} \{\text{Réarrangement des termes}\} \\
               &= \Lambda\indices{^{0}_{\beta}}\Lambda\indices{^{1}_{\alpha}}\Lambda\indices{^{2}_{\delta}}\Lambda\indices{^{3}_{\gamma}}\epsilon^{\beta\alpha\delta\gamma}  \hspace{1cm} \{\text{Permutation paire des indices du tenseur de Levi-Civita} \} \\
               &= \Lambda\indices{^{0}_{\alpha}}\Lambda\indices{^{1}_{\beta}}\Lambda\indices{^{2}_{\gamma}}\Lambda\indices{^{3}_{\delta}}\epsilon^{\alpha\beta\gamma\delta}  \hspace{1cm} \{\text{Redéfinition des indices factices} \} \\
               &= \mathrm{det}(\Lambda) = 1\, .\\ 
\end{align*} 
Comme l'argument est général, on a que $(\epsilon')^{\mu\nu\rho\sigma} = \epsilon^{\mu\nu\rho\sigma},\,\, \forall \mu\nu\rho\sigma \in S_4^{+}$.
L'argument pour une permutation impaire est très similaire et nous mène à conclure que 
\begin{equation}
(\epsilon')^{\mu\nu\rho\sigma} = -\mathrm{det}(\Lambda) = \epsilon^{\mu\nu\rho\sigma}, \,\, \forall \mu\nu\rho\sigma \in S_4^{-}\, .
\end{equation} 
Donc, ayant couvert tous les cas possible, on conclut que
\begin{equation}
        (\epsilon')^{\mu\nu\rho\sigma} = \epsilon^{\mu\nu\rho\sigma}\, .
\end{equation} 
\end{proof}
        

\section{Invariance d'échelle}
Soit un champ scalaire Klein-Gordon $\phi$ de masse $m$ en $d$ dimensions spatiales. Considérons une transformation continue
\begin{align}
        \label{eq:transformation}
        x &\rightarrow bx \\
        \phi(x) &\rightarrow b^{-\Delta}\phi(x)  
\end{align}
où $b \in \mathbb{R}_{>0}$ et $\Delta \in \mathbb{R}$.

\subsection{}
\textbf{Quelle sont les conditions pour que \eqref{eq:transformation} soit une symétrie de la théorie? Appelons l’action
de cette théorie $S_{\star}$. Quel est le courant de Noether associé?}
\vspace{2ex}

La transformation \eqref{eq:transformation} est une symétrie de la théorie si l'action 
\begin{equation}\label{eq:action}
        S_\star = \int d^{d+1}x\, \mathcal{L}_\star(\phi(x), \partial_{\mu}\phi(x))
\end{equation} 
est invariante sous l'application de la transformation. Puisque l'élément de volume devient
\begin{equation}
        d^{d+1}x \rightarrow b^{d+1}(d^{d+1}x)\, ,
\end{equation} 
alors la transformation est une symétrie de la théorie si et seulement si
\begin{equation}
       \mathcal{L}_{\star} \rightarrow b^{-(d+1)}\mathcal{L_{\star}} + \partial_{\mu}\mathcal{J}^{\mu}\, .
\end{equation} 
La divergence $\partial_{\mu}\mathcal{J}^{\mu}$ devient un terme de surface dans l'intégrale de l'action, qui 
s'annule par définition de la dérivée fonctionnelle de l'action.
%\paragraph{Cas $m=0$:} 
%On considère en premier lieu le Lagrangien d'un champ sans masse. 
%Sachant que la dérivée se transforme comme
Comme la dérivée se transforme de la façon suivante
\begin{equation}
        \partial^{\mu} \rightarrow \frac{\partial}{\partial (b x)} = b^{-1}\partial^{\mu}\, ,
\end{equation}
alors on peut montrer que le Lagrangien de Klein-Gordon se transforme comme
\begin{equation}\label{eq:lagrangian transform}
        \mathcal{L}_\star \rightarrow  \mathcal{L}_\star' = b^{-2-2\Delta}\frac{1}{2}(\partial_{\mu}\phi)(\partial^\mu \phi) -m^2b^{-2\Delta}\phi^2 %= b^{-2-2\Delta}\mathcal{L}_\star\, .
\end{equation} 
Pour factoriser le facteur d'échelle, on doit absolument avoir $\boxed{m=0}$, de sortes que
\begin{equation}\label{eq:lagrangian transform}
        \mathcal{L}_\star \rightarrow  \mathcal{L}_\star' = b^{-2-2\Delta}\frac{1}{2}(\partial_{\mu}\phi)(\partial^\mu \phi) = b^{-2-2\Delta}\mathcal{L}_\star\, .
\end{equation} 
Ainsi, la condition pour que la théorie possède une symétrie d'échelle est
\begin{equation}\label{eq:Delta}
        \boxed{\Delta = \frac{d - 1}{2}}\, .
\end{equation} 
Pour déterminer le courant de Noether, on doit considérer la transformation infinitésimale. On pose $b = 1 - \epsilon$, où $\epsilon \rightarrow 0$. 
On écrit la variation infinitésimale du champ, $\delta \phi$, comme une transformation active de $\phi$:
\begin{equation}
       \delta \phi = \phi'(x) - \phi(x) \, ,
\end{equation} 
où
\begin{align*}
        \phi'(x) &= b^{-\Delta}\phi(b^{-1}x) \\
              &= (1 + \epsilon\Delta) \phi(x + \epsilon x) + \mathcal{O}(\epsilon^{2}) \\
              &= \phi(x) + \epsilon(\Delta + x^{\mu}\partial_{\mu})\phi(x) + \mathcal{O}(\epsilon^{2})\, .
\end{align*}
où on a utiliser la dérivée directionnelle pour exprimer la variation du champs dû à un changement de coordonnées 
$\phi(x + a^{\mu}) = \phi(x) + a^{\mu}\partial_{\mu}\phi(x) + \mathcal{O}(a^{2})$.
Donc,
\begin{equation}
        \delta \phi = \epsilon(\Delta + x^\mu \partial_\mu)\phi\, .
\end{equation} 
Puisque l'opérateur $\partial_\mu$ commute avec la variation infinitésimale $\delta$, on peut écrire
\begin{equation}
         \delta (\partial_\mu \phi) = \partial_{\mu}\delta\phi =  \epsilon(\Delta + 1 + x^{\nu}\partial_\nu)\partial_\mu \phi 
\end{equation} 
La variation infinitésimal du Lagrangien devient alors
\begin{align*}
        \delta  \mathcal{L}_\star &= \frac{\partial \mathcal{L}_\star}{\partial (\partial_\mu \phi)} \delta (\partial_\mu \phi) \\
                &= (\partial^\mu \phi)(\partial_\mu \delta \phi) \\
                &= \epsilon (\Delta + 1 + x^\mu \partial_\mu)(\partial^\mu \phi) (\partial_\mu \phi) \\
                &= \epsilon  x^\mu \partial_\mu \mathcal{L}_\star + 2\epsilon (\Delta + 1) \mathcal{L}_\star \\
                %&= \epsilon \partial_\mu (x^\mu \mathcal{L}_\star) +  2 \epsilon (\Delta + 1) \mathcal{L}_\star\, ,
\end{align*}
Le second terme laisse l'action invariante lorsque \eqref{eq:Delta} est respectée puisque ce terme annule le déterminant de la Jacobienne dans l'intégrale d'action. 
%La manipulation à la dernière ligne mérite quelques explications. 
%En effet, on 
On peut donc construire un courant de Noether
\begin{equation}
        j^{\mu} = \frac{\partial \mathcal{L}}{\partial (\partial_{\mu}\phi)} \delta  \phi  - \mathcal{J}^{\mu}\, ,
\end{equation} 
où on trouve, à l'aide de notre expression pour $\delta \mathcal{L}_\star$,
\begin{equation}
        \mathcal{J}^{\mu} = x^{\mu}\mathcal{L}_\star\, .
\end{equation} 
Ainsi
\begin{equation}\label{eq:Noether current}
        \boxed{j^{\mu}_\star = (\partial^{\mu} \phi)(\Delta + x^{\mu}\partial_{\mu})\phi - x^{\mu}\mathcal{L}_\star}
\end{equation} 


\subsection{}
\textbf{Soit une quantité $\mathcal{O}(x)$ qui dépend du champ et ses dérivées au point $x$. Posons que $\mathcal{O}$ 
transforme comme $\phi$ sous \eqref{eq:transformation}, mais avec $\Delta$ remplacé par $\Delta_\mathcal{O}$, appelé la dimension d’échelle
de $\mathcal{O}$. Pour les conditions trouvées en a), quelle est la dimension d’échelle de la densité
Lagrangienne $\mathcal{L}_{\star}$ et de $\phi^{n}$, où $n \in \mathbb{N}$}.
\vspace{2ex}

Avec les conditions trouvées en \textbf{a)}, on a $\boxed{\Delta_{\mathcal{L}_\star} = d+1}$ et $\boxed{\Delta_{\phi^{n}} = \frac{n(d-1)}{2}}$

\subsection{}
\textbf{On considère le Lagrangien avec un terme d'interaction}
\begin{equation}\label{eq:Lagrangien2c}
       \mathcal{L}_{\mathrm{int}} = \mathcal{L}_{\star} - \lambda \phi^{2n} 
\end{equation} 
\textbf{où $n \in \mathbb{N}$}.
\textbf{Quel doit être le signe de $\lambda$ pour que la théorie soit physiquement 
raisonnable? Quelle est l’équation du mouvement pour cette théorie interagissante. Quelle est la nouvelle difficulté?}

\vspace{2ex}

Le signe de $\lambda $ gouverne la forme du potentiel $V(\phi) = \frac{1}{2}m^{2}\phi^{2} + \lambda \phi^{2n}$. 
Dans le cas où $\lambda > 0$, alors le potentiel possède un état fondamental avec une énergie minimale, où $\phi = 0$. 
Dans le cas où $\lambda < 0$, alors on peut, en principe, avoir des états avec une énergie arbitrairement négative.
Donc il n'y a pas d'état fondamental. Ainsi, on doit avoir $\lambda \in \mathbb{R}_{>0}$.
L'équation du mouvement du champ $\phi(x)$ gouverné par $\mathcal{L}_{\mathrm{int}}$ est
\begin{equation}
        \boxed{\Box\phi + \frac{\partial V}{\partial \phi} = \partial^\mu \partial_{\mu} \phi + m^2\phi + 2n\lambda \phi^{2n-1} = 0}
\end{equation} 
La difficulté est de résoudre cette équation différentielle pour $\phi$, qui n'est plus une combination linéaire d'ondes planes. 

\subsection{}
\textbf{En $d \in \{1,2,3\}$ dimensions, quelles sont les conditions pour que la théorie interagissante soit invariante
sous une transformation d’échelle \eqref{eq:transformation}.}
\vspace{2ex}

Le terme d'interaction rajouté, $V_{\mathrm{\phi}}(\phi) = \lambda \phi^{2n}$, se transforme comme
\begin{equation}
        V_{\mathrm{int}} \rightarrow b^{-n(d-1)}V_{\mathrm{int}}\, .
\end{equation} 
puisque $\Delta_{\phi^{n}} = \frac{n(d-1)}{2}$.
Pour que le Lagrangien se transforme comme $\mathcal{L}_{\mathrm{int}} \rightarrow b^{-(d+1)}\mathcal{L}_{\mathrm{int}}$, on doit avoir $m=0$ et 
\begin{equation}
        \boxed{n = \frac{d+1}{d-1}}\, .
\end{equation} 
Ainsi, il n'y a pas de symétrie d'échelle pour les théories interagissantes avec $d=1$ dimension. 
Pour $d=2$, la théorie interagissante avec $n=3$ possède une symétrie d'échelle. Pour $d=3$, la théorie interagissante 
avec $n=2$ possède une symétrie d'échelle.

\section{Champs de jauge}
Soit l'action de Maxwell
\begin{equation}\label{eq:MaxwellAction}
       S = \int d^{d+1}x\, F_{\mu\nu}F^{\mu\nu}\, ,
\end{equation} 
où $d \geq 1$.
\subsection{}
\textbf{Démontrer que l’action est invariante sous transformations de jauge : $A_{\mu}(x) \rightarrow A_{\mu}(x) +
\partial_{\mu}f(x)$, où $f$ est une fonction scalaire suffisamment lisse.}
\vspace{2ex}

%Par définition du tenseur électromagnétique
%\begin{equation}\label{eq:F}
        %F_{\mu\nu} = \partial_{\mu}A_\nu - \partial_{\nu}A_\mu\, ,
%\end{equation} 
%la transformation de jauge induit 
%\begin{equation}
        %F_{\mu\nu} \rightarrow F_{\mu \nu} + \partial_{\mu}\partial_{\nu}f - \partial_{\nu}\partial_{\mu}f\, .
%\end{equation} 
\begin{proof}
Soit le Lagrangien de Maxwell
\begin{equation}
        \mathcal{L} = -\frac{1}{4}F^{\mu \nu}F_{\mu\nu}\, , %= -\frac{1}{2}(\partial^{\mu}A^{\nu})(\partial_{\nu}A_\mu) + \frac{1}{2}(\partial_{\mu}A^{\mu})^{2}
\end{equation} 
où $F_{\mu\nu}$ est le tenseur électromagnétique % It is antisymmetric, with a nul trace
\begin{equation}\label{eq:F}
        F_{\mu\nu} = \partial_{\mu}A_\nu - \partial_{\nu}A_\mu\, .
\end{equation} 
Pour déterminer si la transformation de l'action est invariante sous la transformation de jauge, on doit s'assurer que $\delta S = 0$.
On commence par étudier la variation infinitésimal du Lagrangien
\begin{equation}
        \delta \mathcal{L} = -\frac{1}{2}F^{\mu \nu} \delta F_{\mu \nu}
\end{equation} 
La variation infinitésimale de $F_{\mu \nu}$ sous la transformation de jauge est
\begin{align*}
        \delta F_{\mu \nu} &= \partial_{\mu}\delta A_{\nu} - \partial_{\nu} \delta A_\mu \\
        &= \partial_{\mu}\partial_\nu f - \partial_{\nu} \partial_\mu f \\
        &= 0\, .
\end{align*}
La dernière ligne est une conséquence du théorème de Schwarz, qui dicte que les dérivée secondes d'une fonction scalaire sont symétriques dans une région $\Omega \subset \mathbb{R}^{d+1}$ qui 
contient un point $\mathcal{O} \in \Omega$ où la fonction $f:\Omega \rightarrow \mathbb{R}$ possède des dérivées secondes continues (en d'autre mots, la fonction est lisse). 
Il suit que l'action est invariante sous la transformation de jauge, 
$\delta S = \int d^{d+1} \delta L = 0$.
\end{proof}

\subsection{}
\textbf{Démontrer que l’action est invariante sous transformations de Lorentz. Est-ce qu’il existe
un terme de masse pour le champ $A_{\mu}$ qui serait invariant de jauge et invariant de Lorentz?}
\vspace{2ex}
\begin{proof}
Soit la transformation de Lorentz
\begin{equation}\label{eq:transf1}
        x^{\mu} \rightarrow \Lambda\indices{^\mu_\nu} x^\nu \, .
\end{equation} 
On étudie maintenant comment le Lagrangien se transforme. Pour se faire, on doit déterminer comment une 
dérivée se transforme. On trouve
\begin{equation}
        \partial_\mu \rightarrow \partial_\mu' = \frac{\partial }{\partial (\Lambda\indices{^\mu_\nu} x^\nu)} = (\Lambda^{-1})\indices{^{\nu}_\mu}\partial_{\nu}= \Lambda\indices{_\mu^\nu}\partial_{\nu}\, ,
\end{equation} 
où on a introduit l'inverse transposée de la transformation de Lorentz $(\Lambda^{-1})\indices{^{\mu}_{\nu}} = \Lambda\indices{_\mu^{\nu}}$. Avec ce résultat, 
on déduit la règle de transformation pour le vecteur covariant $A_{\mu}$
\begin{equation}
        A_{\mu} \rightarrow  \Lambda\indices{_{\mu}^{\nu}}A_{\nu}\, .
\end{equation} 
On calcule ensuite la transformation du tenseur électromagnétique
\begin{align}
        F_{\mu\nu}\rightarrow F^{'}_{\mu\nu} &=\Lambda\indices{_{\mu}^{\rho}}\Lambda\indices{_{\nu}^{\sigma}}\partial_{[\rho} A_{\sigma]}\, .
\end{align}
Finalement, on calcule la transformation du Lagrangien. Notons que les vecteurs contravariants $A^{\mu}$ et $\partial^{\mu}$ se transforme 
exactement comme \eqref{eq:transf1}, de sortes que
\begin{align*}
        \mathcal{L} \rightarrow \mathcal{L}' &= -\frac{1}{4}(F')^{\mu\nu}(F')_{\mu\nu} \\
        &= -\frac{1}{4}(\Lambda\indices{^{\mu}_{\alpha}}\Lambda\indices{^{\nu}_{\beta}}\partial^{[\alpha}A^{\beta]})(\Lambda\indices{_{\mu}^{\rho}}\Lambda\indices{_{\nu}^{\sigma}}\partial_{[\rho} A_{\sigma]}) \\
        &= -\frac{1}{4}\delta_\alpha^{\rho}\delta_\beta^{\sigma}(\partial^{[\alpha}A^{\beta]})(\partial_{[\rho} A_{\sigma]}) \\
        &= -\frac{1}{4}(\partial^{[\alpha}A^{\beta]})(\partial_{[\alpha} A_{\beta]}) \\
        &= -\frac{1}{4}F^{\alpha\beta}F_{\alpha\beta} \\
        &= \mathcal{L}\, .
\end{align*}
Donc l'action est invariante sous une transformation de Lorentz
\begin{equation}
        S \rightarrow S' = \int d^{d+1}x\, \mathcal{L}' = \int d^{d+1}x\, \mathcal{L} = S\, .
\end{equation} 

%Oui si le terme de masse est une transformation de jauge... 
%\begin{solution}
      %Le champ $B_\mu = A_\mu + ...$ est aussi un invariant de jauge de Lorentz et possède un terme de masse. 
%\end{solution}
%\begin{proof}
        
%\end{proof}

\end{proof}
\subsection{}
\textbf{Déterminer les équations du mouvement de la théorie. Est-ce qu’elles correspondent aux
équations classiques de Maxwell? Si oui, sous quelles conditions?}
\vspace{2ex}

%Comme le Lagrangien ne dépend que du tenseur électromagnétique, on a que les équations du mouvements sont proportionnelles à
%\begin{equation}
        %\partial_{\mu}\left( \frac{\partial L}{\partial F_{\mu\nu}} \right) \propto \boxed{-\partial_{\mu}F^{\mu\nu} = 0}
%\end{equation} 
%En principe, on devrais prendre la dérivée en terme de $\partial_{\mu}A_\nu$ pour obtenir les bons préfacteur (puisqu'on a normaliser le Lagrangien par $\frac{1}{4}$). 
%Toutefois, ce raccourcis nous donne la bonne équation du mouvement.
On commence par calculer
\begin{align*}
        \frac{\partial \mathcal{L}}{\partial (\partial_{\mu}A_\nu)} &= -\frac{1}{4}\frac{\partial }{\partial (\partial_{\mu}A_\nu)} (\partial^{[\rho}A^{\sigma]})(\partial_{[\rho}A_{\sigma]}) \\
                &= -\frac{1}{2}\frac{\partial }{\partial (\partial_{\mu}A_\nu)} \big((\partial^{\rho}A^{\sigma})(\partial_{\rho}A_\sigma) - (\partial^{\rho}A^{\sigma})(\partial_{\sigma}A_\rho) \big)\\
                &= -(\partial^{\rho}A^{\sigma})\delta_\rho^{\mu}\delta_\sigma^{\nu} + (\partial^{\rho}A^{\sigma})\delta_\sigma^{\mu}\delta_\rho^{\nu}\\
                &= -(\partial^{\mu}A^{\nu}) + (\partial^{\nu}A^{\mu})\\
                &= -F^{\mu\nu}\, .
\end{align*}
D'où les équations du mouvement
\begin{equation}\label{eq:eq mouvement}
        \boxed{\partial_{\mu}F^{\mu\nu} = 0 }\, .
\end{equation} 

On peut maintenant vérifier que cette équation nous permet de retrouver les équations classiques de Maxwell. On pose
\begin{equation}
        A^{\mu} = (\phi, \mathbf{A})\, ,
\end{equation} 
et
\begin{equation}
        \partial^{\mu} = (\partial_{t}, -\grad)\, .
\end{equation} 
On utilise les définitions $\mathbf{E} = -\grad \phi - \partial_t \mathbf{A}$ et $\mathbf{B} = \grad \times \mathbf{A}$ pour les 
champs électriques et magnétiques respectivement.
On commence par l'équation correspondant avec $\nu=0$.
On trouve que
\begin{align*}
        \partial_{\mu}F^{\mu 0} &= \partial_{\mu} \partial^{\mu}\phi - \partial_{\mu}\partial_{t}A^{\mu} \\
                &= (\partial_{t}^{2} - \grad^{2})\phi - (\partial_{t}^{2}\phi - \grad \cdot \partial_{t}\mathbf{A}) \\
                &= \grad \cdot (-\grad \phi - \partial_t \mathbf{A}) \\
                &= \grad \cdot \mathbf{E}\, .
\end{align*}
D'où $\grad \cdot \mathbf{E} = 0$, la première équation de Maxwell dans le vide.
On poursuit avec $\nu = i \in \{1,2,3\}$:
\begin{align*}
        \partial_{\mu}F^{\mu i} &= \partial_{\mu}\partial^{\mu}A^{i} - \partial_{\mu}\partial^{i}A^{\mu} \\
                &= (\partial_{t}^{2} - \grad^{2})A^{i} - \partial_{t}\partial^{i}\phi - \partial_{j}\partial^{i}A^{j} \\
                &= (\partial_{t}^{2} - \grad^{2})A^{i} - \partial_{t}\partial^{i}\phi - \partial^{i}\partial_j A^{j}\, ,
\end{align*}
où la dernière ligne suit du théorème de Schwarz. 
On additionne ensuite les équations pour chaque indice $i$, de sortes que 
\begin{align*}
        0 &= \partial_{t}^{2}\mathbf{A} - \grad^{2}\mathbf{A} + \partial_{t}\grad \phi + \grad (\grad \cdot \mathbf{A}) \\
        \implies 0 &= -\partial_{t}\mathbf{E} + \grad \times \mathbf{B}\, .
\end{align*}
On a utiliser l'identité vectorielle $\grad^{2}\mathbf{A} = \grad (\grad \cdot \mathbf{A}) - \grad \times (\grad \times \mathbf{A})$ à la dernière ligne. 
On trouve la loi d'Ampère dans le vide. 
%La seconde loi de Maxwell suit de l'identité vectorielle $\grad \cdot (\grad \times \mathbf{A}) = 0$, alors que la loi de ... suit de l'identité de Bianchi

%%TODO: a quelles condition


\subsection{}
\textbf{Est-ce que la théorie de Maxwell est invariante sous une transformation d’échelle? Si
oui, quelle sont les conditions appropriées, ainsi que les dimensions d’échelle du champ de jauge
et du champ électrique.}
\vspace{2ex}

Soit la transformation d'échelle
\begin{equation}
\begin{split}
        x &\rightarrow  bx \\
        A^{\mu} &\rightarrow b^{-1}A^{\mu}\,. 
\end{split}
\end{equation} 
Le Lagrangien de Maxwell se transforme comme
\begin{equation}
        \mathcal{L} \rightarrow b^{-4}\mathcal{L}
\end{equation} 
Ce qui laisse invariant l'action
\begin{equation}
        S \rightarrow S' =  b^{d+1 - 4}\int d^{d+1}x \,\mathcal{L} = b^{d-3}S\, ,
\end{equation} 
\textbf{seulement si $\boxed{d=3}$.} La dimension d'échelle du champ de jauge $A^{\mu}$ est $\boxed{\Delta_{A^{\mu}} = 1}$, de sortes que la 
dimension d'échelle du champ électrique se doit d'être $\boxed{\Delta_{\mathbf{E}} = 2}$ puisque
\begin{equation}
        \Delta_{\mathbf{E}} = \Delta_{\partial^{\mu}} + \Delta_{A^{\mu}}
\end{equation} 



\subsection{}
\textbf{Quel est le champ canoniquement conjugué à $A_\mu$ . 
\vspace{2ex}
%Est-ce qu’il y a quelque chose de bizarre
%avec votre réponse ?
}

\section{Phonons}
\subsection{}
%\textbf{En vous basant sur l’analyse faite en classe en 1+1 dimensions, obtenez la théorie classique des champs décrivant les vibrations d’un cristal cubique en 2 et 3 dimensions spatiales.
%Utilisez l’approche "vache sphérique" : posez une forme simple pour l’énergie potentielle
%d’élongation-compression qui généralise le cas en d = 1. Le vrai cas des phonons est plus
%difficile à traiter}

%%On considère une chaîne de $N$ atomes de masse $m$. Chaque atome est connecté par un ressort de raideur $\kappa$. 
%%On définit la fréquence d'oscillation
%%\begin{equation}
        %%\omega \equiv \sqrt{\frac{\kappa}{m}}\, .
%%\end{equation} 
%On utilise les coordonnées canoniques $q_i$ et $p_i$ pour décrire le déplacement horizontal de chaque atome par rapport à leur position d'équilibre. 
%Pour déterminer le Lagrangien, on fait une expansion de Taylor du potentiel de Hooke, $V$, convexe autour de la position d'équilibre du système. Dans cette expansion, 
%on ignore les termes d'interactions, ce qui revient à faire l'approximation que la matrice hessienne du potentiel est diagonale
%\begin{equation}\label{eq:Hooke Pot}
        %V = \frac{\kappa}{2}\sum_{i=1}^Nq_i^2 +\mathcal{O}((q_i - q_{i+1})^{2}) %\frac{\kappa}{2}\sum_{i=1}^N (q_i - q_{i+1})^2 
%\end{equation} 
%où le prochain terme dans l'expansion nous donnerais le potentiel lié aux interactions voisines. 
%On obtient donc le Lagrangien du système
%\begin{equation}
        %L = \sum_{i=1}^{N}\bigg(\frac{1}{2}m\dot{q}_i^2 - \frac{1}{2}m\omega^{2}q_i^2\bigg) %- \frac{1}{2}m\omega^2(q_i - q_{i+1})^2\bigg)
%\end{equation} 
%Puisqu'on est libre de redéfinir les coordonnées canoniques, on absorbe le paramètres de masse dans les $\sqrt{m}q_i \rightarrow q_i$.
%L'Hamiltonien est la transformée de Legendre du Lagrangien, avec les moments conjugués $p_i = \frac{\partial L}{\partial \dot{q}_i} = \dot{q}_i$, d'où
%\begin{equation}
        %H = \sum_{i=1}^N \bigg(\frac{p_i^2}{2} + \frac{1}{2}\omega^{2}q_i^2\bigg) %+ \frac{1}{2}m\omega^2(q_i - q_{i+1})^2\bigg)
%\end{equation} 


\subsection{}
%\textbf{Quelles sont les symétries continues de cette théorie en 2+1 et 3+1 dimensions ? Travaillez avec les symétries internes seulement, 
%c’est-à-dire les symétries ne faisant pas intervenir d’opération sur l’espace-temps. Déterminez les courants de Noether associés.}

\section{Quantification 101}
%Considérons la chaîne harmonique classique en 1 dimension spatiale, telle que vue en classe.
On considère une chaîne de $N$ atomes de masse $m$. Chaque atome est connecté par un ressort de raideur $\kappa$. 
On définit la fréquence d'oscillation
\begin{equation}
        \omega \equiv \sqrt{\frac{\kappa}{m}}\, .
\end{equation} 
On utilise les coordonnées canoniques $q_i$ et $p_i$ pour décrire le déplacement horizontal (1d) de chaque atome par rapport à leur position d'équilibre. 
Pour déterminer le Lagrangien, on fait une expansion de Taylor du potentiel de Hooke, $V$, convexe autour de la position d'équilibre du système. Dans cette expansion, 
on ignore les termes d'interactions \textit{lointains}, ce qui revient à faire l'approximation que la matrice hessienne du potentiel est presque diagonale
\begin{equation}\label{eq:Hooke Pot}
        V = \frac{\kappa}{2}\sum_{i=1}^Nq_i^2 + \frac{\kappa}{2}\sum_{i=1}^N (q_i - q_{i+1})^2 +\mathcal{O}((q_i - q_{i+2})^{2})\, .
\end{equation} 
%où le prochain terme dans l'expansion nous donnerais le potentiel lié aux interactions voisines. 
On obtient donc le Lagrangien du système
\begin{equation}
        L = \sum_{i=1}^{N}\bigg(\frac{1}{2}m\dot{q}_i^2 - \frac{1}{2}m\omega^{2}q_i^2- \frac{1}{2}m\omega^2(q_i - q_{i+1})^2\bigg)\, .
\end{equation} 
Puisqu'on est libre de redéfinir les coordonnées canoniques, on absorbe le paramètres de masse dans les $\sqrt{m}q_i \rightarrow q_i$.
L'Hamiltonien est la transformée de Legendre du Lagrangien, avec les moments conjugués $p_i = \frac{\partial L}{\partial \dot{q}_i} = \dot{q}_i$, d'où
\begin{equation}
        H = \sum_{i=1}^N \bigg(\frac{p_i^2}{2} + \frac{1}{2}\omega^{2}q_i^2 + \frac{1}{2}\omega^2(q_i - q_{i+1})^2\bigg)
\end{equation} 

\subsection{Quantification canonique}
%On considère la quantification canonique de la chaîne discrète, soit un système
%quantique non-relativiste avec un nombre infini, mais discret, de degrés de libertés quantiques. 
%Nous sommes donc encore dans le contexte de la mécanique quantique habituelle.
%Rappel : {A, B} → −i[ Â, B̂], avec ~ = 1. Utilisez le point de vue de Schrödinger.
On performe la (seconde) quantification canonique de la chaîne harmonique discrète selon le point de vue de Schrödinger. 
On commence par la promotion des coordonnées canoniques à des opérateurs, 
$q_i \rightarrow \hat{q}_i$ et $p_i \rightarrow \hat{p}_i$. On assume qu'on connaît la relation des commutateurs canoniques (à temps égale selon notre point de vue), 
de sortes que les crochets de Poisson deviennent ($\hbar = 1$)
\begin{equation}\label{eq:ccr}
\begin{split}
        \{q_i, p_j\} = \delta_{ij} &\overset{\mathrm{QM}}{\longrightarrow } -i[\hat{q}_i, \hat{p}_j] = \delta_{ij}\\
        \{q_i, q_j\} = \{p_i, p_j\} = 0 &\overset{\mathrm{QM}}{\longrightarrow } [\hat{q}_i, \hat{q}_j] = [\hat{p}_i, \hat{p}_j] = 0
\end{split}
\end{equation} 
On exprime ensuite les opérateurs dans la base des modes de Fourier pour diagonaliser l'Hamiltonien
\begin{equation}\label{eq:op fourier}
\begin{split}
        \hat{q}_k &= \frac{1}{\sqrt{N}} \sum_{r=1}^{N} e^{2\pi i kr / N}\hat{Q}_r \\
        \hat{p}_k &= \frac{1}{\sqrt{N}}\sum_{r=1}^{N} e^{2\pi i kr / N}\hat{P}_r\, .
\end{split}
\end{equation} 
Les relations inverses sont données par
\begin{equation}
\begin{split}
        \hat{Q}_r &= \frac{1}{\sqrt{N}}\sum_{k=1}^{N} e^{-2\pi i kr / N}\hat{q}_k \\
        \hat{P}_r &= \frac{1}{\sqrt{N}}\sum_{k=1}^{N} e^{-2\pi i kr / N}\hat{p}_k\, .
\end{split}
\end{equation} 
Pour démontrer que ce choix diagonalise l'Hamiltonien, on utilise la définition du delta de Kronecker
\begin{equation}
        \delta_{mn} = \frac{1}{N}\sum_{k=1}^{N}e^{2\pi i k(m-n)/N}\, .
\end{equation} 
On observe que
\begin{align*}
        \sum_{k=1}^{N}\hat{q}_k^2 &= \sum_{k=1}^{N}\hat{q}_k \hat{q}^{\dagger}_k\\
         &= \frac{1}{N}\sum_{k=1}^{N}\left( \sum_{r=1}^{N} e^{2\pi i k r/N} \hat{Q}_r\right) \left( \sum_{n=1}^{N}e^{-2\pi i k n / N} \hat{Q}^{\dagger}_n \right) \\
         &= \frac{1}{N}\sum_{k=1}^{N}\sum_{r=1}^{N}\sum_{n=1}^{N} e^{2\pi i k (r-n)/N} \hat{Q}_r \hat{Q}^{\dagger}_n \\
         &= \sum_{r=1}^{N}\sum_{n=1}^{N} \delta_{rn} \hat{Q}_r \hat{Q}^{\dagger}_n \\
         &= \sum_{r=1}^{N} \hat{Q}_r \hat{Q}^{\dagger}_r \, .
\end{align*} 
et
\begin{align*}
        \sum_{k=1}^{N}(\hat{q}_k - \hat{q}_{k+1})^{2} &= \sum_{k=1}^{N}(\hat{q}_k - \hat{q}_{k+1})(\hat{q}_k - \hat{q}_{k+1})^{\dagger}\\
        &= \sum_{k=1}^{N}\hat{q}_k\hat{q}^{\dagger}_k + \sum_{k=1}^{N}\hat{q}_{k+1}\hat{q}^{\dagger}_{k+1} 
        - \sum_{k=1}^{N}(\hat{q}_{k}\hat{q}^{\dagger}_{k+1} + \hat{q}_{k+1}\hat{q}^{\dagger}_{k})
\end{align*} 
On impose une condition frontière périodique à la corde, de sortes que $\hat{q}_1 = \hat{q}_{N+1}$. On peut ainsi combiner les 
deux premières sommes pour obtenir
\begin{align*}
        \sum_{k=1}^{N}(\hat{q}_k - \hat{q}_{k+1})^{2} 
        &= 2\sum_{k=1}^{N}\hat{q}_k\hat{q}^{\dagger}_k 
        - \sum_{k=1}^{N}(\hat{q}_{k}\hat{q}^{\dagger}_{k+1} + \hat{q}_{k+1}\hat{q}^{\dagger}_{k})
\end{align*} 
Il s'avère que le terme croisé se simplifie de la façon suivante
\begin{align*}
        \sum_{k=1}^{N}\hat{q}_{k}\hat{q}^{\dagger}_{k+1} &= \frac{1}{N}\sum_{k=1}^{N}\sum_{r=1}^{N}\sum_{n=1}^{N}e^{2\pi ik(r-n)/N}e^{-2\pi i n/N}\hat{Q}_{r}\hat{Q}^{\dagger}_{n} \\
        &= \frac{1}{N}\sum_{r=1}^{N}\sum_{n=1}^{N} \delta_{rn} e^{-2\pi i n/N}\hat{Q}_{r}\hat{Q}^{\dagger}_{n} \\
        &= \sum_{r=1}^{N}\sum_{n=1}^{N} \delta_{rn} e^{-2\pi i n/N}\hat{Q}_{r}\hat{Q}^{\dagger}_{n} \\
        &= \sum_{r=1}^{N} e^{-2\pi i r/N}\hat{Q}_{r}\hat{Q}^{\dagger}_{r}
\end{align*}
En combinant les derniers résultats, et en utilisant l'identité d'Euler
\begin{align*}
        \sum_{k=1}^{N}(\hat{q}_k - \hat{q}_{k+1})^{2} &= 2\sum_{r=1}^{N}\hat{Q}_r\hat{Q}^{\dagger}_r - \sum_{r=1}^{N}( e^{-2\pi i r / N} + e^{2\pi i r / N})\hat{Q}_r\hat{Q}^{\dagger}_r \\
                                                      &= 2\sum_{r=1}^{N}\hat{Q}_r\hat{Q}^{\dagger}_r - 2\sum_{r=1}^{N} \cos \left( \frac{2 \pi r}{N} \right)\hat{Q}_r\hat{Q}^{\dagger}_r \, ,
\end{align*} 
on obtient l'Hamiltonien 
\begin{equation}
        \label{eq:H}
        \hat{H} = \sum_{r=1}^N \bigg(\frac{1}{2}\hat{P}_r\hat{P}^{\dagger}_r + \frac{1}{2}\omega_r^{2}\hat{Q}_r \hat{Q}^{\dagger}_r\bigg) \, .
\end{equation} 
On a définit
\begin{equation}
        \omega_r^{2} \equiv 3\omega^{2} - 4\omega^2 \sin^2 \left( \frac{\pi r}{N} \right) \, .
\end{equation} 
De cette définition, il suit que $\omega_r = \omega_{N-r}$.
%$\hat{H}$ est un opérateur diagonale dans l'espace des modes de Fourier.
Le commutateur entre les opérateurs canoniques est
\begingroup
\allowdisplaybreaks
\begin{align*}
        [\hat{Q}_r, \hat{P}_n] &= \frac{1}{N}\bigg[\sum_{k=1}^{N} e^{-2\pi i kr / N}\hat{q}_k , \sum_{k'=1}^{N} e^{-2\pi i k'n / N}\hat{p}_{k'} \bigg] \\
                             &= \frac{1}{N}\sum_{k=1}^{N}e^{-2\pi i kr / N}\sum_{k'=1}^{N} e^{-2\pi i k'n / N}\bigg[\hat{q}_k , \hat{p}_{k'} \bigg] \\
                             &= \frac{1}{N}\sum_{k=1}^{N}e^{-2\pi i kr / N}\sum_{k'=1}^{N} e^{-2\pi i k'n / N}(i\delta_{k k'}) \\
                             &= \frac{i}{N}\sum_{k=1}^{N}e^{-2\pi i k(r + n) / N}\\
                             &= \frac{i}{N}\sum_{k=1}^{N}e^{2\pi i k((-n) - r) / N}\\
                             &= i \delta_{r,-n}
\end{align*}
\endgroup
%Pour continuer, on doit déterminer la condition aux frontières de la chaîne harmonique. 
On utilise la condition périodique, de sortes que l'indice $-n$ est associé à $N-n$. 
%Cette condition périodique introduit une infinité de degrés de libertés, qu'on décrit de façon succincte 
%en ne traitant que l'ensemble des indices $i\,\, \mathrm{mod}\,\, N$. 
Ainsi,
\begin{equation}
        [\hat{Q}_r, \hat{P}_n] = i\delta_{r,N-n}
\end{equation} 
Puisque les opérateurs $\hat{q}_k$ et $\hat{p}_k$ sont des opérateurs hermitiens ($\hat{q}^{\dagger}_k = \hat{q}_k$), 
la transformée de Fourier est un opération unitaire si et seulement si
\begin{equation}
\begin{split}
        \hat{Q}^{\dagger}_r &= \hat{Q}_{N-r} \\
        \hat{P}^{\dagger}_r &= \hat{P}_{N-r} \\
\end{split}
\end{equation} 
De sortes qu'on peut trouver les commutateurs restants
\begin{equation}
\begin{split}
        [\hat{Q}_r, \hat{P}^{\dagger}_n] &= i\delta_{r,n} \\
        [\hat{Q}_r, \hat{Q}^{\dagger}_n] = [\hat{P}_r, \hat{P}^{\dagger}_n] &= 0
\end{split}
\end{equation} 
On introduit finalement l'opérateur d'échelle $\hat{a}_r$, définit en terme des opérateurs $\hat{Q}_r$ et $\hat{P}_r$.
Encore une fois, on assume une relation pour le commutateur canonique par correspondance avec le modèle classique avec 
un seul oscillateur harmonique
\begin{equation}
        [\hat{a}_r, \hat{a}^{\dagger}_n] = \delta_{rn}\, .
\end{equation} 
Avec cette supposition, on peut trouver un candidat pour l'opérateur d'échelle
\begin{equation}
        \hat{a}_r = \frac{1}{\sqrt{2\omega_r}}(\omega_r \hat{Q}_r + i\hat{P})
\end{equation} 
En effet
\begin{align*}
        [\hat{a}_r, \hat{a}^{\dagger}_n] &= \frac{1}{2 \sqrt{\omega_r \omega_n}} \left([\omega_r \hat{Q}_r + i\hat{P}_r, \omega_n \hat{Q}^{\dagger}_n - i\hat{P}^{\dagger}_n] \right)  \\[2ex]
              &= \frac{1}{2 \sqrt{\omega_r \omega_n}}\left( -i\omega_r[\hat{Q}_r,\hat{P}^{\dagger}_n] 
              + i\omega_n[\hat{P}_r,\hat{Q}^{\dagger}_n]  \right)  \\[2ex]
              &= \delta_{rn}\, ,
\end{align*}
tel que souhaité.
Toutefois, on évite d'identifier immédiatement les opérateurs d'échelle $\hat{a}_r$ et $\hat{a}^{\dagger}_r$ avec les opérateurs 
d'annihilation et de création respectivement. Leur interprétation physique demande plus de travail.
Pour compléter la seconde quantification de notre théorie discrète, on exprime les opérateurs canoniques et l'Hamiltonien en terme de $\hat{a}_r$ et $\hat{a}^{\dagger}_r$. 
Les opérateurs canoniques deviennent
\begin{align}
        \label{eq:Q2}
        \hat{Q}_r &= \frac{1}{\sqrt{2 \omega_r}} (\hat{a}_r + \hat{a}^{\dagger}_{N-r}) \\[2ex]
        \label{eq:P2}
        \hat{P}_r &= -i\sqrt{\frac{ \omega_r}{2}} (\hat{a}_r - \hat{a}^{\dagger}_{N-r})\, .
\end{align} 
On obtient l'Hamiltonien en remplaçant \eqref{eq:Q2} et \eqref{eq:P2} dans \eqref{eq:H}
\begin{align*}
        \hat{H} 
        &= \sum_{r=1}^N \bigg(\frac{1}{2}\frac{(-i^2)\omega_r}{2}(\hat{a}_r - \hat{a}^{\dagger}_{N-r})(\hat{a}^{\dagger}_r - \hat{a}_{N-r}) 
        +\frac{1}{2}\omega_r^{2}\frac{1}{2 \omega_r}(\hat{a}_r + \hat{a}^{\dagger}_{N-r})(\hat{a}^{\dagger}_r + \hat{a}_{N-r})\bigg) \\[2ex]
         &= \frac{1}{2}\sum_{r=1}^N \omega_r\big( \hat{a}_r \hat{a}^{\dagger}_r + \hat{a}^{\dagger}_{N-r}\hat{a}_{N-r}\big) \\
         \implies \Aboxed{ \hat{H}   &= \sum_{r=1}^N \omega_r\bigg( \hat{a}_r^{\dagger} \hat{a}_r + \frac{1}{2}\delta_{rr}\bigg)}
\end{align*} 
%et on donne les équations de mouvement pour une solution de l'équation de Schrödinger $\psi($

\subsection{Limite du continuum}
%Prenez ensuite la limite du continu de cette théorie. Donnez toutes les relations de com-
%mutation entre les opérateurs de champ issus des opérateurs φ̂ I et p̂ I . Ces commutateurs
%respectent-ils la causalité ? Ne vous inquiétez pas trop de la rigueur mathématique de l’ex-
%pansion de Taylor pour des opérateurs.
On prend maintenant la limite du continuum pour notre théorie, ç.-à-d.~qu'on prend $N \rightarrow \infty $, tout en 
prenant simultanément les limites $m \rightarrow 0$ et $\kappa \rightarrow \infty $ pour guarder la fréquence d'oscillation, $\omega$, constante.
Pour ce faire, on doit changer la description des états du système en terme des coordonnées canoniques $q_i$ et $p_i$ 
en faveur de la description en terme des champs $\phi(x)$ et $\pi(x)$, où $x \in \mathbb{R}$ est l'indice continu qui remplace l'indice discret $i \in \{1,\dots,N\}$. 
%En fait, on associe ces champs avec une densité et une densité d'impulsion respectivement.
%, de sortes qu'on absorbe la masse des atomes $m$ dans  
%la définition des champs. 
On introduit $P$ comme la période de $\phi(x)$, soit la longueur de la chaîne harmonique.
%On introduit le delta de Dirac
%\begin{equation}
        %\delta(x - y) = \frac{1}{2\pi}\int_{-\infty}^{\infty}
%\end{equation} 


Par correspondance avec les relations de commutations pour les opérateurs canoniques \eqref{eq:ccr}, on a
\begin{equation}\label{eq:ccr2}
\begin{split}
        \{\phi(x), \pi(y)\} = 2\pi\delta(x - y) &\overset{\mathrm{QM}}{\longrightarrow } -i[\hat{\phi}(x), \hat{\pi}(y)] = 2\pi \delta(x - y)\\
        \{\phi(x), \phi(y)\} = \{\pi(x), \pi(y)\} = 0 &\overset{\mathrm{QM}}{\longrightarrow } [\hat{\phi}(x), \hat{\phi}(y)] = [\hat{\pi}(x), \hat{\pi}(y)] = 0
\end{split}
\end{equation} 
où on a introduit le delta de Dirac $\delta(x) = \frac{1}{2\pi}\int_{-\infty }^{\infty }dk\, e^{ikx}$.
Dans la description continue, on suppose que les champs $\phi(x)$ et $\pi(x)$ sont périodiques, de sortes qu'on 
peut prendre leur transformée de Fourier de la façon suivante
\begin{equation}
\begin{split}
        \hat{ \phi}(x) &= \frac{1}{2\pi} \int_{-\infty }^{\infty} dk\, e^{i kx} \hat{\phi}(k) \\
        \hat{\pi}(x) &= \frac{1}{2\pi}\int_{-\infty }^{\infty } dk\, e^{i kx} \hat{\pi}(k)\, .
\end{split}
\end{equation} 
Les relations inverses sont
\begin{equation}
\begin{split}
        \hat{\phi}(k) &=  \int_{-\infty }^{\infty}dx\,  e^{- i kx} \hat{\phi}(x) \\
        \hat{\pi}(k) &= \int_{-\infty }^{\infty } dx\,e^{- i kx} \hat{\pi}(x)\, .
\end{split}
\end{equation} 
On a choisit une normalisation différente pour faire correspondre nos résultats avec ceux du cours.
Par correspondance avec la définition des opérateurs canoniques \eqref{eq:Q2} et \eqref{eq:P2}, on a de plus
\begin{align}
        \label{eq:Q2c}
        \hat{\phi}(k) &= \frac{1}{\sqrt{2\omega_k}} (\hat{a}_k + \hat{a}^{\dagger}_{-k}) \\[2ex]
        \label{eq:P2c}
        \hat{\pi}(k) &= -i\sqrt{\frac{\omega_k}{2}} (\hat{a}_k - \hat{a}^{\dagger}_{-k})\, ,
\end{align} 
avec l'opérateur d'échelle
\begin{align}
        \hat{a}_k = \frac{1}{\sqrt{2 \omega_k}}(\omega_k \phi(k) + i \pi(k)) \, .
\end{align}
Ainsi, on a les commutateurs canoniques
\begin{align}
        [\hat{\phi}(k), \hat{\pi}(k')] &= 2\pi i\delta(k + k') \\
        [\hat{\phi}(k), \hat{\pi}^{\dagger}(k')] &= 2\pi i\delta(k - k') \\
        [\hat{\phi}(k), \hat{\phi}(k')] = [\hat{\pi}(k), \hat{\pi}(k')]  &= 0 \\
        [\hat{a}_k, \hat{a}^{\dagger}_{k'}] &= 2\pi \delta(k - k') \\
        [\hat{a}_k, \hat{a}_{k'}] = [\hat{a}^{\dagger}_k,\hat{a}^{\dagger}_{k'}] &= 0
\end{align}
L'Hamiltonien devient
\begin{equation}
        \hat{H} = \int_{\mathbb{R}}\frac{dk}{2\pi}\, \left(  \frac{\pi(k)\pi^{\dagger}(k)}{2} + \omega_k^{2}\phi(k)\phi^{\dagger}(k)\right) = \int_{\mathbb{R}}\frac{dk}{2\pi}\, \omega_k\big(\hat{a}^{\dagger}_k \hat{a}_k + \frac{1}{2}2\pi\delta(0)\big)
\end{equation} 
On calcule maintenant les relations de commutations avec l'Hamiltonien dans son espace propre. 
On commence par calculer la commutation avec l'opérateur d'échelle $\hat{a}_k$
\begin{align*}
        [\hat{H}, \hat{a}_k] &= \int_{\mathbb{R}} \frac{dk'}{2\pi} \omega_{k'}[\hat{a}^{\dagger}_{k'} \hat{a}_{k'}, \hat{a}_{k}] \\
        &= \int_{\mathbb{R}} \frac{dk'}{2\pi} \omega_{k'}\left( \hat{a}^{\dagger}_{k'}[ \hat{a}_{k'}, \hat{a}_{k}] + [\hat{a}^{\dagger}_{k'} , \hat{a}_{k}] \hat{a}_{k'}\right)\\
        &= \int_{\mathbb{R}} \frac{dk'}{2\pi} \omega_{k'}(-2\pi) \delta(k' - k)\hat{a}_{k'}\\
        \implies  \Aboxed{ [\hat{H}, \hat{a}_k]  &= -\omega_k \hat{a}_{k}}
\end{align*}
Par un argument similaire, on a
\begin{equation}
        \boxed{[\hat{H}, \hat{a}^{\dagger}_k] = \omega_k\hat{a}^{\dagger}_k}
\end{equation} 
On trouve ainsi que les opérateurs d'échelles sont les vecteurs propres de l'Hamiltonien, ce qui correspond bien à notre image de 
l'oscillateur harmonique simple où l'opérateur d'échelle est utilisé pour construire les modes de différentes énergies. En effet, si on suppose que 
$\hat{H}\psi = E \psi$ est une solution de l'équation de Schrödinger, alors $\hat{a}^{\dagger}_k \psi$ est aussi une solution avec une énergie $E_k = \omega_k + E$:
\begin{equation}
        \hat{H}\hat{a}^{\dagger}_k \psi = (\omega_k+ E)\hat{a}^{\dagger}_k \psi
\end{equation} 
Ensuite, on considère la commutation avec le champ $\phi(k)$
\begingroup
\allowdisplaybreaks
\begin{align*}
        [\hat{H}, \hat{\phi}(k)] &=\int_{\mathbb{R}} \frac{dk'}{2\pi}  \frac{\omega_{k'}}{\sqrt{2\omega_{k}}}[\hat{a}^{\dagger}_{k'} \hat{a}_{k'}, \hat{a}_{k} + \hat{a}^{\dagger}_{-k}] \\
        &= \int_{\mathbb{R}} \frac{dk'}{2\pi} 
         \frac{\omega_{k'}}{\sqrt{2\omega_{k}}}
        \left(  
                \hat{a}^{\dagger}_{k'}[\hat{a}_{k'}, \hat{a}_{k} + \hat{a}^{\dagger}_{-k}] 
                + [\hat{a}^{\dagger}_{k'} , \hat{a}_{k} + \hat{a}^{\dagger}_{-k}]\hat{a}_{k'} 
        \right)\\
        &= \int_{\mathbb{R}} \frac{dk'}{2\pi} 
         \frac{\omega_{k'}}{\sqrt{2\omega_{k}}}
        \left(  
                \hat{a}^{\dagger}_{k'} 2\pi\delta(k' + k) 
                - 2\pi \delta(k' - k)\hat{a}_{k'}
        \right)\\
        &= \sqrt{\frac{\omega_k}{2}} \left(  \hat{a}^{\dagger}_{-k}   - \hat{a}_{k} \right) \\
        %&= -i \hat{\pi}(k)\\
        \implies \Aboxed{[\hat{H}, \hat{\phi}(k)]  &= -i\hat{\pi}(k)}\, ,
\end{align*}
\endgroup
et $\pi(k)$
\begin{align*}
        [\hat{H}, \hat{\pi}(k)] &= -i \int_{\mathbb{R}} \frac{dk'}{2\pi} \omega_{k'} \sqrt{\frac{\omega_k}{2}}[\hat{a}^{\dagger}_{k'} \hat{a}_{k'}, \hat{a}_{k} - \hat{a}^{\dagger}_{-k}] \\
        &= -i \int_{\mathbb{R}} \frac{dk'}{2\pi} 
\omega_{k'} \sqrt{\frac{\omega_k}{2}}
        \left(  
                \hat{a}^{\dagger}_{k'}[\hat{a}_{k'}, \hat{a}_{k} - \hat{a}^{\dagger}_{-k}] 
                + [\hat{a}^{\dagger}_{k'} , \hat{a}_{k} - \hat{a}^{\dagger}_{-k}]\hat{a}_{k'} 
        \right)\\
        &=-i \int_{\mathbb{R}} \frac{dk'}{2\pi} 
\omega_{k'} \sqrt{\frac{\omega_k}{2}}
        \left(  
                -\hat{a}^{\dagger}_{k'} 2\pi\delta(k' + k) 
                - 2\pi \delta(k'- k)\hat{a}_{k'}
        \right)\\
        &= i \omega_k\sqrt{\frac{\omega_k}{2}} \left(  \hat{a}^{\dagger}_{-k}  + \hat{a}_{k} \right) \\
        %&= i \omega^2\hat{\phi}(-k)\\
        \implies \Aboxed{[\hat{H}, \hat{\pi}(k)]  &= i\omega^{2}_k\hat{\phi}(k)}\, . 
\end{align*}
Par soucis de complétude, on dérive aussi les commutateurs pour le champs adjoint $\phi^{\dagger}(k)$
\begin{align*}
        [\hat{H}, \hat{\phi}^{\dagger}(k)] %&= \frac{\omega}{\sqrt{2\omega}}\int_{\mathbb{R}} \frac{dk'}{2\pi} [\hat{a}^{\dagger}_{k'} \hat{a}_{k'}, \hat{a^{\dagger}}_{k} + \hat{a}{-k}] \\
        &= \int_{\mathbb{R}} \frac{dk'}{2\pi} 
         \frac{\omega_{k'}}{\sqrt{2\omega_{k}}}
        \left(  
                \hat{a}^{\dagger}_{k'}[\hat{a}_{k'}, \hat{a}^{\dagger}_{k} + \hat{a}_{-k}] 
                + [\hat{a}^{\dagger}_{k'} , \hat{a}^{\dagger}_{k} + \hat{a}_{-k}]\hat{a}_{k'} 
        \right)\\
        &= \int_{\mathbb{R}} \frac{dk'}{2\pi} 
         \frac{\omega_{k'}}{\sqrt{2\omega_{k}}}
        \left(  
                \hat{a}^{\dagger}_{k'} 2\pi\delta(k'- k) 
                - 2\pi \delta(k' + k)\hat{a}_{k'}
        \right)\\
        &= \sqrt{\frac{\omega_k}{2}} \left(  \hat{a}^{\dagger}_{k}   - \hat{a}_{-k} \right) \\
        &= -i \hat{\pi}(-k)\\
        \implies \Aboxed{[\hat{H}, \hat{\phi}^{\dagger}(k)]  &= -i\hat{\pi}^{\dagger}(k)}\, ,
\end{align*}
et $\pi^{\dagger}(k)$
\begin{align*}
        [\hat{H}, \hat{\pi}(k)] %&=-i \omega \sqrt{\frac{\omega}{2}}\int_{\mathbb{R}} \frac{dk'}{2\pi} [\hat{a}^{\dagger}_{k'} \hat{a}_{k'}, \hat{a}^{\dagger}_{k} - \hat{a}_{-k}] \\
        &= -i  \int_{\mathbb{R}} \frac{dk'}{2\pi} 
\omega_{k'} \sqrt{\frac{\omega_k}{2}}
        \left(  
                \hat{a}^{\dagger}_{k'}[\hat{a}_{k'}, \hat{a}^{\dagger}_{k} - \hat{a}_{-k}] 
                + [\hat{a}^{\dagger}_{k'} , \hat{a}^{\dagger}_{k} - \hat{a}_{-k}]\hat{a}_{k'} 
        \right)\\
        &=-i \int_{\mathbb{R}} \frac{dk'}{2\pi} 
\omega_{k'} \sqrt{\frac{\omega_k}{2}}
        \left(  
                -\hat{a}^{\dagger}_{k'} 2\pi\delta(k' - k) 
                - 2\pi \delta(k'+ k)\hat{a}_{k'}
        \right)\\
        &= i \omega_k\sqrt{\frac{\omega_k}{2}} \left(  \hat{a}^{\dagger}_{k}  + \hat{a}_{-k} \right) \\
        &= i \omega^2_k\hat{\phi}(-k)\\
        \implies \Aboxed{[\hat{H}, \hat{\pi}^{\dagger}(k)]  &= i\omega^{2}_k\hat{\phi}^{\dagger}(k)}\, . 
\end{align*}

On calcule finalement les relations de commutations pour l'observable $\hat{\phi}(x)$ 
\begin{align*}
        [\hat{H}, \hat{\phi}(x)] &= \int_{\mathbb{R}} \frac{dk}{2\pi} e^{ikx}[\hat{H},\hat{\phi}(k)]\\ 
        &= -i \int_{\mathbb{R}} \frac{dk}{2\pi} e^{ikx} \hat{\pi}(k) \\
        \implies \Aboxed{[\hat{H}, \hat{\phi}(x)] &= -i \hat{\pi}(x) }
\end{align*}
Pour $\hat{\pi}(x)$, il est préférable de calculer le commutateur avec l'Hamiltonien dans l'espace physique pour éviter l'apparition de $\omega_k$ dans l'intégrale de phase. 
On fait donc la correspondance avec l'Hamiltonien discret
\begin{equation}
        \hat{H} = \int_{\mathbb{R}}dx\, \left( \frac{\hat{\pi}^{2}(x)}{2} + \frac{1}{2}\hat{\phi}(x)\left( \omega^2 - \partial_{x}^2\right)\hat{\phi}(x) \right)
\end{equation} 
où on a associer le terme $\propto (\hat{q}_i - \hat{q}_{i+1})^2$ au gradient $-\partial_x^2 \phi(x)$. En utilisant les commutateurs canoniques, 
on obtient
\begin{align*}
        [\hat{H}, \hat{\pi} (x)] &= \int_{\mathbb{R}}dx'\,\bigg[\frac{\hat{\pi}^{2}(x')}{2} + \frac{1}{2}\hat{\phi}(x')\left( \omega^2 - \partial_{x}^2\right)\hat{\phi}(x'),\hat{\pi}(x)\bigg]\\ 
        &= \int_{\mathbb{R}}dx'\,(\omega^2 - \partial_x^2)\phi(x')  i\delta(x - x')\\ 
        \implies \Aboxed{ [\hat{H}, \hat{\pi} (x)] &= i(\omega^2 - \partial_x^2)\phi(x)  }
\end{align*}

Ainsi, on a tous les outils en mains pour s'intéresser à la causalité de la théorie. On s'attend à ce que le commutateur entre deux opérateurs 
hermitiens, soit des observables de la théorie, commutent lorsque la séparation entre deux évènements, disons $\mathcal{E}_1 =(x, t)$ et $\mathcal{E}_2 = (x', t')$, est de type spatiale, 
donc $(\mathcal{E}_1 - \mathcal{E}_2)^2 < 0$. On a déjà déterminer les commutateurs à temps égaux ($t = t'$). On prend maintenant le point de vue de 
Heisenberg, où l'état fondamental $| 0 \rangle $ est indépendant du temps et où on fait évoluer un opérateur $\hat{A}$ avec l'opérateur de 
d'évolution temporelle %$\hat{U}(t) = e^{-i\hat{H}t}$
\begin{equation}
        \hat{A}(t) = e^{i\hat{H}t}\hat{A}_S e^{-i\hat{H}t}\, .
\end{equation} 
$\hat{A}_S$ est l'opérateur selon le point de vue de Schrödinger. Au fins de l'exercice, on suppose que $t$ et $t'$ sont petits, de sortes qu'on peut prendre 
l'expansion de Taylor de $\hat{A}(t)$ autour de $\hat{A}_S$
\begin{equation}
        \hat{A}(t) = \hat{A}_S + it[\hat{H}, \hat{A}_S] - \frac{1}{2}t^2 [\hat{H}, [\hat{H}, \hat{A}_S]] + \mathcal{O}(t^3)\, .
\end{equation} 
%Si on considère en plus les translation spatiales
%On s'intéresse en premier lieu au commutateur 
Ainsi, on peut calculer le commutateur entre deux évènements de l'espace-temps pour le champs $\hat{\phi}$ pour une expansion au premier ordre
\begin{align*}
        [\phi(x, t), \phi(x', t')] &= \langle 0 | [\phi(x, t), \phi(x', t') ] | 0 \rangle \\
                                   &\approx \langle 0 | [\phi(x) + it[\hat{H}, \phi(x)], \phi(x') + it'[\hat{H}, \phi(x')]]| 0 \rangle \\
                                   &= \langle 0 | [\phi(x) + t\pi(x), \phi(x') + t'\pi(x')]| 0 \rangle \\
                                   &= 2\pi i (t' - t) \delta(x - x')\, .
\end{align*} 
Le commutateur est nul dès que $x \not= x'$ ou que $t = t'$, de sortes que la causalité est préservée.
En effet, on trouve que l'amplitude de propager une particule d'un point $x$ à $x'$ est toujours nulle, peut importe la différence temporelle.
On conclut que les particules vibrent sur place, sans pouvoir se propager (au moins pour un temps très court). 
%Puisqu'on a construit la théorie en supposant un potentiel sans terme d'interaction local, 
%alors la théorie est devenue \textit{hyper}-local, dans le sens où chaque point de l'espace vibre indépendamment des autres, et qu'aucune action à un endroit de l'espace ne peut affecter un 
%autre endroit de l'espace. Évidemment, un évènement passé peut influencer un évènement futur, à condition que $x = x'$. 

%Pour permettre a la théorie d'avoir une description plus riche, et moins locale, on doit au moins introduire les termes 
%d'interactions voisins dans le potential, ç.-à-d.~faire l'expansion du potentiel de Hooke \eqref{eq:Hooke Pot} jusqu'à l'ordre $\mathcal{O}((q_i + q_{i+2})^2)$.
%Ceci fait apparaître un terme $\propto (\partial_x \phi)^2$ dans la théorie, 
%dans quel cas on recouvre la théorie de Klein-Gordon couverte dans le cours. De façon intéressante, toutes les étapes de la (seconde) quantification ont des résultats identiques  
%pour cette théorie, exceptés les commutateurs entre l'Hamiltonien et les observables et le fait que $\omega_k$ doit être indexé avec l'indice des modes $k$ (et donc ne peut pas sortir de intégrale sur les $k$). 
%En particulier, l'expansion au second 
%ordre du commutateur calculé plus haut 



\subsection{}
%Évaluez ∂ t 2 h φ̂(x)i t à l’aide de la théorie discrète. Passez au continu à la fin du calcul.
%Votre résultat est-il relié à l’équation de Klein-Gordon ? Expliquez.
%On cherche à évaluer $\partial_t^2 \langle  \hat{\phi}(x) \rangle_t $ à l'aide de la théorie discrète.

\end{document}

