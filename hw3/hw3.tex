\documentclass{article}
\usepackage[a4paper, margin=2cm]{geometry}

\usepackage{amsmath}
\usepackage{amssymb}
\usepackage{mathtools}
\usepackage{amstext}
\usepackage{amsthm}
\usepackage{fancyhdr}
\usepackage[utf8]{inputenc} % allow utf-8 input
\usepackage[T1]{fontenc}    % use 8-bit T1 fonts


\usepackage{graphicx}
\usepackage{float}
\usepackage{caption}
\usepackage{subcaption}
\usepackage{booktabs}
\usepackage{physics, tensor}
\usepackage{slashed}

\graphicspath{{figures/}}

\pagestyle{fancy}
\rhead{Alexandre Adam\\ 20090755}
\lhead{William Witczak-Krempa \\ PHY 6812: Théorie des champs 1}
\chead{Devoir 3}
\rfoot{23 novembre 2022}
\cfoot{\thepage}

\newcommand{\angstrom}{\textup{\AA}}
\numberwithin{equation}{section}
\renewcommand\thesubsection{\alph{subsection})}
\renewcommand\thesubsubsection{\Roman{subsubsection}}
\newcommand{\s}{\hspace{0.1cm}}
\DeclareRobustCommand{\bbzero}{\text{\usefont{U}{bbold}{m}{n}0}}
\DeclareRobustCommand{\bbone}{\text{\usefont{U}{bbold}{m}{n}1}}


\theoremstyle{solution}
\newtheorem{solution}{Réponse}[section]

\renewcommand*{\proofname}{Solution}


\begin{document}
\section{Représentations projectives}
On cherche à montrer que les spineurs de Weyl et Dirac forment des représentations projectives du groupe
de Lorentz. 
En d’autres termes, montrez que ces représentations admettent au moins un 2-cocycle non-trivial.

\section{Charge d’un spineur}
\subsection{}
On considère une transformation interne agissant sur un spineur de Dirac
\begin{equation}\label{eq:t_interne}
	\begin{split}
		\psi(x) &\rightarrow e^{i\theta}\psi(x)
	\end{split}
\end{equation} 
où $\theta \in \mathbb{R}$. On cherche à savoir si cette transformation est une symétrie du Lagrangien de Dirac
\begin{equation}\label{eq:dirac_L}
	\mathcal{L} = \bar{\psi}(i \slashed{\partial} - m)\psi
\end{equation} 

\begin{solution}
La transformation interne \eqref{eq:t_interne}, soit une transformation $U(1)$, est une symétrie du Lagrangien de Dirac.	
\end{solution}
\begin{proof}
La preuve suit par calcul direct
\begin{align*}
        \mathcal{L} \rightarrow \mathcal{L}'  &= \bar{\psi}'  (i \slashed{\partial} - m) \psi' \\
                &= e^{-i\theta} \psi^{\dagger} \gamma^0(i \slashed{\partial} - m) e^{i\theta} \psi \\
                &= \bar{\psi} (i \slashed{\partial} - m) e^{i\theta} \psi \\
                &= \mathcal{L}
\end{align*}	
On a utilisé le fait que $\theta \in \mathbb{R}$ est un paramètre constant et global, donc $e^{-i\theta}$ commute avec 
la dérivée $\partial_\mu$.
\end{proof}

\begin{solution}
Le courant de Noether associé à la transformation interne \eqref{eq:t_interne} est (à une phase $\pm 1$ près)
\begin{equation}
        \boxed{j^{\mu} = \bar{\psi} \gamma^{\mu} \psi}
\end{equation} 
\end{solution}
\begin{proof}
Soit la transformation interne infinitésimale
\begin{equation}
        \psi \rightarrow  (1 - i \theta) \psi
\end{equation} 
La transformation infinitésimale du spineur de Dirac est donc 
\begin{equation}
        \delta \psi = -i \psi
\end{equation} 
Il suit de la définition du courant de Noether que
\begin{align*}
        j^{\mu} &= \frac{\partial \mathcal{L}}{\partial (\partial_{\mu} \psi)} \delta \psi \\
        &= \left(  \frac{\partial }{\partial (\partial_{\mu}\psi)} \bar{\psi}(i \gamma^{\nu} \partial_\nu - m) \psi\right) \delta\psi\\
        &= \bar{\psi}\gamma^{\nu} \delta^{\mu}_\nu \psi \\
        &= \bar{\psi}\gamma^{\mu}\psi
\end{align*} 
\end{proof}

\subsection{}
On considère le Lagrangien de Dirac avec masse nulle, $m = 0$. 
On utilise la représentation de Weyl (ou représentation chirale) pour représenter les spineurs, soit
\begin{equation}
        \psi = \begin{pmatrix}
                \psi_L \\ \psi_R
        \end{pmatrix}\, ,
\end{equation} 
et
\begin{equation}
        \gamma^0 = \begin{pmatrix}
                \bbzero & \bbone\\
                \bbone& \bbzero
        \end{pmatrix}
        , 
        \hspace{1cm}
        \gamma^{i} = 
        \begin{pmatrix}
                \bbzero & \sigma^{i} \\
                \bar{\sigma}^{i}& \bbzero
        \end{pmatrix}\, .
\end{equation} 
où $\sigma^{i}$ sont les matrices de Pauli et $\bar{\sigma} = -\sigma^{i}$.
Dans ce cas, une symétrie du Lagrangien émerge du fait que les 
représentations chirales se découples dans l'équation de Dirac. En effet, on obtient les équations de Weyl
\begin{equation}
         i\gamma^{\mu}\partial_\mu \psi = 
         \begin{pmatrix}
                 i \sigma^{\mu} \partial_\mu \psi_R \\[1ex]
                 i \bar{\sigma}^{\mu}\partial_\mu \psi_L 
         \end{pmatrix}
         = 0\, .
\end{equation} 
De ce fait, le Lagrangien se découple en deux termes
\begin{align*}
        \mathcal{L} &=  i\bar{\psi} \gamma^{\mu}\partial_\mu \psi \\
        &= i
        \begin{pmatrix}
                \psi_L^{\dagger} \\
                \psi_R^{\dagger}
        \end{pmatrix}
        \gamma^0
         \begin{pmatrix}
                 \sigma^{\mu} \partial_\mu \psi_R \\[1ex]
                 \bar{\sigma}^{\mu}\partial_\mu \psi_L 
         \end{pmatrix} \\
         &= 
        \begin{pmatrix}
                \psi_R^{\dagger} \\
                \psi_L^{\dagger}
        \end{pmatrix}
         \begin{pmatrix}
                 \sigma^{\mu} \partial_\mu \psi_R \\[1ex]
                 \bar{\sigma}^{\mu}\partial_\mu \psi_L 
         \end{pmatrix} \, .
\end{align*}
D'où
\begin{equation}
        \mathcal{L} = i \psi_L^{\dagger} \bar{\sigma}^{\mu} \partial_\mu \psi_L + i \psi_r^{\dagger} \sigma^{\nu}\partial_\nu \psi_R \, .
\end{equation}
Ainsi, pour des particules de masses nulles, le Lagrangien a maintenant deux symétries internes, soit $U(1)\times U(1)$, donc deux 
courants de Noether
\begin{equation}
        \boxed{j^{\mu}_s = \bar{\psi}_s \sigma^{\mu}_s \psi_s}\, ,
\end{equation} 
où $\sigma^{\mu}_R = \sigma^{\mu}$ et $\sigma^{\mu}_L = \bar{\sigma}^{\mu}$. Ces courants sont valides strictement lorsque les représentations chirales 
sont strictement indépendantes, et n'est donc pas valide dans une limite ultra relativiste. 

En effet, la symétrie du Lagrangien dans la limite où $m \rightarrow  0$, soit une limite ultra relativiste, est légèrement différente. 
Dans ce cas, on peut construire la symétrie axiale
\begin{equation}\label{eq:t_axiale}
        \psi(x) \rightarrow e^{i \theta \gamma^{5}} \psi(x) \, ,
\end{equation} 
où, dans la représentation de Weyl, 
\begin{equation}
        \gamma^{5} \equiv i\gamma^0\gamma^{1}\gamma^{2}\gamma^{3} = 
        i
        \begin{pmatrix}
                -\sigma^{1}\sigma^{2}\sigma^{2} & \bbzero \\
                \bbzero & \sigma^{1}\sigma^{2}\sigma^{3}
        \end{pmatrix}
        =
        \begin{pmatrix}
                -\bbone & \bbzero \\ 
                \bbzero & \bbone
        \end{pmatrix}\, .
\end{equation} 
On peut montrer que cette transformation est une symétrie du Lagrangien de Dirac en utilisant le fait que $(\gamma^{5})^{\dagger} = \gamma^{5}$, 
$(\gamma^{5})^{2} = \bbone$ et 
la formule d'Euler qui résulte de ce fait
\begin{equation}
        e^{i\theta \gamma^{5}} = \cos(\theta) + i \gamma^{5}\sin(\theta)
\end{equation} 
On utilise aussi la relation d'anti-commutation
\begin{equation}
        \{ \gamma^{5}, \gamma^{\mu} \} = 0\, ,
\end{equation} 
de sortes que $e^{i \theta \gamma^{5}} \gamma^{\mu} = \gamma^{\mu}e^{-i \theta \gamma^{5}}$ et
\begin{align*}
        \mathcal{L} \rightarrow  \mathcal{L}' &=\psi^{\dagger} e^{-i \theta \gamma^{5}}\gamma^0  (i\gamma^\mu \partial_\mu - m) e^{i\theta \gamma^{5}}\psi \\
                &=\psi^{\dagger} \gamma^0  e^{i \theta \gamma^{5}}(i\gamma^\mu \partial_\mu - m) e^{i\theta \gamma^{5}}\psi \\
                &= \bar{\psi} (i\gamma^\mu \partial_\mu - me^{2i\theta \gamma^{5}}) \psi \\
\end{align*}
Ainsi, dans la limite $m \rightarrow 0$, la transformation axiale \eqref{eq:t_axiale} est une symétrie du Lagrangien, qui correspond effectivement 
à deux symétries $U(1)$ où les paramètres des transformations correspondent à $\theta_R = -\theta_L$.  
On peut calculer le courant de Noether qui correspond à cette symétrie. On considère la transformation infinitésimale
\begin{equation}
        \psi(x) \rightarrow  (1 - i \theta \gamma^{5}) \psi(x)\, ,
\end{equation} 
d'où
\begin{equation}
        \delta \psi = -i \gamma^{5}
\end{equation} 
et
\begin{equation}
        \boxed{j^{\mu} = \bar{\psi} \gamma^{\mu}\gamma^{5}\psi}
\end{equation} 

\section{Invariance d’échelle}
On considère le champ de Dirac avec $d=3$ dimensions spatiales. On considère la transformation d'échelle
\begin{equation}\label{eq:t_echelle}
        \begin{split}
                x &\rightarrow  b x \\
                \psi(x) &\rightarrow b^{-\Delta} \psi(x)
        \end{split}
\end{equation} 
où $b \in \mathbb{R}_{>0}$ et $\Delta \in \mathbb{R}$.

\subsection{}
\textbf{Quelle sont les conditions pour que \eqref{eq:t_echelle} soit une symétrie de la théorie? Appelons l’action
de cette théorie $S_{\star}$. Quel est le courant de Noether associé?} \\
On ne considère que le cas avec $d = 3$ dimensions spatiales.
La transformation \eqref{eq:t_echelle} est une symétrie de la théorie si l'action 
\begin{equation}\label{eq:action}
        S_\star = \int d^{4}x\, \bar{\psi}(x)(i \slashed{\partial} - m)\psi(x)
\end{equation} 
est invariante sous l'application de la transformation. Puisque l'élément de volume devient
\begin{equation}
        d^{4}x \rightarrow b^{4}(d^{4}x)\, ,
\end{equation} 
et que l'opérateur $\slashed{\partial} \rightarrow b^{-1} \slashed{\partial}$ possède une facteur d'échelle $\Delta_{\slashed{\partial}} = 1$, alors 
la transformation d'échelle est une symétrie du Lagrangien si et seulement si
\begin{equation}
        \boxed{
        \begin{split}
              m = 0 \\
              \Delta_\psi = \frac{3}{2}
        \end{split}
}
\end{equation} 
On pose $b = 1 - \epsilon$, où $|\epsilon| \ll 1$. Ainsi, 
\begin{equation}
        \psi(x) \rightarrow  (1 + \epsilon\Delta ) \psi(x)
\end{equation} 
On prend le point de vue d'une transformation active du champ
\begin{equation}
        \delta \psi = \psi'(x) - \psi(x) \, .
\end{equation} 
On doit donc déterminer la quantité
\begin{align*}
        \psi'(x) &=  (1 + \epsilon\Delta ) \psi(x + \epsilon x) + \mathcal{O}(\epsilon^{2}) \\
\end{align*}
Or, pour simplifier ce terme, on doit déterminer la dérivée de Lie du champs spineur $\psi(x)$, ce qui est non-trivial.
\section{Théorie Yukawa classique}

\section{Opérateurs qui anti-commutent}

\section{Fermions Marojana (Peskins \& Shcroeder 3.4)}
\end{document}

