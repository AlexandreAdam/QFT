\documentclass{article}
\usepackage[a4paper, margin=2cm]{geometry}

\usepackage{amsmath}
\usepackage{amssymb}
\usepackage{mathtools}
\usepackage{amstext}
\usepackage{amsthm}
\usepackage{fancyhdr}
\usepackage[utf8]{inputenc} % allow utf-8 input
\usepackage[T1]{fontenc}    % use 8-bit T1 fonts


\usepackage{graphicx}
\usepackage{float}
\usepackage{caption}
\usepackage{subcaption}
\usepackage{booktabs}
\usepackage{physics, tensor}
\usepackage{slashed}

\graphicspath{{figures/}}

\pagestyle{fancy}
\rhead{Alexandre Adam\\ 20090755}
\lhead{William Witczak-Krempa \\ PHY 6812: Théorie des champs 1}
\chead{Devoir 3}
\rfoot{23 novembre 2022}
\cfoot{\thepage}

\newcommand{\angstrom}{\textup{\AA}}
\numberwithin{equation}{section}
\renewcommand\thesubsection{\alph{subsection})}
\renewcommand\thesubsubsection{\Roman{subsubsection}}
\newcommand{\s}{\hspace{0.1cm}}
\DeclareRobustCommand{\bbzero}{\text{\usefont{U}{bbold}{m}{n}0}}
\DeclareRobustCommand{\bbone}{\text{\usefont{U}{bbold}{m}{n}1}}


\theoremstyle{solution}
\newtheorem{solution}{Réponse}[section]

\renewcommand*{\proofname}{Solution}


\begin{document}
\section{Représentations projectives}
On cherche à montrer que les spineurs de Weyl et Dirac forment des représentations projectives du groupe
de Lorentz. 
En d’autres termes, montrez que ces représentations admettent au moins un 2-cocycle non-trivial.

\section{Charge d’un spineur}
\subsection{}
On considère une transformation interne agissant sur un spineur de Dirac
\begin{equation}\label{eq:t_interne}
	\begin{split}
		\psi(x) &\rightarrow e^{i\theta}\psi(x)
	\end{split}
\end{equation} 
où $\theta \in \mathbb{R}$. On cherche à savoir si cette transformation est une symétrie du Lagrangien de Dirac
\begin{equation}\label{eq:dirac_L}
	\mathcal{L} = \bar{\psi}(i \slashed{\partial} - m)\psi
\end{equation} 

\begin{solution}
La transformation interne \eqref{eq:t_interne}, soit une transformation $U(1)$, est une symétrie du Lagrangien de Dirac.	
\end{solution}
\begin{proof}
La preuve suit par calcul direct
\begin{align*}
        \mathcal{L} \rightarrow \mathcal{L}'  &= \bar{\psi}'  (i \slashed{\partial} - m) \psi' \\
                &= e^{-i\theta} \psi^{\dagger} \gamma^0(i \slashed{\partial} - m) e^{i\theta} \psi \\
                &= \bar{\psi} (i \slashed{\partial} - m)  \psi \\
                &= \mathcal{L}
\end{align*}	
On a utilisé le fait que $\theta \in \mathbb{R}$ est un paramètre constant et global, donc $e^{-i\theta}$ commute avec 
la dérivée $\partial_\mu$.
\end{proof}

\begin{solution}
Le courant de Noether associé à la transformation interne \eqref{eq:t_interne} est (à une phase $\pm 1$ près)
\begin{equation}
        \boxed{j^{\mu} = \bar{\psi} \gamma^{\mu} \psi}
\end{equation} 
\end{solution}
\begin{proof}
Soit la transformation interne infinitésimale
\begin{equation}
        \psi \rightarrow  (1 - i \theta) \psi
\end{equation} 
La transformation infinitésimale du spineur de Dirac est donc 
\begin{equation}
        \delta \psi = -i \psi
\end{equation} 
Il suit de la définition du courant de Noether que
\begin{align*}
        j^{\mu} &= \frac{\partial \mathcal{L}}{\partial (\partial_{\mu} \psi)} \delta \psi \\
        &= \left(  \frac{\partial }{\partial (\partial_{\mu}\psi)} \bar{\psi}(i \gamma^{\nu} \partial_\nu - m) \psi\right) \delta\psi\\
        &= \bar{\psi}\gamma^{\nu} \delta^{\mu}_\nu \psi \\
        &= \bar{\psi}\gamma^{\mu}\psi
\end{align*} 
\end{proof}


\subsection{}
\begin{solution}
Lorsque $m = 0$, les représentations chirales sont découplées et on obtient deux courants de Noether, une 
pour les spineurs de chiralité droite ($s=R$) et une pour les spineurs de chiralité gauche ($s=L$), soit 
une symétrie $U(1) \times U(1)$:
\begin{equation}
        \boxed{j^{\mu}_s = \bar{\psi}_s \sigma^{\mu}_s \psi_s}
\end{equation} 
où $\sigma_R^{\mu} = \sigma^{\mu}$ et $\sigma_L^{\mu} = \bar{\sigma}^{\mu}$, et $\sigma^{\mu} = (\bbone, \sigma^{i})$.\\ 
Dans la limite ultra relativiste $m \rightarrow  0$, le courant de Noether est plutôt
\begin{equation}
        \boxed{j^{\mu} = \bar{ \psi} \gamma^{\mu}\gamma^{5} \psi}\, .
\end{equation} 
Les représentations chirales gauche et droite sont corrélées et spin en sens inverse.
\end{solution}
\begin{proof}
        
On considère le Lagrangien de Dirac avec masse nulle, $m = 0$. 
On utilise la représentation de Weyl (ou représentation chirale) pour représenter les spineurs, soit
\begin{equation}
        \psi = \begin{pmatrix}
                \psi_L \\ \psi_R
        \end{pmatrix}\, ,
\end{equation} 
et
\begin{equation}
        \gamma^0 = \begin{pmatrix}
                \bbzero & \bbone\\
                \bbone& \bbzero
        \end{pmatrix}
        , 
        \hspace{1cm}
        \gamma^{i} = 
        \begin{pmatrix}
                \bbzero & \sigma^{i} \\
                \bar{\sigma}^{i}& \bbzero
        \end{pmatrix}\, .
\end{equation} 
où $\sigma^{i}$ sont les matrices de Pauli et $\bar{\sigma} = -\sigma^{i}$.
Dans ce cas, une symétrie du Lagrangien émerge du fait que les 
représentations chirales se découples dans l'équation de Dirac. En effet, on obtient les équations de Weyl
\begin{equation}
         i\gamma^{\mu}\partial_\mu \psi = 
         \begin{pmatrix}
                 i \sigma^{\mu} \partial_\mu \psi_R \\[1ex]
                 i \bar{\sigma}^{\mu}\partial_\mu \psi_L 
         \end{pmatrix}
         = 0\, .
\end{equation} 
De ce fait, le Lagrangien se découple en deux termes
\begin{align*}
        \mathcal{L} &=  i\bar{\psi} \gamma^{\mu}\partial_\mu \psi \\
        &= i
        \begin{pmatrix}
                \psi_L^{\dagger} \\
                \psi_R^{\dagger}
        \end{pmatrix}
        \gamma^0
         \begin{pmatrix}
                 \sigma^{\mu} \partial_\mu \psi_R \\[1ex]
                 \bar{\sigma}^{\mu}\partial_\mu \psi_L 
         \end{pmatrix} \\
         &= 
        \begin{pmatrix}
                \psi_R^{\dagger} \\
                \psi_L^{\dagger}
        \end{pmatrix}
         \begin{pmatrix}
                 \sigma^{\mu} \partial_\mu \psi_R \\[1ex]
                 \bar{\sigma}^{\mu}\partial_\mu \psi_L 
         \end{pmatrix} \, .
\end{align*}
D'où
\begin{equation}
        \mathcal{L} = i \psi_L^{\dagger} \bar{\sigma}^{\mu} \partial_\mu \psi_L + i \psi_r^{\dagger} \sigma^{\nu}\partial_\nu \psi_R \, .
\end{equation}
Ainsi, pour des particules de masses nulles, le Lagrangien a maintenant deux symétries internes, soit $U(1)\times U(1)$, donc deux 
courants de Noether
\begin{equation}
        \boxed{j^{\mu}_s = \bar{\psi}_s \sigma^{\mu}_s \psi_s}\, ,
\end{equation} 
où $\sigma^{\mu}_R = \sigma^{\mu}$ et $\sigma^{\mu}_L = \bar{\sigma}^{\mu}$. Ces courants sont valides strictement lorsque les représentations chirales 
sont indépendantes, et n'est donc pas valide dans une limite ultra relativiste. 

En effet, la symétrie du Lagrangien dans la limite où $m \rightarrow  0$, soit une limite ultra relativiste, est légèrement différente. 
Dans ce cas, on peut construire la symétrie axiale
\begin{equation}\label{eq:t_axiale}
        \psi(x) \rightarrow e^{i \theta \gamma^{5}} \psi(x) \, ,
\end{equation} 
où, dans la représentation de Weyl, 
\begin{equation}
        \gamma^{5} \equiv i\gamma^0\gamma^{1}\gamma^{2}\gamma^{3} = 
        i
        \begin{pmatrix}
                -\sigma^{1}\sigma^{2}\sigma^{2} & \bbzero \\
                \bbzero & \sigma^{1}\sigma^{2}\sigma^{3}
        \end{pmatrix}
        =
        \begin{pmatrix}
                -\bbone & \bbzero \\ 
                \bbzero & \bbone
        \end{pmatrix}\, .
\end{equation} 
On peut montrer que cette transformation est une symétrie du Lagrangien de Dirac en utilisant le fait que $(\gamma^{5})^{\dagger} = \gamma^{5}$, 
$(\gamma^{5})^{2} = \bbone$ et 
la formule d'Euler qui résulte de ce fait
\begin{equation}
        e^{i\theta \gamma^{5}} = \cos(\theta) + i \gamma^{5}\sin(\theta)
\end{equation} 
On utilise aussi la relation d'anti-commutation
\begin{equation}
        \{ \gamma^{5}, \gamma^{\mu} \} = 0\, ,
\end{equation} 
de sortes que $e^{i \theta \gamma^{5}} \gamma^{\mu} = \gamma^{\mu}e^{-i \theta \gamma^{5}}$ et
\begin{align*}
        \mathcal{L} \rightarrow  \mathcal{L}' &=\psi^{\dagger} e^{-i \theta \gamma^{5}}\gamma^0  (i\gamma^\mu \partial_\mu - m) e^{i\theta \gamma^{5}}\psi \\
                &=\psi^{\dagger} \gamma^0  e^{i \theta \gamma^{5}}(i\gamma^\mu \partial_\mu - m) e^{i\theta \gamma^{5}}\psi \\
                &= \bar{\psi} (i\gamma^\mu \partial_\mu - me^{2i\theta \gamma^{5}}) \psi \\
\end{align*}
Ainsi, dans la limite $m \rightarrow 0$, la transformation axiale \eqref{eq:t_axiale} est une symétrie du Lagrangien, qui correspond effectivement 
à deux symétries $U(1)$ où les paramètres des transformations correspondent à $\theta_R = -\theta_L$.  
On peut calculer le courant de Noether qui correspond à cette symétrie. On considère la transformation infinitésimale
\begin{equation}
        \psi(x) \rightarrow  (1 - i \theta \gamma^{5}) \psi(x)\, ,
\end{equation} 
d'où
\begin{equation}
        \delta \psi = -i \gamma^{5}
\end{equation} 
et
\begin{equation}
        \boxed{j^{\mu} = \bar{\psi} \gamma^{\mu}\gamma^{5}\psi}
\end{equation} 
\end{proof} 



\section{Invariance d’échelle}
On considère le champ de Dirac avec $d=3$ dimensions spatiales. On considère la transformation d'échelle
\begin{equation}\label{eq:t_echelle}
        \begin{split}
                x &\rightarrow  b x \\
                \psi(x) &\rightarrow b^{-\Delta} \psi(x)
        \end{split}
\end{equation} 
où $b \in \mathbb{R}_{>0}$ et $\Delta \in \mathbb{R}$.

\subsection{}
\textbf{Quelle sont les conditions pour que \eqref{eq:t_echelle} soit une symétrie de la théorie? Appelons l’action
de cette théorie $S_{\star}$. Quel est le courant de Noether associé?} \\
On ne considère que le cas avec $d = 3$ dimensions spatiales.
La transformation \eqref{eq:t_echelle} est une symétrie de la théorie si l'action 
\begin{equation}\label{eq:action}
        S_\star = \int d^{4}x\, \bar{\psi}(x)(i \slashed{\partial} - m)\psi(x)
\end{equation} 
est invariante sous l'application de la transformation. Puisque l'élément de volume devient
\begin{equation}
        d^{4}x \rightarrow b^{4}(d^{4}x)\, ,
\end{equation} 
et que l'opérateur $\slashed{\partial} \rightarrow b^{-1} \slashed{\partial}$ possède une facteur d'échelle $\Delta_{\slashed{\partial}} = 1$, alors 
la transformation d'échelle est une symétrie du Lagrangien si et seulement si
\begin{equation}
        \boxed{
        \begin{split}
              m = 0 \\
              \Delta_\psi = \frac{3}{2}
        \end{split}
}
\end{equation} 
On pose $b = 1 - \epsilon$, où $|\epsilon| \ll 1$. Ainsi, 
\begin{equation}
        \psi(x) \rightarrow  (1 + \epsilon\Delta ) \psi(x)
\end{equation} 
On prend le point de vue d'une transformation active du champ
\begin{equation}
        \delta \psi = \psi'(x) - \psi(x) \, .
\end{equation} 
On doit donc déterminer la quantité
\begin{align*}
        \psi'(x) &=  (1 + \epsilon\Delta ) \psi(x + \epsilon x) + \mathcal{O}(\epsilon^{2}) \\
\end{align*}
Or, pour simplifier ce terme, on doit déterminer la dérivée de Lie du champs spineur $\psi(x)$, ce qui est non-trivial.
\section{Théorie Yukawa classique}
Soit la densité lagrangienne pour un champ scalaire réel $\phi$ et un spineur de Dirac $\psi$
\begin{equation}
        \mathcal{L} = \frac{1}{2}\partial^{\mu}\phi \partial_\mu \phi - \frac{m^{2}}{2}\phi^{2} - \lambda \phi^{4} + \bar{\psi}(i \slashed{\partial} - M) \psi - g \phi\bar{\psi}\psi
\end{equation} 
où $m, M$ sont des masses et $\lambda,g$ sont des constantes de couplage.

\subsection{}
\textbf{Est-ce une théorie relativiste?}\\
\begin{solution}
       La théorie de Yukawa classique est une théorie relativiste. 
\end{solution}
\begin{proof}
Soit une transformation de Lorentz $\Lambda \in SO(3, 1)$. La théorie de Yukawa classique est relativiste si 
la densité lagrangienne est un scalaire de Lorentz. Par l'étude des densité lagrangienne de Klein-Gordon,  
on sait que les trois premiers termes sont des scalaires de Lorents. En effet, le champ scalaire $\phi$ est un invariant de Lorentz 
et, puisque $\Lambda\indices{^{\rho}_{\nu}}\Lambda\indices{_{\rho}^{\mu}} = \delta^{\mu}_\nu$, 
\begin{align*}
       \partial_\mu \phi \partial^{\mu}\phi &\rightarrow \Lambda\indices{_{\mu}^{\rho}}\partial_\rho \phi \Lambda\indices{^{\mu}_{\sigma}}\partial^{\sigma}\phi \\
                &\rightarrow  \delta^{\rho}_\sigma \partial_\rho \phi\partial^{\sigma}\phi \\
                &\rightarrow  \partial_\mu \phi \partial^{\mu}\phi  \\
\end{align*} 
On sait aussi que $\phi \overset{\Lambda}{\rightarrow}  \phi$.
La densité lagrangienne de Dirac est aussi un invariant de Lorentz, sachant que 
\begin{equation}
        \begin{split}
                \psi \rightarrow \Lambda_D \psi \\
                \bar{\psi} \rightarrow \bar{\psi}\Lambda_D^{-1}
        \end{split}
\end{equation} 
où $\Lambda_D \in SO(3, 1)$ est une transformation de Lorentz dans la représentation de Dirac $(\frac{1}{2}, 0) \oplus (0, \frac{1}{2})$, 
et que les matrices $\gamma^{\mu}$ se transforme comme
\begin{equation}
        \Lambda_D^{-1} \gamma^{\mu} \Lambda_D = \Lambda\indices{^{\mu}_{\nu}}\gamma^{\nu}\, ,
\end{equation} 
on obtient 
\begin{align*}
       \mathcal{L}_D \rightarrow \mathcal{L}'_D &= \bar{\psi}\Lambda_D^{-1}(i\gamma^{\mu} \Lambda\indices{_{\mu}^{\nu}}\partial_\nu - M)\Lambda_D \psi \\
                 &= \bar{\psi}(i \Lambda_D^{-1}\gamma^{\mu} \Lambda_D \Lambda\indices{_{\mu}^{\nu}}\partial_\nu - M)\psi \\
                 &= \bar{\psi}(i \Lambda\indices{^{\mu}_{\rho}}\gamma^{\rho}\Lambda\indices{_{\mu}^{\nu}}\partial_\nu - M)\psi \\
                 &= \bar{\psi}(i \delta^{\nu}_\rho\gamma^{\rho}\partial_\nu - M)\psi \\
                 &= \mathcal{L}_D
\end{align*}
Finalement,
\begin{equation}
        \bar{\psi}\psi \rightarrow \bar{\psi}\Lambda_D^{-1}\Lambda_D \psi = \bar{\psi} \psi
\end{equation} 
Donc
\begin{equation}
        \mathcal{L} \overset{\Lambda}{\rightarrow} \mathcal{L} 
\end{equation} 
\end{proof}

\subsection{}
\textbf{On cherche les équations du mouvement pour la théorie de Yukawa classique}.\\
\begin{solution}
       L'équation de mouvement pour $\phi$ est
       \begin{equation}
               \boxed{\partial_\mu\partial^{\mu} \phi = -m^{2} \phi - 4\lambda \phi^{3} - g \bar{ \psi} \psi }
       \end{equation} 
\end{solution}
\begin{proof}
La preuve suit par calcul direct. On a
\begin{equation}
        \frac{\partial L}{\partial \phi} = -m^{2}\phi - 4\lambda \phi^{3} - g \bar{\psi}\psi\, ,
\end{equation} 
et
\begin{equation}
        \frac{\partial L}{\partial (\partial_\mu \phi)} = \partial^{\mu}\phi\, .
\end{equation} 
De sortes que
\begin{equation}
        \frac{\partial \mathcal{L}}{\partial \phi} - \partial_\mu \left( \frac{\partial \mathcal{L}}{\partial (\partial_\mu \phi)} \right)
        = -m^{2}\phi - 4\lambda \phi^{3} -g \bar{\psi} \psi - \partial_\mu \partial^{\mu} \phi
        = 0
\end{equation} 
\end{proof}
\begin{solution}
       L'équation de mouvement pour $\psi$ est
       \begin{equation}
               \boxed{i \slashed{\partial} \bar{\psi} = (g \phi - M) \bar{\psi}}
       \end{equation} 
\end{solution}
\begin{proof}
La preuve suit par calcul direct. On a
\begin{equation}
        \frac{\partial \mathcal{L}}{\partial \psi} = g\phi \bar{\psi} - M \bar{\psi}\, ,
\end{equation} 
et
\begin{equation}
        \frac{\partial \mathcal{L}}{\partial (\partial_\mu \psi)} = i \bar{\psi} \gamma^{\mu}\, .
\end{equation} 
D'où
\begin{equation}
        \frac{\partial \mathcal{L}}{\partial \psi} - \partial_\mu \left( \frac{\partial \mathcal{L}}{\partial (\partial_\mu \psi)} \right)
        =
        g\phi \bar{\psi} - M \bar{\psi}
        -
         i\partial_\mu \gamma^{\mu}\bar{\psi} 
        = 0
\end{equation} 
\end{proof}

\begin{solution}
       L'équation de mouvement pour $\bar{\psi}$ est 
       \begin{equation}
               \boxed{(i\slashed{\partial} - M - g\phi) \psi = 0 }
       \end{equation} 
\end{solution}
\begin{proof}
On a que
\begin{equation}
        \frac{\partial \mathcal{L}}{\partial \bar{\psi}} = (i\slashed{\partial} - M) \psi - g \phi \psi\, ,
\end{equation} 
et
\begin{equation}
        \frac{\partial \mathcal{L}}{\partial (\partial_\mu \bar{\psi})} = 0\, ,
\end{equation} 
d'où
\begin{equation}
        \frac{\partial \mathcal{L}}{\partial \bar{\psi}} = (i \slashed{\partial} - M) \psi - g \phi \psi = 0\, .
\end{equation} 
\end{proof}

\subsection{}
\textbf{On cherche les conditions sous lesquelles la théorie de Yukawa classique possède une invariance d'échelle.}
\begin{solution}
Les conditions pour que la théorie de Yukawa classique soit invariante d'échelle sont
\begin{equation}
        \boxed{
        \begin{split}
        \Delta_{\phi} &= 1 \\
        \Delta_{\psi} &=  \frac{3}{2} \\
        M &= m = 0 \\
        \end{split}
}
\end{equation} 
\begin{proof}
Chaque terme du Lagrangien doit posséder un facteur d'échelle total de $4$ pour que la théorie soit invariant d'échelle. 
Ainsi,

        %\mathcal{L} = \frac{1}{2}\partial^{\mu}\phi \partial_\mu \phi - \frac{m^{2}}{2}\phi^{2} - \lambda \phi^{4} + \bar{\psi}(i \slashed{\partial} - M) \psi - g \phi\bar{\psi}\psi
\begin{equation}
        \frac{1}{2}\partial^{\mu}\phi \partial_\mu \phi \rightarrow b^{-2-2\Delta_{\phi}}  \frac{1}{2}\partial^{\mu}\phi \partial_\mu \phi \implies \Delta_{\phi} = 1 
\end{equation} 
et il suit que $m = 0$, autrement le terme $\propto m^{2} \phi^2$ n'a pas le bon facteur d'échelle. Comme 
le terme $\lambda \phi^{4}$ a le bon facteur d'échelle, alors les conditions $\Delta_{\phi} = 1$ et $m = 0$ sont suffisantes 
et nécessaires pour les 3 premiers termes. Comme dérivé au numéro précédent, le Lagrangien de Dirac impose que
\begin{equation}
        \Delta_{\psi} = \frac{3}{2}\hspace{1cm} M = 0
\end{equation} 
On remarque finalement que le terme de couplage
\begin{equation}
        g \phi \bar{\psi} \psi \rightarrow b^{-4} g\phi \bar{\psi}\psi
\end{equation} 
avec les conditions trouvées. 
\end{proof}
        
\end{solution}

\subsection{}
\textbf{On prend $M = 0$ et $\phi(x) = \phi_0$, une constante de l'espace-temps. On pose aussi $|g| \ll 1$. On cherche une solution pour $\phi_0$.}
%\begin{solution}
%\begin{equation}
        %\boxed{\phi_0 = }
%\end{equation} 
%\begin{proof}
        
%\end{proof}
        
%\end{solution}

\section{Opérateurs qui anti-commutent}
On considère l'Hamiltonien suivant
\begin{equation}\label{eq:hamiltonien5}
       \hat{H} = \sum_{a, b}^{N} \hat{t}_{ab}\hat{c}^{\dagger}_a \hat{c}_b \, ,
\end{equation} 
avec les relations d'anti-commutations suivantes
\begin{equation}
        \{\hat{c}_a, \hat{c}^{\dagger}_b \} = \delta_{ab} \hspace{1cm} \{ \hat{c}_a, \hat{c}_b\} = 0\, .
\end{equation} 

\subsection{}
\textbf{On cherche la condition nécessaire telle que $\hat{H}$ soit un opérateur hermitien.}
\begin{solution}
        Les éléments de la matrice $t_{ab}$ sont réels.
\end{solution}
\begin{proof}
Supposons que $\hat{t}_{ab} \in \mathbb{R}$. Il suit que $\hat{t}^{\dagger}_{ab} = \hat{t}_{ba}$ et
\begin{align*}
       \hat{H}^{\dagger} &=  \sum_{a,b}(\hat{t}_{ab}\hat{c}^{\dagger}_a \hat{c}_b)^{\dagger} \\
                         &= \sum_{a,b} \hat{t}_{ba} \hat{c}^{\dagger}_b \hat{c}_a \hspace{1cm} \{\hat{t}_{ab} \in \mathbb{R}\}\\
                         &= \hat{H}
\end{align*} 
où la dernière ligne suit du fait que les indices $a \leftrightarrow b$ sont symétriques.
La condition $\hat{t}_{ab} \in \mathbb{R}$ suit du fait qu'on requiert $[\hat{t}_{ab},\, \hat{c}^{\dagger}_b \hat{c}_a] = 0$ et $\hat{t}_{ab}^{*} = \hat{t}_{ab}$.
\end{proof}

\subsection{}
On suppose $N=1$. On s'intéresse aux états stationnaires de l'Hamiltonien \eqref{eq:hamiltonien5}.


\subsection{}

\section{Fermions Majorana (Peskins \& Schroeder 3.4)}

\subsection{}
On veut montrer que
\end{document}

