\documentclass{article}
\usepackage[a4paper, margin=2cm]{geometry}

\usepackage{amsmath}
\usepackage{amssymb}
\usepackage{mathtools}
\usepackage{amstext}
\usepackage{amsthm}
\usepackage{fancyhdr}
\usepackage[utf8]{inputenc} % allow utf-8 input
\usepackage[T1]{fontenc}    % use 8-bit T1 fonts


\usepackage{graphicx}
\usepackage{float}
\usepackage{caption}
\usepackage{subcaption}
\usepackage{booktabs}
\usepackage{physics, tensor}
\usepackage{slashed, cancel}

\graphicspath{{figures/}}

\pagestyle{fancy}
\rhead{Alexandre Adam\\ 20090755}
\lhead{William Witczak-Krempa \\ PHY 6812: Théorie des champs 1}
\chead{Devoir 4}
\rfoot{6 décembre 2022}
\cfoot{\thepage}

\newcommand{\angstrom}{\textup{\AA}}
\numberwithin{equation}{section}
\renewcommand\thesubsection{\alph{subsection})}
\renewcommand\thesubsubsection{\Roman{subsubsection}}
\newcommand{\s}{\hspace{0.1cm}}
\DeclareRobustCommand{\bbzero}{\text{\usefont{U}{bbold}{m}{n}0}}
\DeclareRobustCommand{\bbone}{\text{\usefont{U}{bbold}{m}{n}1}}


\theoremstyle{solution}
\newtheorem{solution}{Réponse}[section]

\renewcommand*{\proofname}{Solution}


\begin{document}


\section{Algèbre du moment cinétique total}
On cherche les relations de commutations pour l'opérateur de moment cinétique
\begin{equation}
        \hat{\mathbf{J}} = \int_{\mathbf{x}} \hat{\psi}^{\dagger}(\mathbf{x})(\mathbf{x} \times(-i \grad) + \mathbf{L}) \hat{\psi}(\mathbf{x})
\end{equation} 
Soit,
\begin{align*}
        [\hat{\mathbf{J}}_i, \hat{\mathbf{J}}_j] 
        &= \int_{\mathbf{x}, \mathbf{y}}
        [ \hat{\psi}^{\dagger}(\mathbf{x})(\mathbf{x} \times(-i \grad) + \mathbf{L})_i \hat{\psi}(\mathbf{x}),\,
        \hat{\psi}^{\dagger}(\mathbf{y})(\mathbf{y} \times(-i \grad) + \mathbf{L})_j \hat{\psi}(\mathbf{y})
        ] \\
        &= \int_{\mathbf{x}, \mathbf{y}}
        [ \hat{\psi}^{\dagger}(\mathbf{x})\mathbf{L}_i \hat{\psi}(\mathbf{x}),\,
        \hat{\psi}^{\dagger}(\mathbf{y})(\mathbf{y} \times(-i \grad))_j \hat{\psi}(\mathbf{y})
        ] 
        \\
        &\hspace{.6cm}+
        [ \hat{\psi}^{\dagger}(\mathbf{x})(\mathbf{x} \times(-i \grad))_i \hat{\psi}(\mathbf{x}),\,
        \hat{\psi}^{\dagger}(\mathbf{y})(\mathbf{y} \times(-i \grad))_j \hat{\psi}(\mathbf{y})
        ] 
        \\
        &\hspace{.6cm}+
        [ \hat{\psi}^{\dagger}(\mathbf{x})(\mathbf{x} \times(-i \grad))_i \hat{\psi}(\mathbf{x}),\,
        \hat{\psi}^{\dagger}(\mathbf{y})\mathbf{L}_j \hat{\psi}(\mathbf{y})
        ] 
        \\
        &\hspace{.6cm}+
        [ \hat{\psi}^{\dagger}(\mathbf{x})\mathbf{L}_i \hat{\psi}(\mathbf{x}),\,
        \hat{\psi}^{\dagger}(\mathbf{y})\mathbf{L}_j \hat{\psi}(\mathbf{y})
        ] 
\end{align*}
En appliquant la règle $[A, BC] = B[A, C] + [A, B]C$ à répétition, on peut utiliser la relation de commutation canonique
\begin{equation}
        [\psi(\mathbf{x}), \psi^{\dagger}(\mathbf{y})] = \delta^{(3)}(\mathbf{x} - \mathbf{y}) \bbone
\end{equation} 
pour simplifier l'intégrale. De plus,
\begin{equation}
        [\mathbf{x} \times (-i \grad), \mathbf{L}_i] = 0
\end{equation} 
puisque $\mathbf{L}$ ne dépend pas de la position $\mathbf{x}$. 
On obtient alors
\begin{align*}
        [\hat{\mathbf{J}}_{i}, \hat{\mathbf{J}}_j] 
        &= 
        \int_{\mathbf{x}} \hat{\psi}^{\dagger}(\mathbf{x})\big([(\mathbf{x} \times (-i \grad))_{i},(\mathbf{x} \times (-i \grad)_{j}] + [\mathbf{L}_{i}, \mathbf{L}_j]\big)\hat{\psi}(\mathbf{x}) \\ 
        &= 
        \int_{\mathbf{x}} \hat{\psi}^{\dagger}(\mathbf{x})\big([(\mathbf{x} \times (-i \grad))_{i},(\mathbf{x} \times (-i \grad)_{k}] + \epsilon_{ijk} \mathbf{L}_{k}\big)\hat{\psi}(\mathbf{x})
\end{align*}
On calcule maintenant le dernier commutateur
\begin{align*}
        [(\mathbf{x} \times (-i \grad))_{i},(\mathbf{x} \times (-i \grad))_{k}] 
        &=  i^2( \epsilon_{i \ell m}\mathbf{x}_{\ell} \partial_{m} \epsilon_{jnp}\mathbf{x}_n \partial_{p} - \epsilon_{jnp} \mathbf{x}_{n}\partial_{p} \epsilon_{i\ell m} \mathbf{x}_{\ell} \partial_{m}) \\
        &=  - \epsilon_{i \ell m}\epsilon_{jnp}\mathbf{x}_{\ell} \delta_{mn} \partial_{p} + \epsilon_{jnp}\epsilon_{i\ell m} \mathbf{x}_{n} \delta_{p \ell} \partial_{m} \\
        &=   - \epsilon_{i \ell m}\epsilon_{jmp}\mathbf{x}_{\ell} \partial_{p} + \epsilon_{jn\ell}\epsilon_{i\ell m} \mathbf{x}_{n} \partial_{m} 
\end{align*}
On sait que
\begin{equation}
        \epsilon_{abc}\epsilon_{efg} = \delta_{bf}\delta_{cg} - \delta_{bg}\delta_{cf}
\end{equation} 
donc, en permutant les indices $i$ et $j$ à la seconde position du tenseur de Levi-Civita pour utiliser l'identité, et 
en permutant le premier tenseur une seconde fois pour conserver la somme sur l'indice de $\mathbf{x}$, on obtient
\begin{align*}
        [(\mathbf{x} \times (-i \grad))_{i},(\mathbf{x} \times (-i \grad))_{k}] 
        &=  \epsilon_{m i \ell}\epsilon_{m j p}\mathbf{x}_{\ell} \partial_{p} - \epsilon_{\ell j n}\epsilon_{\ell im} \mathbf{x}_{n} \partial_{m}  \\
        &=  (\delta_{i j} \delta_{\ell p} - \delta_{i p} \delta_{\ell j})\mathbf{x}_{\ell} \partial_{p} - (\delta_{ij} \delta_{nm} - \delta_{jm}\delta_{n i})\mathbf{x}_{n} \partial_{m}  \\
        &=   -\delta_{i p} \delta_{\ell j}\mathbf{x}_{\ell} \partial_{p} + \delta_{jm}\delta_{n i}\mathbf{x}_{n} \partial_{m}  \\
        &=   -\mathbf{x}_{j} \partial_{i} + \mathbf{x}_{i} \partial_{j}  \\
        &= -\epsilon_{ijk}\mathbf{x}_i\partial_{j} \\
        &= -(\mathbf{x} \times \grad)_{k}
\end{align*}
On obtient finalement (si on peut retrouver le facteur $i$ manquant),
\begin{equation}
        \boxed{[\hat{\mathbf{J}}_{i}, \hat{\mathbf{J}}_{j}] = \epsilon_{ijk}\hat{\mathbf{J}}_{k}}
\end{equation} 

\section{Représentations du groupe de Lorentz $SO(2, 1)$}

\subsection{Représentations irréductibles de $SO(2, 1)$}
Le groupe $SO(2, 1)$ est le groupe de Lorentz dans l'espace de Minkowsky $(2 + 1)$-dimensionel avec la signature de la métrique $\eta_{\mu\nu} = \mathrm{diag}(+, -, -)$. 
Ce groupe est définit comme l'ensemble des actions $g \in SO(2, 1)$ avec déterminant $ \mathrm{det}(g) = 1$
qui préservent le produit intérieur pour deux vecteurs dans l'espace de Minkowsky $x, y \in \mathbb{R}^{2+1}$
\begin{equation}
       \langle x | y \rangle = x_0y_0 - x_1y_1 - x_2y_2\, .
\end{equation} 
\textbf{Le représentation définissante} du groupe $SO(2, 1)$ agit sur l'espace des vecteurs $\mathbb{R}^{2+1}$ et est définit 
comme l'ensemble des matrices réelles $3\times 3$ de déterminant $1$ qui laisse la métrique $\eta$ invariante
\begin{equation}
        SO(2, 1) = \left\{ \Lambda\indices{^{\mu}_{\nu}} \in \mathbb{M}^{2+1}(\mathbb{R}) \mid \Lambda^{T}\eta\Lambda = \eta,\, \mathrm{det}(\Lambda) = 1 \right\}
\end{equation} 
Si on ne considère que le sous-groupe $SO^{+}(2, 1)$ connecté de façon continu à l'identité, alors on peut construire 
3 générateurs pour l'algèbre de Lie. On considère la transformation infinitésimale 
\begin{equation}
        \Lambda\indices{^{\alpha}_{\beta}} = \delta^{\alpha}_{\beta} - \frac{i}{2}\omega_{\mu \nu}(\mathcal{J}^{\mu\nu})\indices{^{\alpha}_{\beta}}
\end{equation} 
où $\mathcal{J}^{\mu\nu}$ est un générateur de l'algèbre de Lie $\mathfrak{so}(2, 1)$. Puisque, $\Lambda$ doit 
laisser la métrique invariante, alors $\mathcal{J}^{\mu\nu}$ doit être antisymétrique ($\mathcal{J}^{\mu \nu} = -\mathcal{J}^{\nu \mu}$), 
ce qui ne laisse que $3$ éléments indépendants: les générateurs de boost, $\mathcal{J}^{01}$ et $\mathcal{J}^{02}$, 
et le générateur de rotation $\mathcal{J}^{12}$. On peut construire ces générateurs à partir 
de la métrique
\begin{equation}
        (\mathcal{J}^{\mu \nu})^{\alpha \beta} = i (\eta^{\mu \alpha} \eta^{\nu \beta} - \eta^{\nu \alpha}\eta^{\mu \beta})\, .
\end{equation} 
On obtient la forme bilinéaire en appliquant la métrique sur le tenseur $(\mathcal{J}^{\mu\nu})^{\alpha \beta}$
\begin{equation}
        (\mathcal{J}^{\mu \nu})\indices{^{\alpha}_{\beta}} = i( \eta^{\mu \alpha} \delta\indices{^{\nu}_{\beta}} - \eta^{\nu \alpha} \delta\indices{^{\mu}_{\beta}})\, ,
\end{equation} 
d'où on obtient
\begin{equation}\label{eq:generators}
        \mathcal{J}^{01} = 
        \begin{pmatrix}
                0 & i & 0 \\
                i & 0 & 0 \\
                0 & 0 & 0
        \end{pmatrix}
        \hspace{1cm}
        \mathcal{J}^{02} = 
        \begin{pmatrix}
                0 & 0 & i \\
                0 & 0 & 0 \\
                i & 0 & 0
        \end{pmatrix}
        \hspace{1cm}
        \mathcal{J}^{12} = 
        \begin{pmatrix}
                0 & 0 & 0 \\
                0 & 0 & -i \\
                0 & i & 0
        \end{pmatrix}
\end{equation} 
L'algèbre de Lie $\mathfrak{so}(2, 1)$ est donc définit par les commutateurs 
\begin{align}
        [\mathcal{J}^{01}, \mathcal{J}^{12}] &= i \mathcal{J}^{02} \\
        [\mathcal{J}^{02}, \mathcal{J}^{12}] &= -i \mathcal{J}^{01}  \\
        [\mathcal{J}^{01}, \mathcal{J}^{02}] &= i \mathcal{J}^{12}
\end{align}
ce qui peut être vérifié explicitement avec les générateurs exprimées dans la représentation définissante \eqref{eq:generators} 
\begingroup
\allowdisplaybreaks
\begin{align*}
        [\mathcal{J}^{01}, \mathcal{J}^{12}] &= 
        \begin{pmatrix}
                0 & i & 0 \\
                i & 0 & 0 \\
                0 & 0 & 0
        \end{pmatrix}
        \begin{pmatrix}
                0 & 0 & 0 \\
                0 & 0 & -i \\
                0 & i & 0
        \end{pmatrix}
        -
        \begin{pmatrix}
                0 & 0 & 0 \\
                0 & 0 & -i \\
                0 & i & 0
        \end{pmatrix}
        \begin{pmatrix}
                0 & i & 0 \\
                i & 0 & 0 \\
                0 & 0 & 0
        \end{pmatrix}
        \\
        &= 
        \begin{pmatrix}
                0 & 0 & 1 \\
                0 & 0 & 0 \\
                0 & 0 & 0
        \end{pmatrix}
        -
        \begin{pmatrix}
                0 & 0 & 0 \\
                0 & 0 & 0 \\
                -1 & 0 & 0
        \end{pmatrix}
        \\
        &= 
        i \mathcal{J}^{02} \\
        [\mathcal{J}^{02}, \mathcal{J}^{12}] &= 
        \begin{pmatrix}
                0 & 0 & i \\
                0 & 0 & 0 \\
                i & 0 & 0
        \end{pmatrix}
        \begin{pmatrix}
                0 & 0 & 0 \\
                0 & 0 & -i \\
                0 & i & 0
        \end{pmatrix}
        -
        \begin{pmatrix}
                0 & 0 & 0 \\
                0 & 0 & -i \\
                0 & i & 0
        \end{pmatrix}
        \begin{pmatrix}
                0 & 0 & i \\
                0 & 0 & 0 \\
                i & 0 & 0
        \end{pmatrix}
        \\
        &= 
        \begin{pmatrix}
                0 & -1 & 0 \\
                0 & 0 & 0 \\
                0 & 0 & 0
        \end{pmatrix}
        -
        \begin{pmatrix}
                0 & 0 & 0 \\
                1 & 0 & 0 \\
                0 & 0 & 0
        \end{pmatrix}
        \\
        &= 
        -i \mathcal{J}^{01}\\
        [\mathcal{J}^{01}, \mathcal{J}^{02}] &= 
        \begin{pmatrix}
                0 & i & 0 \\
                i & 0 & 0 \\
                0 & 0 & 0
        \end{pmatrix}
        \begin{pmatrix}
                0 & 0 & i \\
                0 & 0 & 0 \\
                i & 0 & 0
        \end{pmatrix}
        -
        \begin{pmatrix}
                0 & 0 & i \\
                0 & 0 & 0 \\
                i & 0 & 0
        \end{pmatrix}
        \begin{pmatrix}
                0 & i & 0 \\
                i & 0 & 0 \\
                0 & 0 & 0
        \end{pmatrix}
        \\
        &= 
        \begin{pmatrix}
                0 & 0 & 0 \\
                0 & 0 & -1 \\
                0 & 0 & 0
        \end{pmatrix}
        -
        \begin{pmatrix}
                0 & 0 & 0 \\
                0 & 0 & 0 \\
                0 & -1 & 0
        \end{pmatrix}
        \\
        &= 
        i \mathcal{J}^{12}
\end{align*}
\endgroup
On procède ensuite à la complexification de l'algèbre de Lie. On introduit les générateurs de boosts $K_{\pm}$ et le 
générateur de rotation $L$
\begin{align*}
        K_{\pm} &= \frac{1}{\sqrt{2}}(\mathcal{J}^{01} \mp i \mathcal{J}^{02}) \\
        L &= \sqrt{2}\mathcal{J}^{12}
\end{align*}
Les commutateurs dans ce nouvel espace sont
\begin{align}
        [K_{-}, K_{+}] &= L \\
        [K_{\pm}, L] &= \mp K_{\pm} \, ,
\end{align}
ce qui peut être déduit des relation de commutations pour les matrices $\mathcal{J}^{\mu \nu}$
\begin{align*}
        [K_{\pm}, L] &= [\mathcal{J}^{01} \mp i \mathcal{J}^{02}, \mathcal{J}^{12}] \\
                     &= (i\mathcal{J}^{02} \mp i(-i)\mathcal{J}^{01}) \\
                     &= \mp K_{\pm} \\
        [K_{-}, K_{+}] &= \frac{1}{2}[\mathcal{J}^{01} + i \mathcal{J}^{02}, \mathcal{J}^{01} - i \mathcal{J}^{02}] \\
                &= \frac{1}{2}(-i^2\mathcal{J}^{12} - i^2\mathcal{J}^{12})\\ 
                &= L
\end{align*}
Ces relations de commutations rendent claire l'isomorphisme
\begin{equation}
        \mathfrak{so}(2, 1) \cong \mathfrak{sl}(2, \mathbb{R}) 
\end{equation} 
Le groupe $SL(2, \mathbb{R})$ est définit comme
\begin{equation}
        SL(2, \mathbb{R}) = \left\{ M \in \mathbb{M}^{2}(\mathbb{R}) \mid \mathrm{det}(M) = 1 \right\}
\end{equation} 
et l'algèbre de Lie $\mathfrak{sl}(2, \mathbb{R})$ possède les générateurs
\begin{equation}
        e = 
        \begin{pmatrix}
                0 & 1 \\
                0 & 0
        \end{pmatrix}
        \hspace{1cm}
        f = 
        \begin{pmatrix}
                0 & 0 \\
                1 & 0
        \end{pmatrix}
        \hspace{1cm}
        h = 
        \frac{1}{2}
        \begin{pmatrix}
                1 & 0 \\
                0 & -1
        \end{pmatrix}
\end{equation} 
avec les relations de commutations
\begin{align}
        [e, f] &= h \\
        [f, h] &= f \\
        [e, h] &= -e
\end{align}
Ces relations soit identiques aux relations de l'algèbre de Lie $\mathfrak{so}(2, 1)$ complexifiée.

On peut donc construire une classification des représentations finies de $\mathfrak{so}(2, 1)$ à partir de la classification 
connue des représentations finies de $\mathfrak{sl}(2, \mathbb{R})$, qui est très similaires à $\mathfrak{su}(2)$. On doit simplement 
se rappeler qu'on obtiendra seulement les actions du groupes $SO^{+}(2, 1)$ de cette façon, et non le groupe complet 
$SO(2, 1)$ qui n'est pas simplement connecté.

%Une représentation du groupe $\mathfrak{sl}(2, \mathbb{R})$ dans dans un espace vectoriel $V$ de dimension $\dim_{\mathbb{C}}(V) = m = 2\ell + 1$, où 
%$m \in \mathbb{N}$ et $\ell$ est un nombre demi-entier positif. Les représentations irréductibles de du groupe sont classifiées 
%à l'aide des valeurs propres de l'opérateur $L$. 
Posons $v \in V$ un élément de l'espace vectoriel de la représentation et un vecteur propre de $L$ avec comme valeur propre $\lambda \in \mathbb{C}$.
On remarque que 
\begin{align*}
        L (K_{+} v) &=  (K_+ L  + [L, K_{+}])v \\
        &= (\lambda + 2) K_+ v \\
        L(K_{-} v) &=  (K_{-} L + [L, K_{-}])v \\
        &= (\lambda - 2)K_{-}v
\end{align*} 
Ainsi, les opérateurs $K_{\pm}$ agissent comme des opérateurs d'échelles. 

On peut montrer par induction qu'il existe un vecteur propre $v_{j} = K_{+}^{j}v$ tels que
\begin{equation}
        L v_j = (\lambda + 2 j) v_{j} 
\end{equation} 
Pour une représentation irréductible de dimension finie, il ne peut exister qu'un nombre fini de vecteur propres. 
Supposons dans ce cas que $\lambda$ est la dernière valeur propre de $L$.
Dans ce cas, on doit avoir que $K_{+} v_m = 0$ pour un $m \in \mathbb{N}$ quelconque et $v_{m} \not= 0$, 
autrement $v_{m+1}$ serait un vecteur propre de $L$ avec une valeur propre $\lambda + 2$. 
Dans ca cas, $\lambda = m$, et on obtient une représentation de dimension de finie qui est construite par 
l'espace vectoriel de dimension $\mathrm{dim}_{\mathbb{C}}(V) = m+1$ des vecteurs propres de $L$
\begin{equation}
        V = \mathrm{span}(\{v_{0}, v_1, \dots, v_m\})
\end{equation} 
En terme de l'indice spinoriel $\ell = 0, \frac{1}{2}, 1, \frac{3}{2}, \dots$, on a
\begin{equation}
       m = 2 \ell 
\end{equation} 
et $\mathrm{dim}_{\mathbb{C}}(V) = 2\ell + 1$.

\subsection{Première représentation non-triviale de $SO(2, 1)$}
Puisque $\ell=0$ est une représentation triviale, on considère plutôt le premier cas non-trivial $\ell=\frac{1}{2}$. 
Posons $\psi \in V$ un élément de l'espace vectoriel de dimension $\mathrm{dim}_{\mathbb{C}}(V) = 2$. Dans cette représentation, les générateurs 
de l'algèbre de Lie $\mathfrak{so}(2, 1)$ sont les matrices $e$, $f$ et $h$
\begin{equation}
        K_{-} = 
        \begin{pmatrix}
        0 & 1 \\
        0 & 0
        \end{pmatrix}
        \hspace{1cm}
        K_{+} = 
        \begin{pmatrix}
        0 & 0 \\
        1 & 0
        \end{pmatrix}
        \hspace{1cm}
        L = 
        \begin{pmatrix}
        1 & 0 \\
        0 & -1
        \end{pmatrix}
        \hspace{1cm}
\end{equation} 
La transformation de Lorentz d'un spineur $\psi \in V$ devient
\begin{equation}
        \psi(x) \rightarrow \exp(-\frac{i}{2} \beta_1 K_{-} - \frac{i}{2}\beta_2 K_{+} - \frac{i}{2}\theta L) \psi(x)
\end{equation} 
%On remarque que les générateurs obtenus satisfont les relations suivantes
%\begin{align*}
        %K_{-} &= \frac{\sigma_{1} + i\sigma_{2}}{2} \\
        %K_{+} &= \frac{\sigma_{1} - i\sigma_{2}}{2} \\
        %L &= \frac{\sigma_{3}}{2}
%\end{align*}
%Finalement, on obtient comment ces vecteurs se transforment sous l'effet d'une rotation $R(\theta)$ ou d'un boost $\Lambda(v)$. On sait que 
%la rotation correspond à la map exponentiel du générateur $\mathcal{J}^{12}$, aussi associée à l'élément $L$ de la représentation $\ell=\frac{1}{2}$.
%Donc, en appliaquant la map exponentielle, on a
%\begin{align*}
        %x &\rightarrow R(\theta) x \\
        %\psi(x) & \rightarrow  \exp(i \theta L) \psi(x)
%\end{align*}
%pour le boost, on a
%\begin{align*}
        %x &\rightarrow  \Lambda(v) x \\
        %\psi(x) &\rightarrow \exp(i v )
%\end{align*}

\subsection{}

\end{document}

