\documentclass{article}
\usepackage[a4paper, margin=2cm]{geometry}

\usepackage[utf8]{inputenc} % allow utf-8 input
\usepackage[T1]{fontenc}    % use 8-bit T1 fonts
\usepackage{hyperref}       % hyperlinks
\hypersetup{
    colorlinks,
    linkcolor={red!50!black},
    citecolor={blue!50!black},
    urlcolor={blue!80!blue}
}
\usepackage{xcolor}
\usepackage{nicefrac}       % compact symbols for 1/2, etc.
\usepackage{bm, bbm}

\usepackage{amsmath}
\usepackage{amssymb}
\usepackage{mathtools}
\usepackage{amstext}
\usepackage{amsthm}
\usepackage{physics}
\usepackage{tensor}

\usepackage{fancyhdr}
\pagestyle{fancy}
\rhead{Alexandre Adam \\ 20090755}
\lhead{Théorie des champs 1 (PHY 6812) \\ William Witczak-Krempa}
\chead{Devoir 1}
\rfoot{\today}
\cfoot{\thepage}

\numberwithin{equation}{section}
\renewcommand\thesubsection{\alph{subsection})}
\renewcommand\thesubsubsection{\Roman{subsubsection}}
\DeclareRobustCommand{\bbone}{\text{\usefont{U}{bbold}{m}{n}1}}

%\newtheorem*{remark}{Remark}
\theoremstyle{solution}
\newtheorem{solution}{Réponse}[section]

\renewcommand*{\proofname}{Solution}

\begin{document}
\section{Un peu de relativité}

\subsection{}
\textbf{Classiquement, un photon se propageant dans le vide peut-il se désintégrer en une paire
électron-positron ? 
%Expliquez votre résultat mathématique de manière heuristique. 
Que ce passera-t-il en théorie quantique des champs (QFT) ?}
\subsection{}
\textbf{Quelle est la condition pour qu’un tenseur soit invariant sous transformations de Lorentz ?}
\vspace{2ex}

Soit un tenseur de rang $(m,n)$, $m,n \in \mathbb{N}$, avec les éléments $T\indices{^{\mu_1\dots \mu_m}_{\nu_1\dots\nu_n}}$, et la métrique de Minkowsky $\eta$. Soit une transformation de 
Lorentz $\Lambda$ définit tel que
\begin{equation}
        \eta_{\mu \nu} = \Lambda\indices{^\alpha_\mu} \Lambda\indices{^\beta_\nu} \eta_{\alpha\beta}
\end{equation} 
Pour que $T$ soit un invariant de Lorentz, il doit satisfaire
\begin{equation}
        \boxed{ T\indices{^{\mu_1\dots\mu_m}_{\nu_1\dots\nu_n}} = \Lambda\indices{^{\mu_1}_{\alpha_1}} \dots \Lambda\indices{^{\mu_m}_{\alpha_m}} 
        \Lambda\indices{^{\beta_1}_{\nu_1}} \dots \Lambda\indices{^{\beta_n}_{\nu_n}} T\indices{^{\alpha_1\dots \alpha_m}_{\beta_1 \dots \beta_n}}} \, ,
\end{equation}
similairement à la métrique de Minkowsky.

\subsection{}
\textbf{Les tenseurs $\delta_{\mu}^{\nu}$ et  $\epsilon^{\mu\nu\rho\sigma}$ (Levi-Civita) sont-ils invariants de Lorentz ?}
\vspace{2ex}
\begin{solution}
        Le delta de Kronecker $\delta_\mu^{\nu}$ est un invariant de Lorentz.
\end{solution}
\begin{proof}
On remarque que $\delta^{\nu}_{\mu} = \eta_{\mu \alpha} \eta^{\alpha\nu}$. Puisque la métrique de Minkowsky est une invariant de 
Lorentz (par définition), alors $\delta^{\nu}_{\mu}$ l'est aussi.
\end{proof}
%\begin{proof}[Solution alternative]
%Soit $\Lambda$ une transformation de Lorentz. Alors,
%\begin{align}
        %(\delta')^{\nu}_\mu &= \Lambda\indices{^{\nu}_{\alpha}} \Lambda\indices{^{\beta}_{\mu}} \delta^{\beta}_{\alpha} \\
%\end{align}
        
%\end{proof}

\begin{solution}
Le tenseur de Levi-Civita est un invariant de Lorentz.
\end{solution}
\begin{proof}
On applique une transformation de Lorentz sur le tenseur de Levi-Civita
\begin{align*}
        (\epsilon')^{\mu\nu\rho\sigma} = \Lambda\indices{^{\mu}_{\alpha}} \Lambda\indices{^{\nu}_{\beta}} \Lambda\indices{^{\rho}_{\gamma}} \Lambda\indices{^{\sigma}_{\delta}} \epsilon^{\alpha\beta\gamma\delta }\, .
\end{align*} 
On examine maintenant certains éléments du tenseur transformé. Je note $S_n^{+} = \{\sigma \in S_n \mid \mathrm{sgn}(\sigma) = +1\}$ les permutations paires 
pour $n$ indices, où $\mathrm{sgn}(\sigma)$ dénote le signe du déterminant de la matrice associé à la permutation $\sigma$.
Par exemple, $S_3^{+} = \{(1\, 2\, 3),\, (3\, 1\, 2), (2\, 3\, 1)\}$; la matrice associé 
à $\sigma = (3\, 1\, 2)$ est
\begin{equation*}
M_{\sigma} = \begin{pmatrix}
        0 & 0 & 1 \\
        1 & 0 & 0 \\
        0 & 1 & 0
\end{pmatrix}\, .
\end{equation*} 
Elle possède un déterminant positif, $\mathrm{det}(M_\sigma) = 1$, de sortes que $\mathrm{sgn}(\sigma) = +1$. 
L'ensemble des permutations impaires est noté $S_n^{-}$. Naturellement, $|S_n^{+}| = |S_n^{-}| = \frac{n!}{2}$. 
En utilisant la propriété complètement antisymétrique du tenseur de Levi-Civita
\begin{equation}
        \epsilon^{\mu\nu\rho\sigma} = \begin{cases}
                +1 & \mu\nu\rho\sigma \in S_4^{+} \\
                -1 & \mu\nu\rho\sigma \in S_4^{-} \\
                0 & \mathrm{autrement}
        \end{cases}\, ,
\end{equation} 
on trouve que
\begin{equation}
        (\epsilon')^{\mu\nu\rho\sigma} = 
        \sum_{\alpha\beta\gamma\delta \in S_4^{+}} 
 \Lambda\indices{^{\mu}_{\alpha}} \Lambda\indices{^{\nu}_{\beta}} \Lambda\indices{^{\rho}_{\gamma}} \Lambda\indices{^{\sigma}_{\delta}} 
-
        \sum_{\alpha\beta\gamma\delta \in S_4^{-}} 
 \Lambda\indices{^{\mu}_{\alpha}} \Lambda\indices{^{\nu}_{\beta}} \Lambda\indices{^{\rho}_{\gamma}} \Lambda\indices{^{\sigma}_{\delta}} \, .
\end{equation} 

\paragraph{Cas des indices identiques:} Sous cette forme, on peut résoudre les valeurs du tenseur transformé lorsqu'au moins 2 indices sont identiques. Dans ce cas, 
$(\epsilon')^{\mu\nu\rho\sigma} = \epsilon^{\mu\nu\rho\sigma} = 0$. Par exemple, 
considérons le cas $\mu\nu\rho\sigma = 0012$, où les deux premiers indices sont identiques.
\begin{align*}
        (\epsilon')^{0012} &= 
        \sum_{\alpha\beta\gamma\delta \in S_4^{+}} 
 \Lambda\indices{^{0}_{\alpha}} \Lambda\indices{^{0}_{\beta}} \Lambda\indices{^{1}_{\gamma}} \Lambda\indices{^{2}_{\delta}} 
-
        \sum_{\alpha\beta\gamma\delta \in S_4^{-}} 
 \Lambda\indices{^{0}_{\alpha}} \Lambda\indices{^{0}_{\beta}} \Lambda\indices{^{1}_{\gamma}} \Lambda\indices{^{2 }_{\delta}}  \\
                           &= \dots + 
 \Lambda\indices{^{0}_{0}} \Lambda\indices{^{0}_{1}} \Lambda\indices{^{1}_{2}} \Lambda\indices{^{2}_{3}} 
-
\Lambda\indices{^{0}_{1}} \Lambda\indices{^{0}_{0}} \Lambda\indices{^{1}_{2}} \Lambda\indices{^{2}_{3}} 
-\dots \\
&= 0\, .
\end{align*}
En effet, pour chaque permutations $\alpha\beta\gamma\delta \in S_4^{+}$, on peut échanger $\alpha$ et $\beta$, 
de sortes qu'on peut toujours trouver le terme $\beta\alpha\gamma\delta \in S_4^{-}$ dans la somme 
sur les indices impaires qui annule le terme $\alpha\beta\gamma\delta$ dans la première somme.
Cet argument se généralise à tout autre cas où au moins deux indices parmis $\mu\nu\rho\sigma$ sont identiques.

\paragraph{Cas des indices distincts:} 
Je poursuis la démonstration avec l'exemple $\mu\nu\rho\sigma=0123$. Pour le résoudre, on doit démontrer que 
$(\epsilon')^{0123} = \mathrm{det}(\Lambda) = 1$. Le fait qu'une transformation de Lorentz a un déterminant unité suit du fait qu'une transformation de Lorentz doit satisfaire la condition
\begin{equation}
        \Lambda(-v) = \Lambda(v)^{-1}\, ,
\end{equation} 
où $v$ est le paramètre de la transformation (vitesse pour un boost, angle pour une rotation). Puisque 
$\Lambda(-v) = R(\pi)^{T} \Lambda(v) R(\pi) \implies \mathrm{det}(\Lambda(-v)) = \mathrm{det}(\Lambda(v))$, il suit que, 
$\mathrm{det}(\Lambda) = \pm 1$. Finalement, on requiert que $\Lambda(0) = \bbone$, la matrice identité, donc $ \mathrm{det}(\Lambda(0)) = 1\implies \mathrm{det}(\Lambda(v)) = 1$. 

La formule de Leibniz pour le déterminant nous indique que
\begin{equation}\label{eq:Leibniz}
        \mathrm{det}(\Lambda) = \Lambda\indices{^{0}_{\alpha}}\Lambda\indices{^{1}_{\beta}}\Lambda\indices{^{2}_{\gamma}}\Lambda\indices{^{3}_{\delta}}\epsilon^{\alpha\beta\gamma\delta}\, .
\end{equation} 
Or, le côté droit de l'égalité est précisément le résultat de la transformation de Lorentz pour l'élement $0123$ du tenseur de Levi-Civita, 
donc $(\epsilon')^{0123} = \mathrm{det}(\Lambda) = 1$. 
Les autres cas suivent par la permutation des indices contravariant au côté droit de la formule de Leibniz. 
Par exemple, supposons que $\mu\nu\rho\sigma = 1032 \in S_4^{+}$:
\begin{align*}
        (\epsilon')^{1032} &= \Lambda\indices{^{1}_{\alpha}}\Lambda\indices{^{0}_{\beta}}\Lambda\indices{^{3}_{\gamma}}\Lambda\indices{^{2}_{\delta}}\epsilon^{\alpha\beta\gamma\delta}\\
                           &= \Lambda\indices{^{0}_{\beta}}\Lambda\indices{^{1}_{\alpha}}\Lambda\indices{^{2}_{\delta}}\Lambda\indices{^{3}_{\gamma}}\epsilon^{\alpha\beta\gamma\delta}  \hspace{1cm} \{\text{Réarrangement des termes}\} \\
               &= \Lambda\indices{^{0}_{\beta}}\Lambda\indices{^{1}_{\alpha}}\Lambda\indices{^{2}_{\delta}}\Lambda\indices{^{3}_{\gamma}}\epsilon^{\beta\alpha\delta\gamma}  \hspace{1cm} \{\text{Permutation paire des indices du tenseur de Levi-Civita} \} \\
               &= \Lambda\indices{^{0}_{\alpha}}\Lambda\indices{^{1}_{\beta}}\Lambda\indices{^{2}_{\gamma}}\Lambda\indices{^{3}_{\delta}}\epsilon^{\alpha\beta\gamma\delta}  \hspace{1cm} \{\text{Redéfinition des indices factices} \} \\
               &= \mathrm{det}(\Lambda) = 1\, .\\ 
\end{align*} 
Comme l'argument est général, on a que $(\epsilon')^{\mu\nu\rho\sigma} = \epsilon^{\mu\nu\rho\sigma},\,\, \forall \mu\nu\rho\sigma \in S_4^{+}$.
L'argument pour une permutation impaire est très similaire et nous mène à conclure que 
\begin{equation}
(\epsilon')^{\mu\nu\rho\sigma} = -\mathrm{det}(\Lambda) = \epsilon^{\mu\nu\rho\sigma}, \,\, \forall \mu\nu\rho\sigma \in S_4^{-}\, .
\end{equation} 
Donc, ayant couvert tous les cas possible, on conclut que
\begin{equation}
        (\epsilon')^{\mu\nu\rho\sigma} = \epsilon^{\mu\nu\rho\sigma}\, .
\end{equation} 
\end{proof}
        




\section{Invariance d'échelle}
Soit un champ scalaire Klein-Gordon $\phi$ de masse $m$ en $d$ dimensions spatiales. Considérons une transformation continue
\begin{align}
        \label{eq:transformation}
        x' &= bx \\
       \phi'(bx) &= b^{-\Delta}\phi(x)  
\end{align}
où $b,\Delta \in \mathbb{R}_{>0}$.

\subsection{}
\textbf{Quelle sont les conditions pour que \eqref{eq:transformation} soit une symétrie de la théorie? Appelons l’action
de cette théorie $S_{\star}$. Quel est le courant de Noether associé?}
\subsection{}
\textbf{Soit une quantité $\mathcal{O}(x)$ qui dépend du champ et ses dérivées au point $x$. Posons que $\mathcal{O}$ 
transforme comme $\phi$ sous \eqref{eq:transformation}, mais avec $\Delta$ remplacé par $\Delta \mathcal{O}$, appelé la dimension d’échelle
de $\mathcal{O}$. Pour les conditions trouvées en a), quelle est la dimension d’échelle de la densité
Lagrangienne $\mathcal{L}_{\star}$ et de $\phi^{n}$, où $n \in \mathbb{N}$}
\subsection{}
\textbf{On considère le Lagrangien avec un terme d'intéraction}
\begin{equation}\label{eq:Lagrangien2c}
       \mathcal{L}_{\mathrm{int}} = \mathcal{L}_{\star} - \lambda \phi^{2n} 
\end{equation} 
\textbf{où $n \in \{1,2,3,\dots \}$}.
\textbf{Quel doit être le signe de $\lambda$ pour que la théorie soit physiquement 
raisonnable? Quelle est l’équation du mouvement pour cette théorie intéragissante. Quelle est la nouvelle difficulté?}
\subsection{}
\textbf{En $d \in \{1,2,3\}$ dimensions, quelles sont les conditions pour que la théorie intéragissante soit invariante
sous une transformation d’échelle \eqref{eq:transformation}.}



\section{Champs de jauge}
Soit l'action de Maxwell
\begin{equation}\label{eq:MaxwellAction}
       S = \int d^{d+1}x\, F_{\mu\nu}F^{\mu\nu}
\end{equation} 
où $d \geq 1$.
\subsection{}
\textbf{Démontrer que l’action est invariante sous transformations de jauge : $A_{\mu}(x) \rightarrow A_{\mu}(x) +
        \partial_{\mu}f(x)$, où $f$ est une fonction scalaire suffisamment lisse.}
\subsection{}
\subsection{}
\subsection{}
\subsection{}

\section{Phonons}
\subsection{}
\subsection{}
\subsection{}

\section{Quantification 101}

\subsection{}
\subsection{}
\subsection{}
\subsection{}

\end{document}

