\documentclass{article}
\usepackage[a4paper, margin=2cm]{geometry}

\usepackage{amsmath}
\usepackage{amssymb}
\usepackage{mathtools}
\usepackage{amstext}
\usepackage{amsthm}
\usepackage{fancyhdr}
\usepackage[utf8]{inputenc} % allow utf-8 input
\usepackage[T1]{fontenc}    % use 8-bit T1 fonts


\usepackage{graphicx}
\usepackage{float}
\usepackage{caption}
\usepackage{subcaption}
\usepackage{booktabs}
\usepackage{physics, tensor}
\usepackage{slashed}

\graphicspath{{figures/}}

\pagestyle{fancy}
\rhead{Alexandre Adam\\ 20090755}
\lhead{William Witczak-Krempa \\ PHY 6812: Théorie des champs 1}
\chead{Devoir 2}
\rfoot{23 novembre 2022}
\cfoot{\thepage}

\newcommand{\angstrom}{\textup{\AA}}
\numberwithin{equation}{section}
\renewcommand\thesubsection{\alph{subsection})}
\renewcommand\thesubsubsection{\Roman{subsubsection}}
\newcommand{\s}{\hspace{0.1cm}}

\theoremstyle{solution}
\newtheorem{solution}{Réponse}[section]

\renewcommand*{\proofname}{Solution}


\begin{document}
\section{Représentations projectives}
On cherche à montrer que les spineurs de Weyl et Dirac forment des représentations projectives du groupe
de Lorentz. 
En d’autres termes, montrez que ces représentations admettent au moins un 2-cocycle non-trivial.

\section{Charge d’un spineur}
\subsection{}
On considère une transformation interne agissant sur un spineur de Dirac
\begin{equation}\label{eq:t_interne}
	\begin{split}
		x' &\rightarrow  x \\
		\psi(x) &\rightarrow e^{i\theta}\psi(x)
	\end{split}
\end{equation} 
où $\theta \in \mathbb{R}$. On cherche à savoir si cette transformation est une symétrie du Lagrangien de Dirac
\begin{equation}\label{eq:dirac_L}
	\mathcal{L} = \bar{\psi}(i \slashed{\partial} - m^{2})\psi
\end{equation} 

\begin{solution}
	La transformation interne \eqref{eq:t_interne} est une symétrie du Lagrangien de Dirac.	
\end{solution}
\begin{proof}
	
\end{proof}

\noindent
On s'intéresse maintenant aux courants de Noether associés à cette transformation.

\subsection{}
On considère le Lagrangien de Dirac avec masse nulle, $m = 0$. On cherche les symétries de rotation de phase associé à ce Lagrangien.

\noindent
On s'intéresse finalement aux courants de Noether associé aux symétries identifiées précédemment.

\section{Invariance d’échelle}

\section{Théorie Yukawa classique}

\section{Opérateurs qui anti-commutent}

\section{Fermions Marojana (Peskins \& Shcroeder 3.4)}
\end{document}

