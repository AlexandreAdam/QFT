\documentclass{article}
\usepackage[a4paper, margin=2cm]{geometry}

\usepackage[utf8]{inputenc} % allow utf-8 input
\usepackage[T1]{fontenc}    % use 8-bit T1 fonts
\usepackage{hyperref}       % hyperlinks
\hypersetup{
    colorlinks,
    linkcolor={red!50!black},
    citecolor={blue!50!black},
    urlcolor={blue!80!blue}
}
\usepackage{xcolor}
\usepackage{nicefrac}       % compact symbols for 1/2, etc.
\usepackage{bm, bbm}

\usepackage{amsmath}
\usepackage{amssymb}
\usepackage{mathtools}
\usepackage{amstext}
\usepackage{amsthm}
\usepackage{physics}
\usepackage{tensor}

\usepackage{fancyhdr}
\pagestyle{fancy}
\rhead{Alexandre Adam \\ 20090755}
\lhead{Théorie des champs 1 (PHY 6812) \\ William Witczak-Krempa}
\chead{Devoir 1}
\rfoot{\today}
\cfoot{\thepage}

\numberwithin{equation}{section}
\renewcommand\thesubsection{\alph{subsection})}
\renewcommand\thesubsubsection{\Roman{subsubsection}}
\DeclareRobustCommand{\bbone}{\text{\usefont{U}{bbold}{m}{n}1}}

%\newtheorem*{remark}{Remark}
\theoremstyle{solution}
\newtheorem{solution}{Réponse}[section]

\renewcommand*{\proofname}{Solution}

\begin{document}
\section{Un peu de relativité}

\subsection{}
\textbf{Classiquement, un photon se propageant dans le vide peut-il se désintégrer en une paire
électron-positron ? 
%Expliquez votre résultat mathématique de manière heuristique. 
Que ce passera-t-il en théorie quantique des champs (QFT) ?}
\subsection{}
\textbf{Quelle est la condition pour qu’un tenseur soit invariant sous transformations de Lorentz ?}
\vspace{2ex}

Soit un tenseur de rang $(m,n)$, $m,n \in \mathbb{N}$, avec les éléments $T\indices{^{\mu_1\dots \mu_m}_{\nu_1\dots\nu_n}}$, et la métrique de Minkowsky $\eta$. Soit une transformation de 
Lorentz $\Lambda$ définit tel que
\begin{equation}
        \eta_{\mu \nu} = \Lambda\indices{^\alpha_\mu} \Lambda\indices{^\beta_\nu} \eta_{\alpha\beta}
\end{equation} 
Pour que $T$ soit un invariant de Lorentz, il doit satisfaire
\begin{equation}
        \boxed{ T\indices{^{\mu_1\dots\mu_m}_{\nu_1\dots\nu_n}} = \Lambda\indices{^{\mu_1}_{\alpha_1}} \dots \Lambda\indices{^{\mu_m}_{\alpha_m}} 
        \Lambda\indices{^{\beta_1}_{\nu_1}} \dots \Lambda\indices{^{\beta_n}_{\nu_n}} T\indices{^{\alpha_1\dots \alpha_m}_{\beta_1 \dots \beta_n}}} \, ,
\end{equation}
similairement à la métrique de Minkowsky.

\subsection{}
\textbf{Les tenseurs $\delta_{\mu}^{\nu}$ et  $\epsilon^{\mu\nu\rho\sigma}$ (Levi-Civita) sont-ils invariants de Lorentz ?}
\vspace{2ex}
\begin{solution}
        Le delta de Kronecker $\delta_\mu^{\nu}$ est un invariant de Lorentz.
\end{solution}
\begin{proof}
On remarque que $\delta^{\nu}_{\mu} = \eta_{\mu \alpha} \eta^{\alpha\nu}$. Puisque la métrique de Minkowsky est une invariant de 
Lorentz (par définition), alors $\delta^{\nu}_{\mu}$ l'est aussi.
\end{proof}
%\begin{proof}[Solution alternative]
%Soit $\Lambda$ une transformation de Lorentz. Alors,
%\begin{align}
        %(\delta')^{\nu}_\mu &= \Lambda\indices{^{\nu}_{\alpha}} \Lambda\indices{^{\beta}_{\mu}} \delta^{\beta}_{\alpha} \\
%\end{align}
        
%\end{proof}

\begin{solution}
Le tenseur de Levi-Civita est un invariant de Lorentz.
\end{solution}
\begin{proof}
On applique une transformation de Lorentz sur le tenseur de Levi-Civita
\begin{align*}
        (\epsilon')^{\mu\nu\rho\sigma} = \Lambda\indices{^{\mu}_{\alpha}} \Lambda\indices{^{\nu}_{\beta}} \Lambda\indices{^{\rho}_{\gamma}} \Lambda\indices{^{\sigma}_{\delta}} \epsilon^{\alpha\beta\gamma\delta }\, .
\end{align*} 
On examine maintenant certains éléments du tenseur transformé. Je note $S_n^{+} = \{\sigma \in S_n \mid \mathrm{sgn}(\sigma) = +1\}$ les permutations paires 
pour $n$ indices, où $\mathrm{sgn}(\sigma)$ dénote le signe du déterminant de la matrice associé à la permutation $\sigma$.
Par exemple, $S_3^{+} = \{(1\, 2\, 3),\, (3\, 1\, 2), (2\, 3\, 1)\}$; la matrice associé 
à $\sigma = (3\, 1\, 2)$ est
\begin{equation*}
M_{\sigma} = \begin{pmatrix}
        0 & 0 & 1 \\
        1 & 0 & 0 \\
        0 & 1 & 0
\end{pmatrix}\, .
\end{equation*} 
Elle possède un déterminant positif, $\mathrm{det}(M_\sigma) = 1$, de sortes que $\mathrm{sgn}(\sigma) = +1$. 
L'ensemble des permutations impaires est noté $S_n^{-}$. Naturellement, $|S_n^{+}| = |S_n^{-}| = \frac{n!}{2}$. 
En utilisant la propriété complètement antisymétrique du tenseur de Levi-Civita
\begin{equation}
        \epsilon^{\mu\nu\rho\sigma} = \begin{cases}
                +1 & \mu\nu\rho\sigma \in S_4^{+} \\
                -1 & \mu\nu\rho\sigma \in S_4^{-} \\
                0 & \mathrm{autrement}
        \end{cases}\, ,
\end{equation} 
on trouve que
\begin{equation}
        (\epsilon')^{\mu\nu\rho\sigma} = 
        \sum_{\alpha\beta\gamma\delta \in S_4^{+}} 
 \Lambda\indices{^{\mu}_{\alpha}} \Lambda\indices{^{\nu}_{\beta}} \Lambda\indices{^{\rho}_{\gamma}} \Lambda\indices{^{\sigma}_{\delta}} 
-
        \sum_{\alpha\beta\gamma\delta \in S_4^{-}} 
 \Lambda\indices{^{\mu}_{\alpha}} \Lambda\indices{^{\nu}_{\beta}} \Lambda\indices{^{\rho}_{\gamma}} \Lambda\indices{^{\sigma}_{\delta}} \, .
\end{equation} 

\paragraph{Cas des indices identiques:} Sous cette forme, on peut résoudre les valeurs du tenseur transformé lorsqu'au moins 2 indices sont identiques. Dans ce cas, 
$(\epsilon')^{\mu\nu\rho\sigma} = \epsilon^{\mu\nu\rho\sigma} = 0$. Par exemple, 
considérons le cas $\mu\nu\rho\sigma = 0012$, où les deux premiers indices sont identiques.
\begin{align*}
        (\epsilon')^{0012} &= 
        \sum_{\alpha\beta\gamma\delta \in S_4^{+}} 
 \Lambda\indices{^{0}_{\alpha}} \Lambda\indices{^{0}_{\beta}} \Lambda\indices{^{1}_{\gamma}} \Lambda\indices{^{2}_{\delta}} 
-
        \sum_{\alpha\beta\gamma\delta \in S_4^{-}} 
 \Lambda\indices{^{0}_{\alpha}} \Lambda\indices{^{0}_{\beta}} \Lambda\indices{^{1}_{\gamma}} \Lambda\indices{^{2 }_{\delta}}  \\
                           &= \dots + 
 \Lambda\indices{^{0}_{0}} \Lambda\indices{^{0}_{1}} \Lambda\indices{^{1}_{2}} \Lambda\indices{^{2}_{3}} 
-
\Lambda\indices{^{0}_{1}} \Lambda\indices{^{0}_{0}} \Lambda\indices{^{1}_{2}} \Lambda\indices{^{2}_{3}} 
-\dots \\
&= 0\, .
\end{align*}
En effet, pour chaque permutations $\alpha\beta\gamma\delta \in S_4^{+}$, on peut échanger $\alpha$ et $\beta$, 
de sortes qu'on peut toujours trouver le terme $\beta\alpha\gamma\delta \in S_4^{-}$ dans la somme 
sur les indices impaires qui annule le terme $\alpha\beta\gamma\delta$ dans la première somme.
Cet argument se généralise à tout autre cas où au moins deux indices parmis $\mu\nu\rho\sigma$ sont identiques.

\paragraph{Cas des indices distincts:} 
Je poursuis la démonstration avec l'exemple $\mu\nu\rho\sigma=0123$. Pour le résoudre, on doit démontrer que 
$(\epsilon')^{0123} = \mathrm{det}(\Lambda) = 1$. Le fait qu'une transformation de Lorentz a un déterminant unité suit du fait qu'une transformation de Lorentz doit satisfaire la condition
\begin{equation}
        \Lambda(-v) = \Lambda(v)^{-1}\, ,
\end{equation} 
où $v$ est le paramètre de la transformation (vitesse pour un boost, angle pour une rotation). Puisque 
$\Lambda(-v) = R(\pi)^{T} \Lambda(v) R(\pi) \implies \mathrm{det}(\Lambda(-v)) = \mathrm{det}(\Lambda(v))$, il suit que, 
$\mathrm{det}(\Lambda) = \pm 1$. Finalement, on requiert que $\Lambda(0) = \bbone$, la matrice identité, donc $ \mathrm{det}(\Lambda(0)) = 1\implies \mathrm{det}(\Lambda(v)) = 1$. 

La formule de Leibniz pour le déterminant nous indique que
\begin{equation}\label{eq:Leibniz}
        \mathrm{det}(\Lambda) = \Lambda\indices{^{0}_{\alpha}}\Lambda\indices{^{1}_{\beta}}\Lambda\indices{^{2}_{\gamma}}\Lambda\indices{^{3}_{\delta}}\epsilon^{\alpha\beta\gamma\delta}\, .
\end{equation} 
Or, le côté droit de l'égalité est précisément le résultat de la transformation de Lorentz pour l'élement $0123$ du tenseur de Levi-Civita, 
donc $(\epsilon')^{0123} = \mathrm{det}(\Lambda) = 1$. 
Les autres cas suivent par la permutation des indices contravariant au côté droit de la formule de Leibniz. 
Par exemple, supposons que $\mu\nu\rho\sigma = 1032 \in S_4^{+}$:
\begin{align*}
        (\epsilon')^{1032} &= \Lambda\indices{^{1}_{\alpha}}\Lambda\indices{^{0}_{\beta}}\Lambda\indices{^{3}_{\gamma}}\Lambda\indices{^{2}_{\delta}}\epsilon^{\alpha\beta\gamma\delta}\\
                           &= \Lambda\indices{^{0}_{\beta}}\Lambda\indices{^{1}_{\alpha}}\Lambda\indices{^{2}_{\delta}}\Lambda\indices{^{3}_{\gamma}}\epsilon^{\alpha\beta\gamma\delta}  \hspace{1cm} \{\text{Réarrangement des termes}\} \\
               &= \Lambda\indices{^{0}_{\beta}}\Lambda\indices{^{1}_{\alpha}}\Lambda\indices{^{2}_{\delta}}\Lambda\indices{^{3}_{\gamma}}\epsilon^{\beta\alpha\delta\gamma}  \hspace{1cm} \{\text{Permutation paire des indices du tenseur de Levi-Civita} \} \\
               &= \Lambda\indices{^{0}_{\alpha}}\Lambda\indices{^{1}_{\beta}}\Lambda\indices{^{2}_{\gamma}}\Lambda\indices{^{3}_{\delta}}\epsilon^{\alpha\beta\gamma\delta}  \hspace{1cm} \{\text{Redéfinition des indices factices} \} \\
               &= \mathrm{det}(\Lambda) = 1\, .\\ 
\end{align*} 
Comme l'argument est général, on a que $(\epsilon')^{\mu\nu\rho\sigma} = \epsilon^{\mu\nu\rho\sigma},\,\, \forall \mu\nu\rho\sigma \in S_4^{+}$.
L'argument pour une permutation impaire est très similaire et nous mène à conclure que 
\begin{equation}
(\epsilon')^{\mu\nu\rho\sigma} = -\mathrm{det}(\Lambda) = \epsilon^{\mu\nu\rho\sigma}, \,\, \forall \mu\nu\rho\sigma \in S_4^{-}\, .
\end{equation} 
Donc, ayant couvert tous les cas possible, on conclut que
\begin{equation}
        (\epsilon')^{\mu\nu\rho\sigma} = \epsilon^{\mu\nu\rho\sigma}\, .
\end{equation} 
\end{proof}
        




\section{Invariance d'échelle}
Soit un champ scalaire Klein-Gordon $\phi$ de masse $m$ en $d$ dimensions spatiales. Considérons une transformation continue
\begin{align}
        \label{eq:transformation}
        x &\rightarrow bx \\
        \phi(x) &\rightarrow b^{-\Delta}\phi(x)  
\end{align}
où $b \in \mathbb{R}_{>0}$ et $\Delta \in \mathbb{R}$.

\subsection{}
\textbf{Quelle sont les conditions pour que \eqref{eq:transformation} soit une symétrie de la théorie? Appelons l’action
de cette théorie $S_{\star}$. Quel est le courant de Noether associé?}
\vspace{2ex}

La transformation \eqref{eq:transformation} est une symétrie de la théorie si l'action 
\begin{equation}\label{eq:action}
        S_\star = \int d^{d+1}x\, \mathcal{L}_\star(\phi(x), \partial_{\mu}\phi(x))
\end{equation} 
est invariante sous l'application de la transformation. Puisque l'élément de volume devient
\begin{equation}
        d^{d+1}x \rightarrow b^{d+1}d^{d+1}x\, ,
\end{equation} 
alors la transformation est une symétrie de la théorie si et seulement si
\begin{equation}
       \mathcal{L}_{\star} \rightarrow b^{-(d+1)}\mathcal{L_{\star}} + \partial_{\mu}\mathcal{J}^{\mu}\, .
\end{equation} 
La divergence $\partial_{\mu}\mathcal{J}^{\mu}$ devient un terme de surface dans l'intégrale de l'action, qui 
s'annule par définition de la dérivée fonctionnelle de l'action. Puisque $\partial_{\mu} \rightarrow b^{-1}\partial_{\mu}$, alors 
la densité du Lagrangien de Klein-Gordon transformée devient
\begin{equation}\label{eq:lagrangian transform}
        \mathcal{L}_\star' = b^{-2-2\Delta}\frac{1}{2}(\partial_{\mu}\phi)(\partial^\mu \phi) - b^{-2\Delta}\frac{1}{2}m^{2}\phi^{2}\, .
\end{equation} 
Ainsi, la transformation d'échelle \eqref{eq:transformation} est une symétrie de la théorie seulement lorsque $\boxed{m=0 \text{ et } \Delta = \frac{d-1}{2}}$.

Le courant de Noether est définit comme
\begin{equation}
        j^{\mu} = \frac{\partial \mathcal{L}}{\partial (\partial_{\mu}\phi)} \delta_\alpha  \phi  - \mathcal{J}^{\nu}.
\end{equation} 
Par l'éq.~\eqref{eq:lagrangian transform}, on trouve que $\mathcal{J}^{\nu} = 0$. On cherche maintenant la variation infinitésimale 
du champ, $\delta_{\alpha} \phi$, en terme du paramètre de la transformation infinitésimale, $\alpha$. On écrit cette variation 
infinitésimale comme une transformation active de $\phi$:
\begin{equation}
       \alpha\delta_\alpha \phi = \phi'(x) - \phi(x) 
\end{equation} 
On pose $b = 1 - \alpha$, de sortes que
\begin{align*}
        \phi'(x) &= b^{-\Delta}\phi(b^{-1}x) \\
              &= (1 + \alpha\Delta) \phi(x + \alpha x) + \mathcal{O}(\alpha^{2}) \\
              &= \phi(x) + \alpha(\Delta + x^{\mu}\partial_{\mu})\phi(x) + \mathcal{O}(\alpha^{2})
\end{align*}
On trouve finalement
\begin{equation}\label{eq:Noether current}
        \boxed{j^{\mu}_\star = \partial^{\mu} \phi(\Delta + x^{\mu}\partial_{\mu})\phi }
\end{equation} 

\subsection{}
\textbf{Soit une quantité $\mathcal{O}(x)$ qui dépend du champ et ses dérivées au point $x$. Posons que $\mathcal{O}$ 
transforme comme $\phi$ sous \eqref{eq:transformation}, mais avec $\Delta$ remplacé par $\Delta_\mathcal{O}$, appelé la dimension d’échelle
de $\mathcal{O}$. Pour les conditions trouvées en a), quelle est la dimension d’échelle de la densité
Lagrangienne $\mathcal{L}_{\star}$ et de $\phi^{n}$, où $n \in \mathbb{N}$}.
\vspace{2ex}

Avec les conditions trouvées en \textbf{a)}, on a $\boxed{\Delta_{\mathcal{L}_\star} = d+1}$ et $\boxed{\Delta_{\phi^{n}} = \frac{n(d-1)}{2}}$

\subsection{}
\textbf{On considère le Lagrangien avec un terme d'interaction}
\begin{equation}\label{eq:Lagrangien2c}
       \mathcal{L}_{\mathrm{int}} = \mathcal{L}_{\star} - \lambda \phi^{2n} 
\end{equation} 
\textbf{où $n \in \mathbb{N}$}.
\textbf{Quel doit être le signe de $\lambda$ pour que la théorie soit physiquement 
raisonnable? Quelle est l’équation du mouvement pour cette théorie interagissante. Quelle est la nouvelle difficulté?}

\vspace{2ex}

Le signe de $\lambda $ gouverne la forme du potentiel $V(\phi) = \frac{1}{2}m^{2}\phi^{2} + \lambda \phi^{2n}$. 
Dans le cas où $\lambda > 0$, alors le potentiel possède un état fondamental avec une énergie minimale, où $\phi = 0$. 
Dans le cas où $\lambda < 0$, alors on peut, en principe, avoir des états avec une énergie arbitrairement négative.
Donc il n'y a pas d'état fondamental. Ainsi, on doit avoir $\lambda \in \mathbb{R}_{>0}$.
L'équation du mouvement du champ $\phi(x)$ gouverné par $\mathcal{L}_{\mathrm{int}}$ est
\begin{equation}
        \boxed{\Box\phi + \frac{\partial V}{\partial \phi} = \partial^\mu \partial_{\mu} \phi + m^2\phi + 2n\lambda \phi^{2n-1} = 0}
\end{equation} 
La difficulté est de résoudre cette équation différentielle pour $\phi$, qui n'est plus une combination linéaire d'ondes planes. 

\subsection{}
\textbf{En $d \in \{1,2,3\}$ dimensions, quelles sont les conditions pour que la théorie interagissante soit invariante
sous une transformation d’échelle \eqref{eq:transformation}.}
\vspace{2ex}

Le terme d'interaction rajouté, $V_{\mathrm{\phi}}(\phi) = \lambda \phi^{2n}$, se transforme comme
\begin{equation}
        V_{\mathrm{int}} \rightarrow b^{-2n\Delta}V_{\mathrm{int}}
\end{equation} 
Pour que le Lagrangien se transforme comme $\mathcal{L}_{\mathrm{int}} \rightarrow b^{-2-2\Delta}\mathcal{L}_{\mathrm{int}}$, on doit avoir $m=0$ et 
\begin{equation}
        n = \frac{\Delta + 1}{\Delta}
\end{equation} 
En utilisant la condition dérivée en \textbf{a)} pour que la transformation d'échelle soit une symétrie de la théorie, soit que $\Delta = \frac{d - 1}{2}$, on trouve
\begin{equation}
        n = \frac{d}{d - 1} 
\end{equation} 
Ainsi, il n'y a pas de symétrie d'échelle pour les théories interagissantes avec $d=1$ dimension, puisque $n = \infty \not\in \mathbb{N}$. 
Pour $d=2$, la théorie interagissante avec $\boxed{n=1 \text{ et } m=0}$ possède une symétrie d'échelle. 
Il n'y a pas de symétrie d'échelle pour les théorie interagissante avec $d=3$ dimension, puisque $n = \frac{3}{2} \not\in \mathbb{N}$ n'est pas 
une théorie physique.

\section{Champs de jauge}
Soit l'action de Maxwell
\begin{equation}\label{eq:MaxwellAction}
       S = \int d^{d+1}x\, F_{\mu\nu}F^{\mu\nu}
\end{equation} 
où $d \geq 1$.
\subsection{}
\textbf{Démontrer que l’action est invariante sous transformations de jauge : $A_{\mu}(x) \rightarrow A_{\mu}(x) +
\partial_{\mu}f(x)$, où $f$ est une fonction scalaire suffisamment lisse.}
\vspace{2ex}

%Par définition du tenseur électromagnétique
%\begin{equation}\label{eq:F}
        %F_{\mu\nu} = \partial_{\mu}A_\nu - \partial_{\nu}A_\mu\, ,
%\end{equation} 
%la transformation de jauge induit 
%\begin{equation}
        %F_{\mu\nu} \rightarrow F_{\mu \nu} + \partial_{\mu}\partial_{\nu}f - \partial_{\nu}\partial_{\mu}f\, .
%\end{equation} 
\begin{proof}
Soit le Lagrangien de Maxwell
\begin{equation}
        \mathcal{L} = -\frac{1}{4}F^{\mu \nu}F_{\mu\nu} %= -\frac{1}{2}(\partial^{\mu}A^{\nu})(\partial_{\nu}A_\mu) + \frac{1}{2}(\partial_{\mu}A^{\mu})^{2}
\end{equation} 
où $F_{\mu\nu}$ est le tenseur électromagnétique % It is antisymmetric, with a nul trace
\begin{equation}\label{eq:F}
        F_{\mu\nu} = \partial_{\mu}A_\nu - \partial_{\nu}A_\mu\, .
\end{equation} 
Pour déterminer si la transformation de l'action est invariante sous la transformation de jauge, on doit s'assurer que $\delta S = 0$.
On commence par étudier la variation infinitésimal du Lagrangien
\begin{equation}
        \delta \mathcal{L} = -\frac{1}{2}F^{\mu \nu} \delta F_{\mu \nu}
\end{equation} 
La variation infinitésimale de $F_{\mu \nu}$ sous la transformation de jauge est
\begin{align*}
        \delta F_{\mu \nu} &= \partial_{\mu}\delta A_{\nu} - \partial_{\nu} \delta A_\mu \\
        &= \partial_{\mu}\partial_\nu f - \partial_{\nu} \partial_\mu f \\
        &= 0\, .
\end{align*}
La dernière ligne est une conséquence du théorème de Schwarz, qui dicte que les dérivée secondes d'une fonction scalaire sont symétriques dans une région $\Omega \subset \mathbb{R}^{d+1}$ qui 
contient un point $\mathcal{O} \in \Omega$ où la fonction $f:\Omega \rightarrow \mathbb{R}$ est lisse. Il suit que l'action est invariante sous la transformation de jauge, 
$\delta S = \int d^{d+1} \delta L = 0$.
\end{proof}

\subsection{}
\textbf{Démontrer que l’action est invariante sous transformations de Lorentz. Est-ce qu’il existe
un terme de masse pour le champ $A_{\mu}$ qui serait invariant de jauge et invariant de Lorentz?}
\vspace{2ex}
\begin{proof}
Soit la transformation de Lorentz
\begin{equation}
        x^{\mu} \rightarrow \Lambda\indices{^\mu_\nu} x^\nu
\end{equation} 
On étudie maintenant comment le Lagrangien se transforme. Pour se faire, on doit déterminer comment une 
dérivée se transforme. On trouve
\begin{equation}
        \partial_\mu \rightarrow \partial_\mu' = \frac{\partial }{\partial (\Lambda\indices{^\mu_\nu} x^\nu)}
\end{equation} 
        
\end{proof}
\subsection{}
\textbf{Déterminer les équations du mouvement de la théorie. Est-ce qu’elles correspondent aux
équations classiques de Maxwell? Si oui, sous quelles conditions?}
\vspace{2ex}
\subsection{}
\textbf{Est-ce que la théorie de Maxwell est invariante sous une transformation d’échelle? Si
oui, quelle sont les conditions appropriées, ainsi que les dimensions d’échelle du champ de jauge
et du champ électrique.}
\vspace{2ex}
\subsection{}
\textbf{Quel est le champ canoniquement conjugué à $A_\mu$ . 
\vspace{2ex}
%Est-ce qu’il y a quelque chose de bizarre
%avec votre réponse ?
}

\section{Phonons}
\subsection{}
\subsection{}
\subsection{}

\section{Quantification 101}

\subsection{}
\subsection{}
\subsection{}
\subsection{}

\end{document}

